%% LyX 2.3.0 created this file.  For more info, see http://www.lyx.org/.
%% Do not edit unless you really know what you are doing.
\documentclass[twocolumn]{article}
\usepackage[latin9]{inputenc}
\usepackage{geometry}
\geometry{verbose,tmargin=3cm,bmargin=3cm,lmargin=3cm,rmargin=3cm}
\usepackage{float}
\usepackage{booktabs}
\usepackage{amsmath}

\makeatletter

%%%%%%%%%%%%%%%%%%%%%%%%%%%%%% LyX specific LaTeX commands.
%% Because html converters don't know tabularnewline
\providecommand{\tabularnewline}{\\}

%%%%%%%%%%%%%%%%%%%%%%%%%%%%%% User specified LaTeX commands.
\usepackage{titlesec}

\titleformat*{\section}{\normalsize\bfseries}

\makeatother

\begin{document}
\twocolumn[   
\begin{@twocolumnfalse}

\title{On the interpretation of the material conditional and the intuitive
formalisation of implication by logical consequence}

\author{Christopher G. Albert}
\maketitle
\begin{abstract}
The material conditional $a\rightarrow b\equiv\neg a\lor b$ is a
basic definition in formal logic and commonly used in mathematics.
Minimalistic or sloppy translation to natural language has led to
substantial confusion which can be restored either by making the translation
precise or by a change in the logical formalism. Both involve leaving
the result open in case the premise $a$ is false. Separating this
\emph{irrelevant }case recovers a consistent formalisation of logical
consequence \emph{``If a, then b.'' }at the cost of ambiguity. Irrelevance
is one way to introduce a paraconsistent logic and is realised in
particular in the \emph{Logic of Paradox} and \emph{Relevance Logic
}(see \emph{G. Priest. An Introduction to Non-Classical Logic: From
If to Is. }Cambridge, 2008\emph{ }\cite{priest2008introduction}).
The discussed definition of implication breaks the principle of absolute
logical equivalence by making it dependent on the truth value of a
statement. For the case of an implication being true, logical equivalence
and proofs remain classical, including indirect proofs. If an implication
is false, differences appear due to ambiguity by irrelevant cases.
\end{abstract}
\end{@twocolumnfalse}
]

\section*{Material consequence and its interpretation}

In mathematics the classical material consequence or conditional 
\begin{equation}
a\rightarrow b\equiv\neg a\lor b
\end{equation}
is often spoken as \emph{``If a, then b.'',} ``\emph{a implies
b.'', ``b follows from a.'' }or \emph{``a only if b.''} which,
as we will see, can fail to capture its actual meaning\emph{. }Values
of $a\rightarrow b$ and its descendents depending on truth values
of $a$ and $b$ are shown in Table \ref{tab:Classical-truth-table}.

\begin{table}[h]
\begin{centering}
\begin{tabular}{ccccc}
\toprule 
\emph{a}  & \emph{b}  & $a\rightarrow b$ & $\lnot(a\rightarrow b)$ & $a\leftrightarrow b$\tabularnewline
\midrule
\midrule 
\emph{F } & \emph{F } & \emph{T} & \emph{F} & \emph{T}\tabularnewline
\midrule 
\emph{F } & \emph{T } & \emph{T} & \emph{F} & \emph{F}\tabularnewline
\midrule 
\emph{T } & \emph{F } & \emph{F} & \emph{T} & \emph{F}\tabularnewline
\midrule 
\emph{T } & \emph{T } & \emph{T} & \emph{F} & \emph{T}\tabularnewline
\bottomrule
\end{tabular}
\par\end{centering}
\caption{Classical truth table of material consequence, its negation, and equivalence.\label{tab:Classical-truth-table}}
\end{table}

An intuitive interpretation of the material conditional translated
to words runs into trouble for the case $a=F$ leading to $(a\rightarrow b)=T$,
independent from $b$. A well-known feature of (ab-)using the material
conditional as a rule of implication is the principle of explosion,
i.e. a single false statement in a formal system can be used to prove
arbitrary statements, thus destroying the system. This will be resolved
at the end of the section.

Despite its counterintuitiveness, an advantage of setting $(a\rightarrow b)=T$
for $A=F$ lies in the simplification of proof. To show that $(a\rightarrow b)=T$
holds, it is enough to show that $a=T$ and $b=F$ cannot be realised
at the same time. Therefore, to show the universal truth of $a\rightarrow b$,
it is enough to show it under the condition that $a=T$, as it is
true per definition for $a=F$. This also eases proofs of equivalence
\begin{equation}
a\leftrightarrow b\equiv(a\rightarrow b)\land(b\rightarrow a).\label{eq:equiv}
\end{equation}
 In order to show $(a\leftrightarrow b)=T$ it is enough to show truth
of material consequences in both directions, thereby also covering
the case $(a,b)=(F,F)$ per default.

This seemingly advantage turns into a disadvantage for the negation
of the material consequence,
\begin{equation}
\lnot(a\rightarrow b)\equiv a\land\neg b,
\end{equation}
which would translate to \emph{``If a, then not b''} when looking
only at the last two columns of Table \ref{tab:Classical-truth-table}.
To prove $\lnot(a\rightarrow b)=T$ (i.e. $(a\rightarrow b)=F$) we
need not only exclude the possibility $(a,b)=(T,T)$, but also avoid
$a=F$, so prove that $a=T$. The latter refers to $a$ alone and
should intuitively have nothing to do with a relation between $a$
and $b$. 

The mentioned issues are resolved if language is made precise. A minimalistic
way to talk about the material conditional would be directly based
on its definition\footnote{Here \emph{``a is false''} is intentionally used instead of \emph{``not
a''} and \emph{``b is true''} instead of \emph{``b''} alone,
as the truth values are respectively identical, but the language is
more clear in the present version.},\emph{
\begin{equation}
a\rightarrow b\equiv\neg a\lor b:\text{\textquotedblleft a is false, or b is true.\textquotedblright}\label{eq:material-1}
\end{equation}
}and for its negation 
\begin{equation}
\lnot(a\rightarrow b)\equiv a\land\neg b:\emph{\text{\textquotedblleft a is true, and b is false.\textquotedblright}}\label{eq:antimaterial-1}
\end{equation}
Here we see that in particular the negation of the material conditional
intuitively \emph{assigns} truth values to $a$ and $b$. The statement
$\lnot(a\rightarrow b)$ doesn't describe a relation, but rather two
facts.

For an interpretation of $a\rightarrow b$ as a consequence \emph{``If
a, then b}'', one could argue that the case $a=F$ is implicitly
included by the law of the excluded middle, resulting in a true statement
per default. As this can lead to misinterpretation, a slightly more
redunant, but inuitive translation of the material consequence into
words should be\emph{
\begin{equation}
a\rightarrow b:\text{\textquotedblleft If a, then b, or a is false.\textquotedblright}\label{eq:material}
\end{equation}
}and its negation\footnote{Of course the following sentence is a rather complicated way to describe
\eqref{eq:antimaterial-1}, but it is useful, as it allows its last
part to be omitted. } 
\begin{equation}
\lnot(a\rightarrow b):\,\emph{\text{\textquotedblleft If a, then not b, and a is true.\textquotedblright}}\label{eq:antimaterial}
\end{equation}
Thus we could translate the meaning of the material conditional into
words by making the extra conditions precise that we have omitted
before.

The explosion problem is resolved at least intuitively, since the
addition of \emph{``or a is false''} when speaking of $a\rightarrow b$
in \eqref{eq:material} exactly says that if $a$ is false, so we
don't care about the value of $b$. Only the omission of the last
part of \eqref{eq:antimaterial} leads to the incorrect interpretation
that we could \emph{conclude} on the truth of $b$ from the material
conditional. The material conditional says nothing about the truth
value of $b$ if $a=F$, but only of the overall expression $(a\rightarrow b)\equiv(\lnot a\lor b)=T$.
Explosion doesn't occur, as any chain of proofs containing a false
premise in a material conditional says nothing about the truth of
the conclusion. It is misleading and unintuitive to define a formal
proof by checking if $a\rightarrow b$ is true, as one would intuitively
believe to have proven $b$ in this manner. The latter is only true
for $a=T$, but not for $a=F$. This must be kept in mind when using
``$\rightarrow$'' as a rule of inference.

\section*{Logical consequence and irrelevance}

The other possibility to fit intuitive language and logic consists
in changing to a logical rule $\Rightarrow$ different from the material
conditional $\rightarrow$ to represent the implication \emph{``If
a, then b.''} This can be done by starting with $\rightarrow$ and
postulating a false premise \emph{$a=F$} doesn't matter. Introducing
this ambiguity restores an intuitive definition of implication by
the logical consequence\footnote{often spoken as ``\emph{a entails b.'' }and denoted by $\models$
in literature, but we intentionally keep the symbol $\Rightarrow$
which is often used with a logical, rather than a material consequence
in mind.} $a\Rightarrow b$. In such a case we say that the combinations $(a,b)$
are \emph{irrelevant} for $a\Rightarrow b$ and denote it with $I$
rather than $T$ or $F$. For logical primitives $\land$,$\lor$
involving $I$ we introduce rules in Table~\ref{tab:Classical-truth-table-1-1}
together with $\lnot I=I$.

\begin{table}[h]
\begin{centering}
\begin{tabular}{cccc}
\toprule 
\emph{a}  & \emph{b}  & $a\lor b$ & $a\land b$\tabularnewline
\midrule
\midrule 
\emph{F } & \emph{I } & \emph{I} & \emph{F}\tabularnewline
\midrule
\emph{T} & \emph{I } & \emph{T} & \emph{I}\tabularnewline
\midrule
\emph{I} & \emph{F} & \emph{I} & \emph{F}\tabularnewline
\midrule
\emph{I} & \emph{T} & \emph{T} & \emph{I}\tabularnewline
\bottomrule
\end{tabular}
\par\end{centering}
\caption{Logical primitives involving irrelevance.\label{tab:Classical-truth-table-1-1}}
\end{table}
The only cases \emph{relevant} for $a\Rightarrow b$, where a fixed
truth value should be assigned are the ones with $a=T$. After distinguishing
irrelevant cases with $a=F$, the logical consequence $a\Rightarrow b$
can be written inside the truth table~\ref{tab:Classical-truth-table-1}.

\begin{table}[h]
\begin{centering}
\begin{tabular}{cccccc}
\toprule 
\emph{a}  & \emph{b}  & $a\Rightarrow b$ & $\lnot(a\Rightarrow b)$ & $a\Leftrightarrow b$ & $\lnot b\Rightarrow\lnot a$\tabularnewline
\midrule
\midrule 
\emph{F } & \emph{F } & \emph{I} & \emph{I} & \emph{I} & \emph{T}\tabularnewline
\midrule 
\emph{F } & \emph{T } & \emph{I} & \emph{I} & \emph{F} & \emph{I}\tabularnewline
\midrule 
\emph{T } & \emph{F } & \emph{F} & \emph{T} & \emph{F} & \emph{F}\tabularnewline
\midrule 
\emph{T } & \emph{T } & \emph{T} & \emph{F} & \emph{T} & \emph{I}\tabularnewline
\bottomrule
\end{tabular}
\par\end{centering}
\caption{Logical implications with irrelevance.\label{tab:Classical-truth-table-1}}
\end{table}
The material conditional and its negation can then be recovered by
adding truth values for $a$ via
\begin{align}
a & \rightarrow b\equiv(a\Rightarrow b)\lor\lnot a,\\
\lnot(a & \rightarrow b)\equiv(a\Rightarrow b)\lor\lnot a,
\end{align}

This also corresponds exactly to the verbose way of translating the
material conditional into words in (\ref{eq:material}-\ref{eq:antimaterial}).

Weak equivalence 
\begin{equation}
a\Leftrightarrow b\equiv(a\Rightarrow b)\land(b\Rightarrow a)
\end{equation}
can be considered a mixture of $a\land b$ and $a\rightarrow b$ with
only $a=T$ or $b=T$ being relevant, but not $(a,b)=(F,F)$. This
reflects the fact that a proof from either direction by a logical
consequence takes at least one to be true, and we didn't assign a
fixed truth value for the case $(a,b)=(F,F)$. For strong equivalence
$(a\leftrightarrow b)$ we keep the classical definition \eqref{eq:equiv}
leading to values of Table~\ref{tab:Classical-truth-table}. It is
easily re-written in terms of weak equivalence by
\begin{equation}
a\leftrightarrow b\equiv(a\Leftrightarrow b)\land(\lnot a\Leftrightarrow\lnot b).
\end{equation}
Thus, to show \emph{strong }equivalency $\leftrightarrow$ of two
statements it is not enough to prove ``$\Rightarrow$'' from both
directions. Remember that when constructing a proof for $a\leftrightarrow b$
from both directions of $\rightarrow$, any sloppiness in language
could be masked by the ``dirty trick'' of setting the default value
$a\rightarrow b=T$ in case $a=F$. If this is replaced by irrelevance,
the verification of both, $a\Rightarrow b$ and $b\Rightarrow a$
only states that if either $a=T$ or $b=T$, no contradiction to $a\leftrightarrow b$
arises. The case $(a,b)=(F,F)$ is irrelevant to both directions,
and stays so when they are combined. 

\section*{Example}

Let us analyse the unintuitive behavior of the material conditional
and its resolution by an example,

$a$: \emph{``The moon is (entirely) made of cheese.''} and 

$b$: ``\emph{My birthday is on date }$x$.'' 

Statement $a$ is false\footnote{as in \emph{``It has been checked that at least parts of the moon
are rocks, and we know rocks are not cheese.''}}, and $b$ may be either true or false, depending on $x$. According
to the usual translation of material conditional, the statement

\emph{``If the moon is made of cheese, then my birthday is on date
$x$.''}

is true, independent from \emph{x}. This sentence may be interpreted
as paradoxical as soon as you see me celebrating every day because
we started with a false premise. The two possible resolutions are 
\begin{enumerate}
\item The correct translation of the material conditional into words as\emph{}\\
\emph{``The moon is not made of cheese, or my birthday is on any
date $x$ you can imagine'',}\\
or\emph{}\\
\emph{``If the moon is made of cheese, then my birthday is on date
$x$, or the moon is not made of cheese'',}\\
which are perfectly true statements.
\item The introduction of irrelevant cases for a false premise, in which
case\\
\emph{``If the moon is made of cheese, then my birthday is on date
$x$''}\\
becomes an irrelevant statement which can be either true or false.
\end{enumerate}
It should be noted that if we interpret 

\emph{``If the moon is made of cheese, then I'll eat my hat (, or
the moon is not made of cheese).'' }

in the sense of a material conditional including the part in brackets,
I am actually saying\emph{}\\
\emph{``The moon is not made of cheese, or I'll eat my hat''.}\\
In the material sense this statement is true, in the logical sense
it is irrelevant without brackets, and true with brackets. Both variants
leave me free choice whether or not to eat my hat, since \emph{``the
moon is made of cheese'' }is false. 

\section*{Representing statements by logical rules}

For a more precise definition of logical equivalence and formal proof
of statements in a system including irrelevance we will now split
the system into cases where logical rules apply. This also helps to
illustrate how equivalence between two statements becomes dependent
on their truth value. An insightful way to view the concept of irrelevance
is by separation of tables of allowed and forbidden combinations for
the cases of a statement being true or false, respectively. The case
$(a\Rightarrow b)=T$ is shown in Table~ \ref{tab:Logical-rules-for}.

\begin{table}[h]
\begin{centering}
\begin{minipage}[t]{0.4\columnwidth}%
\begin{center}
\textbf{allowed}
\par\end{center}
\begin{center}
\begin{tabular}{cc}
\toprule 
a  & b\tabularnewline
\midrule
\midrule 
\emph{F } & \emph{F}\tabularnewline
\midrule 
\emph{F } & \emph{T}\tabularnewline
\midrule 
\emph{T } & \emph{T}\tabularnewline
\bottomrule
\end{tabular}
\par\end{center}%
\end{minipage}%
\begin{minipage}[t]{0.4\columnwidth}%
\begin{center}
\textbf{forbidden}
\par\end{center}
\begin{center}
\begin{tabular}{cc}
\toprule 
a  & b\tabularnewline
\midrule
\midrule 
\emph{T} & \emph{F}\tabularnewline
\bottomrule
\end{tabular}
\par\end{center}%
\end{minipage}
\par\end{centering}
\caption{Rules for $(a\Rightarrow b)=T\equiv(a\rightarrow b)=T\equiv(\protect\lnot b\rightarrow\protect\lnot a)=T$.\label{tab:Logical-rules-for}}
\end{table}

Rules for $(a\Rightarrow b)=T$ are the same as for $(a\rightarrow b)=T$,
leading to their equivalence in that case. Combinations with false
premise $a=F$ are allowed for $(a\Rightarrow b)=T$, as this case
is irrelevant for the truth of $a\Rightarrow b$, and are true per
definition for $(a\rightarrow b)=T$. For a true premise $a=T$, only
the case of a true consequence $b=T$ is allowed, but not $b=F$.
From the perspective of rules of allowed and forbidden combinations
of $a$ and $b$, the statement $(a\Rightarrow b)=T$ is further equivalent
the contrapositive $(\lnot b\Rightarrow\lnot a)=T$.

The negation $(a\Rightarrow b)=F$ ($\equiv$ $\lnot(a\Rightarrow b)=T$)
has rules listed in Table~\ref{tab:Logical-rules-for-1}. Here the
two additional irrelevant cases with $a=F$ appear in the allowed
cases, as opposed to $(a\rightarrow b)=F$ with rules in Table~\ref{tab:Logical-rules-for-1-1}.
Allowing irrelevant combinations of $a$ and $b$ for $(a\Rightarrow b)=F$
has the advantage that for a false premise $a$, no artificial limitations
are imposed, e.g.

\emph{``If the moon is made of cheese, then my birthday is not on
day x.''}

can be true or not, independent from \emph{x}.

\begin{table}[h]
\begin{centering}
\begin{minipage}[t]{0.4\columnwidth}%
\begin{center}
\textbf{allowed}
\par\end{center}
\begin{center}
\begin{tabular}{cc}
\toprule 
a  & b\tabularnewline
\midrule
\midrule 
\emph{F } & \emph{F}\tabularnewline
\midrule 
\emph{F } & \emph{T}\tabularnewline
\midrule 
\emph{T } & \emph{F}\tabularnewline
\bottomrule
\end{tabular}
\par\end{center}%
\end{minipage}%
\begin{minipage}[t]{0.4\columnwidth}%
\begin{center}
\textbf{forbidden}
\par\end{center}
\begin{center}
\begin{tabular}{cc}
\toprule 
a  & b\tabularnewline
\midrule
\midrule 
\emph{T} & \emph{T}\tabularnewline
\bottomrule
\end{tabular}
\par\end{center}%
\end{minipage}
\par\end{centering}
\caption{Rules for $(a\Rightarrow b)=F$.\label{tab:Logical-rules-for-1}}
\end{table}

\begin{table}[H]
\begin{centering}
\begin{minipage}[t]{0.4\columnwidth}%
\begin{center}
\textbf{allowed}
\par\end{center}
\begin{center}
\begin{tabular}{cc}
\toprule 
a  & b\tabularnewline
\midrule
\midrule 
\emph{T } & \emph{F}\tabularnewline
\bottomrule
\end{tabular}
\par\end{center}%
\end{minipage}%
\begin{minipage}[t]{0.4\columnwidth}%
\begin{center}
\textbf{forbidden}
\par\end{center}
\begin{center}
\begin{tabular}{cc}
\toprule 
a  & b\tabularnewline
\midrule
\midrule 
\emph{F} & \emph{F}\tabularnewline
\midrule 
\emph{F } & \emph{T}\tabularnewline
\midrule 
\emph{T } & \emph{T}\tabularnewline
\bottomrule
\end{tabular}
\par\end{center}%
\end{minipage}
\par\end{centering}
\caption{Rules for $(a\rightarrow b)=F$.\label{tab:Logical-rules-for-1-1}}
\end{table}

The case $(a\Rightarrow b)=F$ is clearly not equivalent to $(a\rightarrow b)=F$,
where only a single case is allowed, and three are forbidden and also
not and also not to the negated contrapositive $(\lnot b\Rightarrow\lnot a)=F$
listed in Table~\ref{tab:Logical-rules-for-1-1-1}. Interestingly,
$(a\Rightarrow b)=F$ is equivalent to $(b\Rightarrow a)=F$ and $(a\Leftrightarrow b)=F$.

Truth of strong equivalence and weak equivalence has the same rules
(Table \ref{tab:Logical-rules-for-1-2}), but their falsehood does
not, since the case $(a,b)=(F,F)$ is not allowed in the strong form.

\begin{table}[h]
\begin{centering}
\begin{minipage}[t]{0.4\columnwidth}%
\begin{center}
\textbf{allowed}
\par\end{center}
\begin{center}
\begin{tabular}{cc}
\toprule 
a  & b\tabularnewline
\midrule
\midrule 
\emph{F } & \emph{T}\tabularnewline
\midrule 
\emph{T} & \emph{F}\tabularnewline
\midrule 
\emph{T} & \emph{T}\tabularnewline
\bottomrule
\end{tabular}
\par\end{center}%
\end{minipage}%
\begin{minipage}[t]{0.4\columnwidth}%
\begin{center}
\textbf{forbidden}
\par\end{center}
\begin{center}
\begin{tabular}{cc}
\toprule 
a  & b\tabularnewline
\midrule
\midrule 
\emph{F} & \emph{F}\tabularnewline
\bottomrule
\end{tabular}
\par\end{center}%
\end{minipage}
\par\end{centering}
\caption{Rules for $(\protect\lnot b\Rightarrow\protect\lnot a)=F$.\label{tab:Logical-rules-for-1-1-1}}
\end{table}

\begin{table}[h]
\begin{centering}
\begin{minipage}[t]{0.4\columnwidth}%
\begin{center}
\textbf{allowed}
\par\end{center}
\begin{center}
\begin{tabular}{cc}
\toprule 
a  & b\tabularnewline
\midrule
\midrule 
\emph{F } & \emph{F}\tabularnewline
\midrule 
\emph{T } & \emph{T}\tabularnewline
\bottomrule
\end{tabular}
\par\end{center}%
\end{minipage}%
\begin{minipage}[t]{0.4\columnwidth}%
\begin{center}
\textbf{forbidden}
\par\end{center}
\begin{center}
\begin{tabular}{cc}
\toprule 
a  & b\tabularnewline
\midrule
\midrule 
\emph{F} & \emph{T}\tabularnewline
\midrule 
\emph{T} & \emph{F}\tabularnewline
\bottomrule
\end{tabular}
\par\end{center}%
\end{minipage}
\par\end{centering}
\caption{Rules for $(a\Leftrightarrow b)=T\equiv(a\leftrightarrow b)=T$.\label{tab:Logical-rules-for-1-2}}
\end{table}

\begin{table}[h]
\begin{centering}
\begin{minipage}[t]{0.4\columnwidth}%
\begin{center}
\textbf{allowed}
\par\end{center}
\begin{center}
\begin{tabular}{cc}
\toprule 
a  & b\tabularnewline
\midrule
\midrule 
\emph{F } & \emph{T}\tabularnewline
\midrule 
\emph{T } & \emph{F}\tabularnewline
\bottomrule
\end{tabular}
\par\end{center}%
\end{minipage}%
\begin{minipage}[t]{0.4\columnwidth}%
\begin{center}
\textbf{forbidden}
\par\end{center}
\begin{center}
\begin{tabular}{cc}
\toprule 
a  & b\tabularnewline
\midrule
\midrule 
\emph{F} & \emph{F}\tabularnewline
\midrule 
\emph{T} & \emph{T}\tabularnewline
\bottomrule
\end{tabular}
\par\end{center}%
\end{minipage}
\par\end{centering}
\caption{Rules for $(a\leftrightarrow b)=F$.\label{tab:Logical-rules-for-1-2-1-1}}
\end{table}


\section*{Additional properties and \protect \\
formulation of proofs}

For both, classical operators such as $\rightarrow$ and non-classical
$\Rightarrow$, allowed and forbidden cases are complete, i.e. all
cases are covered, and complementary to each other, i.e. no case can
ever be allowed and forbidden at once. This feature is essential for
the common definition of proof which we state as

\emph{To prove a statement to be true, only the realisation of allowed
cases may be possible.}

or

\emph{To prove a statement to be true, the possibility of a forbidden
case to be realised must be excluded.}

To formulate proofs, one must show that a forbidden case is never
realised.

So to show $(a\Rightarrow b)=T$ one needs to show the impossibility
of the case $(a,b)=(T,F)$. Ways to do this include the classical
ones:
\begin{enumerate}
\item Direct proof. Take $a=T$ and deduce that $b=T$. 
\item Indirect proof by contrapositive. Take $b=F$ and deduce $a=F$. 
\item Indirect proof by contradiction. Assume $(a\Rightarrow b)=F$ and
deduce that $(a,b)=(T,F)$ is possible, thereby violating a forbidden
case and proofing the opposite $(a\Rightarrow b)=T$. 
\end{enumerate}
To show the negation $(a\Rightarrow b)=F$, one can use its equivalent
$(a\Rightarrow\lnot b)=T$ and apply the rules above.

Truth of weak or strong equivalence $(a\Leftrightarrow b)=T\equiv(a\leftrightarrow b)=T$
is shown as usual by $(a\Rightarrow b)=T$ and $(b\Rightarrow a)=T,$
thus excluding mixed truth values for a and b. To show the negation
of strong equivalence, $(a\leftrightarrow b)=F$, one must show that
$(a,b)=(F,F)$ is impossible in addition to $(a\Leftrightarrow b)=F\equiv((a\Rightarrow b)\land(b\Rightarrow a))=F$.

Classical operators additionally fulfil the requirement of complementarity
between allowed cases for a statement and its negation, e.g. $(a\rightarrow b)=T$
and $(a\rightarrow b)=F$ must not share any allowed cases. This leads
to a single classical truth table, containing only \emph{T }and \emph{F,}
like in Table~\ref{tab:Classical-truth-table}. In non-classical
operators there is no restriction, e.g. $(a\Rightarrow b)=T$ and
$(a\Rightarrow b)=F$ share some allowed cases. These are the irrelevant
cases in the above definition and are marked by \emph{I} in a truth
tables like Table~\ref{tab:Classical-truth-table-1}.

\section*{Conclusion}

After pointing out possible misunderstandings of the material conditional
$a\rightarrow b$ and seemingly paradox results, two ways of resolution
have been identified. The first consists in adding an extra clause
to \emph{``If a, then b'' }when speaking of a material conditional.
The principle of explosion can be avoided by restricting the use of
$a\rightarrow b$ as a law of inference to true premisses $a$. Fo
inference we should always ask: do we show $b$ to be true, do we
show $a\rightarrow b$ to be true, or do we show $b$ to be true already
assuming that $a$ is true.

An alternative is the use of a logical consequence $a\Rightarrow b$
instead of a material conditional, containing the term \emph{irrelevant}
in addition to \emph{true }and \emph{false }for the case of a false
premise\emph{ $a$. }Considering allowed and forbidden cases for statements
$a$ and $b$ individually for true and false logical consequence,
identities and techniques of proofs could be found to be classical
if $a\Rightarrow b=T$, and different from the classical case for
$a\Rightarrow b=F$. This leads to the conclusion that for $a=T$
it is irrelevant, which technique to use to prove $a\Rightarrow b=T$.
For equivalance and negation, special care must be taken, especially
when using the contrapositive. The case where irrelevant inputs $a,b$
enter $a\Rightarrow b$ has not been discussed. This would most likely
lead to a fourth possibility ``unknown'' or \emph{``U''} to cover
undecidable combinations.

\bibliographystyle{plain}
\nocite{*}
\bibliography{logic}


\end{document}
