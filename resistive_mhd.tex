%% LyX 2.3.1 created this file.  For more info, see http://www.lyx.org/.
%% Do not edit unless you really know what you are doing.
\documentclass[english]{article}
\usepackage[T1]{fontenc}
\usepackage[latin9]{inputenc}
\usepackage{amsmath}
\usepackage{babel}
\begin{document}
\global\long\def\tht{\vartheta}%
\global\long\def\ph{\varphi}%
\global\long\def\balpha{\boldsymbol{\alpha}}%
\global\long\def\btheta{\boldsymbol{\theta}}%
\global\long\def\bJ{\boldsymbol{J}}%
\global\long\def\bGamma{\boldsymbol{\Gamma}}%
\global\long\def\bOmega{\boldsymbol{\Omega}}%
\global\long\def\d{\text{d}}%
\global\long\def\t#1{\text{#1}}%
\global\long\def\m{\text{m}}%
\global\long\def\v#1{\boldsymbol{#1}}%

\global\long\def\t#1{\mathbf{#1}}%


\paragraph{Resistive MHD equations}

(see Freidberg p. 9, beginning of Chap. 8 and for linear perturbations,
and problem 8.1 on p. 377 to change Ohm's law)

\begin{align}
\text{Mass:}\, & \frac{\partial\rho}{\partial t}+\nabla\cdot(\rho\v v)=0\\
\text{Momentum:}\, & \rho\frac{\d\v v}{\d t}=\v J\times\v B-\nabla p\\
\text{Energy (adiabatic):}\, & \frac{\d}{\d t}\left(\frac{p}{\rho^{\gamma}}\right)=0\\
\text{Ohm's law:}\, & \v E+\v v\times\v B=\eta\v J\\
\text{Faraday:}\, & \nabla\times\v E=-\frac{\partial\v B}{\partial t}\\
\text{Amp�re:}\, & \nabla\times\v B=\mu_{0}\v J\label{eq:ampere}\\
\text{divB:}\, & \nabla\cdot\v B=0
\end{align}
Here $\d/\d t=\partial/\partial t+\v v\cdot\nabla$. We have 14 unknowns:
$\rho,p,\v v,\v J,\v E,\v B$ for 17 equations (so many?). Parameters:
$\mu_{0}$ (fixed by the universe), adiabatic index $\gamma$, which
we usually set to $5/3$ (monoatomic ideal gas), and resistivity $\eta$
in Ohms which can be computed from the Spitzer resistivity (bottom
of p. 110 in Freidberg). Displacement currents neglected in Eq.~(\ref{eq:ampere}).
Gauss' law $\varepsilon_{0}\nabla\cdot\v E=n_{i}-n_{e}\approx0$ replaced
by quasineutrality condition $n_{i}=n_{e}$. Eq. (\ref{eq:ampere})

\paragraph{Ideal MHD equilibrium}

The usual starting point for further computations is the ideal MHD
equilibrium given by 

\begin{align}
\nabla p_{0} & =\v J_{0}\times\v B_{0}\\
\nabla\times\v B_{0} & =\mu_{0}\v J_{0}\\
\nabla\cdot\v B_{0} & =0
\end{align}
7 unknowns: $p,\v J,\v B$ for 7 equations.

\paragraph{Linear ideal MHD}

Applying a perturbation with functions $g=g_{0}+g_{1}$ and neglecting
flow $\boldsymbol{v}_{0}=\boldsymbol{v}_{1}=0$ yields

\begin{align}
\nabla p_{1} & =\v J_{0}\times\v B_{1}+\v J_{1}\times\v B_{0}\\
\nabla\times\v B_{1} & =\mu_{0}\v J_{1}\\
\nabla\cdot\v B_{1} & =0
\end{align}

Here we subtracted zero-index versions of each equations to make them
disappear, e.g. $\nabla\times B_{0}=\mu_{0}\v J_{0}$ from Amp�re's
law. Usually, we split the force balance from Maxwell's equation and
solve the two together iteratively (see Master's thesis of Patrick
Lainer).

\paragraph{Linear resistive MHD}

We apply a perturbation $\propto e^{-i\omega t}$ to our MHD equilibrium
and drop terms of second order. In the full resistive system this
means that
\begin{align}
\text{Mass:}\, & -i\omega\rho_{1}+\nabla\cdot(\rho_{0}\v v_{1}+\rho_{1}\boldsymbol{v}_{0})=0\\
\text{Momentum:}\, & -i\omega(\rho_{0}\v v_{1}+\rho_{1}\boldsymbol{v}_{0})=\v J_{0}\times\v B_{1}+\v J_{1}\times\v B_{0}-\nabla p_{1}\\
\text{Energy (adiabatic):}\, & p_{1}\rho_{0}-\gamma p_{0}\rho_{1}=0\\
\text{Ohm's law:}\, & \v E_{1}+\v v_{1}\times\v B_{0}+\boldsymbol{v}_{0}\times\boldsymbol{B}_{1}=\eta\v J_{1}\\
\text{Faraday:}\, & \nabla\times\v E_{1}=-i\omega\v B_{1}\\
\text{Amp�re:}\, & \nabla\times\v B_{1}=\mu_{0}\v J_{1}\label{eq:ampere-1}\\
\text{divB:}\, & \nabla\cdot\v B_{1}=0
\end{align}
In the energy equation we have used a Taylor expansion in the perturbation
of $\rho$, and multiplied the equation by $\rho_{0}^{\gamma+1}/(i\omega)$.
Again we already subtracted zero-index versions of each equations. 

Expressing $\rho_{1}$ via the Energy law by
\[
\rho_{1}=\frac{p_{1}}{\gamma p_{0}}\rho_{0}
\]
and insertion into the Mass law yields
\[
-i\omega\frac{\rho_{0}}{\gamma p_{0}}p_{1}+\nabla\cdot(\rho_{0}\v v_{1}+\frac{p_{1}}{\gamma p_{0}}\rho_{0}\boldsymbol{v}_{0})=0,
\]
and in the Momentum law
\[
-i\omega(\rho_{0}\v v_{1}+\frac{\rho_{0}}{\gamma p_{0}}p_{1}\boldsymbol{v}_{0})=\v J_{0}\times\v B_{1}+\v J_{1}\times\v B_{0}-\nabla p_{1}.
\]
In the static case $\omega=0$, the equations reduce to linear ideal
MHD, as velocities don't appear in the Momentum equation anymore. 

\paragraph{Splitting the problem without flow}

Part 1: Solve
\begin{align}
-i\omega\rho_{0}\v v_{1} & =\v J_{0}\times\v B_{1}+\v J_{1}\times\v B_{0}-\nabla p_{1}\\
-i\omega\rho_{0}p_{1} & =-\gamma p_{0}\nabla\cdot(\rho_{0}\v v_{1})\\
-i\omega\v B_{1} & =\nabla\times(\eta\v J_{1}-\v v_{1}\times\v B_{0})
\end{align}
for given $\v B_{1}$. 7 unknowns $p_{1},\v J_{1},\v v_{1}$ for 7
equations.

Part 2: Solve
\begin{align}
\nabla\times\v B_{1} & =\mu_{0}\v J_{1}\label{eq:ampere-1-1-1}\\
\nabla\cdot\v B_{1} & =0
\end{align}
for given $\v J_{1}$. 3 unknowns $\v B_{1}$ for 4 equations. 

\paragraph{Splitting the problem further}
\begin{enumerate}
\item Parallel momentum (pressure):
\begin{equation}
-i\omega(\rho_{0}v_{1\parallel})B_{0}=-\v B_{1}\cdot\nabla p_{0}-\v B_{0}\cdot\nabla p_{1}
\end{equation}
\item Perpendicular momentum (current):
\begin{equation}
-i\omega\rho_{0}\v h_{0}\times\v v_{1\perp}=\v B_{0}\times(\v J_{0}\times\v B_{1})+B_{0}\v J_{1}-\v B_{0}\times\nabla p_{1}
\end{equation}
\item Ohm+Faraday:
\begin{equation}
-i\omega\v B_{1}=\nabla\times(\eta\v J_{1}-\v v_{1\perp}\times\v B_{0})
\end{equation}
\item Mass+Adiabatic:
\begin{equation}
-i\omega\rho_{0}p_{1}=-\gamma p_{0}\nabla\cdot(\rho_{0}\v v_{1})
\end{equation}
Divergence
\begin{align}
\nabla\cdot\v u & =\nabla\cdot\left(\frac{u_{\parallel}}{B_{0}}\v B_{0}+\v u_{\perp}\right)=\v B_{0}\cdot\nabla\left(\frac{u_{\parallel}}{B_{0}}\right)+\nabla\cdot\v u_{\perp}\\
\nabla\cdot(\rho_{0}\v v_{1}) & =\v B_{0}\cdot\nabla\left(\frac{\rho_{0}v_{1\parallel}}{B_{0}}\right)+\nabla\cdot(\rho_{0}\v v_{1\perp})
\end{align}
With the possibility to pull out $\rho_{0}$ if it is constant along
field lines $(\v B_{0}\cdot\nabla\rho_{0})$. 
\item Magnetostatics:
\begin{align}
\nabla\times\v B_{1} & =\mu_{0}\v J_{1}\label{eq:ampere-1-1-1-1}\\
\nabla\cdot\v B_{1} & =0
\end{align}
\end{enumerate}
Steps of solution
\begin{enumerate}
\item solve parallel Momentum with given $\v B_{1},v_{1\parallel}$ $\Rightarrow p_{1}$
\item solve perpendicular Momentum with given $\v B_{1},\v v_{\perp},p_{1}\Rightarrow\v J_{1}$
\item solve Ohm+Faraday with Mass+Adiabatic with given $\v B_{1},p_{1},\v J_{1}\Rightarrow v_{1\parallel},\v v_{1\perp}$
\item solve Amp�re with given $\v J_{1}$$\Rightarrow\v B_{1}$
\end{enumerate}
Rosenbluth (slab), Furth (cylinder) tearing modes. Those are ``classical''
tearing modes as opposed to neoclassical ones. Generally, such modes
describe formation of magnetic islands, where flux surfaces are ``torn
apart''.

Also: link to Alfven waves. There singularity is resolved by finite
electron mass but problem is shifted to short-scale mode.

\paragraph{Solution in flux coordinates coordinates at zero frequency and/or
zero flow}
\begin{enumerate}
\item Parallel momentum (pressure)
\begin{equation}
\v B_{0}(\boldsymbol{x})\cdot\nabla p_{1}(\boldsymbol{x})=-\v B_{1}(\boldsymbol{x})\cdot\nabla p_{0}(r)
\end{equation}
becomes
\begin{equation}
B_{0}^{\tht}(\boldsymbol{x})\frac{\partial p_{1}(\boldsymbol{x})}{\partial\tht}+B_{0}^{\ph}\frac{\partial p_{1}(\boldsymbol{x})}{\partial\ph}=-B_{1}^{r}(\boldsymbol{x})p_{0}^{\prime}(r).
\end{equation}
Using vector density components $\sqrt{g}B^{k}\equiv\mathcal{B}^{k}$
with $\mathcal{B}_{0}^{\tht}=\mathcal{B}_{0}^{\tht}(r)$ and $\mathcal{B}_{0}^{\ph}=\mathcal{B}_{0}^{\ph}(r)=q\mathcal{B}_{0}^{\tht}(r)$
in flux coordinates for $\v B_{0}$ we have
\begin{equation}
\mathcal{B}_{0}^{\tht}(r)\left(\frac{\partial p_{1}(\boldsymbol{x})}{\partial\tht}+q(r)\frac{\partial p_{1}(\boldsymbol{x})}{\partial\ph}\right)=-\mathcal{B}_{1}^{r}(\boldsymbol{x})p_{0}^{\prime}(r).
\end{equation}
Written in harmonic expansion in angles $\tht,\ph$ we have
\begin{equation}
i\mathcal{B}_{0}^{\tht}(r)\left(mp_{mn}(r)+nq(r)p_{mn}(r)\right)=-\mathcal{B}_{mn}^{r}(\boldsymbol{x})p_{0}^{\prime}(r)
\end{equation}
with the solution at each flux surface given by
\begin{equation}
p_{mn}(r)=i\frac{\mathcal{B}_{mn}^{r}(r)}{\mathcal{B}_{0}^{\tht}(r)}\frac{p_{0}^{\prime}(r)}{m+nq(r)}.
\end{equation}
Here we clearly see that at flux surfaces where $q(r)=-m/n$ the solution
diverges. 
\item Perpendicular momentum (current)
\begin{equation}
0=\v B_{0}\times(\v J_{0}\times\v B_{1})+B_{0}\v J_{1}-\v B_{0}\times\nabla p_{1}
\end{equation}
becomes
\begin{equation}
0=\v B_{0}\times(\v J_{0}\times\v B_{1})+B_{0}\v J_{1}-i\v B_{0}\times\v kp_{1}
\end{equation}
\end{enumerate}

\paragraph{Momentum equation with flow in flux coordinates}

Adding a finite flow velocity $\v v$ results in the following equation,
\begin{equation}
-i\omega\rho_{0}(r)B_{0}(\boldsymbol{x})\left(v_{1\parallel}(\v x)+\frac{v_{0\parallel}(\boldsymbol{x})}{\gamma p_{0}(r)}p_{1}(\v x)\right)+\v B_{0}(\boldsymbol{x})\cdot\nabla p_{1}(\boldsymbol{x})=-\v B_{1}(\boldsymbol{x})\cdot\nabla p_{0}(r).
\end{equation}
Here $v_{0\parallel}=\v B_{0}\cdot\v v_{0}$ and $v_{1\parallel}=\v B_{0}\cdot\v v_{1}$.
Written in flux coordinates we obtain
\begin{equation}
-i\omega\rho_{0}(r)\mathcal{B}_{0}^{\tht}(r)\left(v_{1\tht}(\v x)+q(r)v_{1\ph}(\v x)+\frac{v_{0\tht}(\v x)+q(r)v_{0\ph}(\v x)}{\gamma p_{0}(r)}p_{1}(\v x)\right)+\mathcal{B}_{0}^{\tht}(r)\left(\frac{\partial p_{1}(\boldsymbol{x})}{\partial\tht}+q(r)\frac{\partial p_{1}(\boldsymbol{x})}{\partial\ph}\right)=-\mathcal{B}_{1}^{r}(\boldsymbol{x})p_{0}^{\prime}(r).
\end{equation}
In the general case of non-zero equilibrium flow $v_{0\parallel}(\boldsymbol{x})$
we cannot expand this in angles locally on a flux surface, as a non-linear
term appears in its product with $p_{1}(\v x)$. Assuming $v_{0\tht}=v_{0\tht}(r)$
and $v_{0\ph}=v_{0\ph}(r)$ we can still use this method and obtain
a radially local Fourier expansion (skipping dependencies on $r$
in notation) with
\begin{equation}
-i\omega\rho_{0}\left(v_{mn\,\tht}+qv_{mn\,\ph}+\frac{v_{0\tht}+qv_{0\ph}}{\gamma p_{0}}p_{mn}\right)+i\left(m+nq\right)p_{mn}=-\frac{p_{0}^{\prime}}{\mathcal{B}_{0}^{\tht}}B_{mn}^{r}.
\end{equation}
The solution in $p_{mn}$ is given by
\begin{equation}
p_{mn}=\frac{i\frac{p_{0}^{\prime}}{\mathcal{B}_{0}^{\tht}}B_{mn}^{r}+\omega\rho_{0}\left(v_{mn\,\tht}+qv_{mn\,\ph}\right)}{m+nq-\omega\frac{\rho_{0}}{\gamma p_{0}}(v_{0\tht}+qv_{0\ph})}.
\end{equation}
Here the introduction of $v_{0\parallel}$ just shifts the resonance
radially and cannot resolve it. Only if the numerator tends towards
zero faster than the numerator tends to infinity, a finite solution
can follow in this model.
\end{document}
