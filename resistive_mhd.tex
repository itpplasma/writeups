%% LyX 2.3.0 created this file.  For more info, see http://www.lyx.org/.
%% Do not edit unless you really know what you are doing.
\documentclass[english]{article}
\usepackage[T1]{fontenc}
\usepackage[latin9]{inputenc}
\usepackage{amsmath}
\usepackage{babel}
\begin{document}
\global\long\def\tht{\vartheta}
\global\long\def\ph{\varphi}
\global\long\def\balpha{\boldsymbol{\alpha}}
\global\long\def\btheta{\boldsymbol{\theta}}
\global\long\def\bJ{\boldsymbol{J}}
\global\long\def\bGamma{\boldsymbol{\Gamma}}
\global\long\def\bOmega{\boldsymbol{\Omega}}
\global\long\def\d{\text{d}}
\global\long\def\t#1{\text{#1}}
\global\long\def\m{\text{m}}
\global\long\def\v#1{\boldsymbol{#1}}

\global\long\def\t#1{\mathbf{#1}}


\paragraph{Resistive MHD equations}

(see Freidberg p. 9, beginning of Chap. 8 and for linear perturbations,
and problem 8.1 on p. 377 to change Ohm's law)

\begin{align}
\text{Mass:}\, & \frac{\partial\rho}{\partial t}+\nabla\cdot(\rho\v v)=0\\
\text{Momentum:}\, & \rho\frac{\d\v v}{\d t}=\v J\times\v B-\nabla p\\
\text{Energy (adiabatic):}\, & \frac{\d}{\d t}\left(\frac{p}{\rho^{\gamma}}\right)=0\\
\text{Ohm's law:}\, & \v E+\v v\times\v B=\eta\v J\\
\text{Faraday:}\, & \nabla\times\v E=-\frac{\partial\v B}{\partial t}\\
\text{Amp�re:}\, & \nabla\times\v B=\mu_{0}\v J\label{eq:ampere}\\
\text{divB:}\, & \nabla\cdot\v B=0
\end{align}
Here $\d/\d t=\partial/\partial t+\v v\cdot\nabla$. We have 14 unknowns:
$\rho,p,\v v,\v J,\v E,\v B$ for 17 equations (so many?). Parameters:
$\mu_{0}$ (fixed by the universe), adiabatic index $\gamma$, which
we usually set to $5/3$ (monoatomic ideal gas), and resistivity $\eta$
in Ohms which can be computed from the Spitzer resistivity (bottom
of p. 110 in Freidberg). Displacement currents neglected in Eq.~(\ref{eq:ampere}).
Gauss' law $\varepsilon_{0}\nabla\cdot\v E=n_{i}-n_{e}\approx0$ replaced
by quasineutrality condition $n_{i}=n_{e}$. Eq. (\ref{eq:ampere})

\paragraph{Ideal MHD equilibrium}

The usual starting point for further computations is the ideal MHD
equilibrium given by 

\begin{align}
\nabla p_{0} & =\v J_{0}\times\v B_{0}\\
\nabla\times\v B_{0} & =\mu_{0}\v J_{0}\\
\nabla\cdot\v B_{0} & =0
\end{align}
7 unknowns: $p,\v J,\v B$ for 7 equations.

\paragraph{Linear ideal MHD}

Applying a perturbation with functions $g=g_{0}+g_{1}$ and neglecting
flow $\boldsymbol{v}_{0}=\boldsymbol{v}_{1}=0$ yields

\begin{align}
\nabla p_{1} & =\v J_{0}\times\v B_{1}+\v J_{1}\times\v B_{0}\\
\nabla\times\v B_{1} & =\mu_{0}\v J_{1}\\
\nabla\cdot\v B_{1} & =0
\end{align}

Here we subtracted zero-index versions of each equations to make them
disappear, e.g. $\nabla\times B_{0}=\mu_{0}\v J_{0}$ from Amp�re's
law. Usually, we split the force balance from Maxwell's equation and
solve the two together iteratively (see Master's thesis of Patrick
Lainer).

\paragraph{Linear resistive MHD}

We apply a perturbation $\propto e^{-i\omega t}$ to our MHD equilibrium
and drop terms of second order. In the full resistive system this
means that
\begin{align}
\text{Mass:}\, & -i\omega\rho_{1}+\nabla\cdot(\rho_{0}\v v_{1}+\rho_{1}\boldsymbol{v}_{0})=0\\
\text{Momentum:}\, & -i\omega(\rho_{0}\v v_{1}+\rho_{1}\boldsymbol{v}_{0})=\v J_{0}\times\v B_{1}+\v J_{1}\times\v B_{0}-\nabla p_{1}\\
\text{Energy (adiabatic):}\, & p_{1}\rho_{0}-\gamma p_{0}\rho_{1}=0\\
\text{Ohm's law:}\, & \v E_{1}+\v v_{1}\times\v B_{0}+\boldsymbol{v}_{0}\times\boldsymbol{B}_{1}=\eta\v J_{1}\\
\text{Faraday:}\, & \nabla\times\v E_{1}=-i\omega\v B_{1}\\
\text{Amp�re:}\, & \nabla\times\v B_{1}=\mu_{0}\v J_{1}\label{eq:ampere-1}\\
\text{divB:}\, & \nabla\cdot\v B_{1}=0
\end{align}
In the energy equation we have used a Taylor expansion in the perturbation
of $\rho$, and multiplied the equation by $\rho_{0}^{\gamma+1}/(i\omega)$.
Again we already subtracted zero-index versions of each equations. 

Expressing $\rho_{1}$ via the Energy law by
\[
\rho_{1}=\frac{p_{1}}{\gamma p_{0}}\rho_{0}
\]
and insertion into the Mass law yields
\[
-i\omega\frac{\rho_{0}}{\gamma p_{0}}p_{1}+\nabla\cdot(\rho_{0}\v v_{1}+\frac{p_{1}}{\gamma p_{0}}\rho_{0}\boldsymbol{v}_{0})=0,
\]
and in the Momentum law
\[
-i\omega(\rho_{0}\v v_{1}+\frac{\rho_{0}}{\gamma p_{0}}p_{1}\boldsymbol{v}_{0})=\v J_{0}\times\v B_{1}+\v J_{1}\times\v B_{0}-\nabla p_{1}.
\]
In the static case $\omega=0$, the equations reduce to linear ideal
MHD, as velocities don't appear in the Momentum equation anymore. 

\paragraph{Splitting the problem without flow}

Part 1: Solve
\begin{align}
-i\omega\rho_{0}\v v_{1} & =\v J_{0}\times\v B_{1}+\v J_{1}\times\v B_{0}-\nabla p_{1}\\
-i\omega\rho_{0}p_{1} & =-\gamma p_{0}\nabla\cdot(\rho_{0}\v v_{1})\\
-i\omega\v B_{1} & =\nabla\times(\eta\v J_{1}-\v v_{1}\times\v B_{0})
\end{align}
for given $\v B_{1}$. 7 unknowns $p_{1},\v J_{1},\v v_{1}$ for 7
equations.

Part 2: Solve
\begin{align}
\nabla\times\v B_{1} & =\mu_{0}\v J_{1}\label{eq:ampere-1-1-1}\\
\nabla\cdot\v B_{1} & =0
\end{align}
for given $\v J_{1}$. 3 unknowns $\v B_{1}$ for 4 equations. 

\paragraph{Splitting the problem further}
\begin{enumerate}
\item Parallel momentum (pressure):
\begin{equation}
-i\omega(\rho_{0}v_{1\parallel})B_{0}=-\v B_{1}\cdot\nabla p_{0}-\v B_{0}\cdot\nabla p_{1}
\end{equation}
\item Perpendicular momentum (current):
\begin{equation}
-i\omega\rho_{0}\v h_{0}\times\v v_{1\perp}=\v B_{0}\times(\v J_{0}\times\v B_{1})+B_{0}\v J_{1}-\v B_{0}\times\nabla p_{1}
\end{equation}
\item Ohm+Faraday:
\begin{equation}
-i\omega\v B_{1}=\nabla\times(\eta\v J_{1}-\v v_{1\perp}\times\v B_{0})
\end{equation}
\item Mass+Adiabatic:
\begin{equation}
-i\omega\rho_{0}p_{1}=-\gamma p_{0}\nabla\cdot(\rho_{0}\v v_{1})
\end{equation}
Divergence
\begin{align}
\nabla\cdot\v u & =\nabla\cdot\left(\frac{u_{\parallel}}{B_{0}}\v B_{0}+\v u_{\perp}\right)=\v B_{0}\cdot\nabla\left(\frac{u_{\parallel}}{B_{0}}\right)+\nabla\cdot\v u_{\perp}\\
\nabla\cdot(\rho_{0}\v v_{1}) & =\v B_{0}\cdot\nabla\left(\frac{\rho_{0}v_{1\parallel}}{B_{0}}\right)+\nabla\cdot(\rho_{0}\v v_{1\perp})
\end{align}
With the possibility to pull out $\rho_{0}$ if it is constant along
field lines $(\v B_{0}\cdot\nabla\rho_{0})$. 
\item Magnetostatics:
\begin{align}
\nabla\times\v B_{1} & =\mu_{0}\v J_{1}\label{eq:ampere-1-1-1-1}\\
\nabla\cdot\v B_{1} & =0
\end{align}
\end{enumerate}
Steps of solution
\begin{enumerate}
\item solve parallel Momentum with given $\v B_{1},v_{1\parallel}$ $\Rightarrow p_{1}$
\item solve perpendicular Momentum with given $\v B_{1},\v v_{\perp},p_{1}\Rightarrow\v J_{1}$
\item solve Ohm+Faraday with Mass+Adiabatic with given $\v B_{1},p_{1},\v J_{1}\Rightarrow v_{1\parallel},\v v_{1\perp}$
\item solve Amp�re with given $\v J_{1}$$\Rightarrow\v B_{1}$
\end{enumerate}
Rosenbluth (slab), Furth (cylinder) tearing modes. Those are ``classical''
tearing modes as opposed to neoclassical ones. Generally, such modes
describe formation of magnetic islands, where flux surfaces are ``torn
apart''.

Also: link to Alfven waves. There singularity is resolved by finite
electron mass but problem is shifted to short-scale mode.

\paragraph{Solution in flux coordinates coordinates at zero frequency and/or
zero flow}
\begin{enumerate}
\item Parallel momentum (pressure)
\begin{equation}
\v B_{0}(\boldsymbol{x})\cdot\nabla p_{1}(\boldsymbol{x})=-\v B_{1}(\boldsymbol{x})\cdot\nabla p_{0}(r)
\end{equation}
becomes
\begin{equation}
B_{0}^{\tht}(\boldsymbol{x})\frac{\partial p_{1}(\boldsymbol{x})}{\partial\tht}+B_{0}^{\ph}\frac{\partial p_{1}(\boldsymbol{x})}{\partial\ph}=-B_{1}^{r}(\boldsymbol{x})p_{0}^{\prime}(r).
\end{equation}
Using vector density components $\sqrt{g}B^{k}\equiv\mathcal{B}^{k}$
with $\mathcal{B}_{0}^{\tht}=\mathcal{B}_{0}^{\tht}(r)$ and $\mathcal{B}_{0}^{\ph}=\mathcal{B}_{0}^{\ph}(r)=q\mathcal{B}_{0}^{\tht}(r)$
in flux coordinates for $\v B_{0}$ we have
\begin{equation}
\mathcal{B}_{0}^{\tht}(r)\left(\frac{\partial p_{1}(\boldsymbol{x})}{\partial\tht}+q(r)\frac{\partial p_{1}(\boldsymbol{x})}{\partial\ph}\right)=-\mathcal{B}_{1}^{r}(\boldsymbol{x})p_{0}^{\prime}(r).
\end{equation}
Written in harmonic expansion in angles $\tht,\ph$ we have
\begin{equation}
i\mathcal{B}_{0}^{\tht}(r)\left(mp_{mn}(r)+nq(r)p_{mn}(r)\right)=-\mathcal{B}_{mn}^{r}(\boldsymbol{x})p_{0}^{\prime}(r)
\end{equation}
with the solution at each flux surface given by
\begin{equation}
p_{mn}(r)=i\frac{\mathcal{B}_{mn}^{r}(r)}{\mathcal{B}_{0}^{\tht}(r)}\frac{p_{0}^{\prime}(r)}{m+nq(r)}.\label{eq:pmn}
\end{equation}
Here we clearly see that at flux surfaces where $q(r)=-m/n$ the solution
diverges. 
\item For current:
\begin{equation}
\v J_{0}\times\v B_{1}+\v J_{1}\times\v B_{0}-\nabla p_{1}
\end{equation}
We have
\begin{equation}
\nabla\cdot\v J_{0}=\partial_{i}\mathcal{J}_{0}^{\,i}=0
\end{equation}
so 
\begin{equation}
\partial_{r}\mathcal{J}_{0}^{\,r}+\partial_{\tht}\mathcal{J}_{0}^{\tht}+\partial_{\ph}\mathcal{J}_{0}^{\ph}=0.
\end{equation}
and since we have axisymmetry, $\partial_{\ph}$ vanishes, and in
flux coordinates $\mathcal{J}_{0}^{\,r}=0$. What remains is $\mathcal{J}_{0}^{\tht}$,
so 
\begin{equation}
\partial_{\tht}\mathcal{J}_{0}^{\tht}=0\Rightarrow\mathcal{J}_{0}^{\,\tht}=\sqrt{g}J_{0}^{\,\tht}=\mathcal{J}_{0}^{\,\tht}(r).
\end{equation}
In addition we have
\begin{equation}
\mathcal{B}_{0}^{\,\tht}=\mathcal{B}_{0}^{\,\tht}(r),\quad\mathcal{B}_{0}^{\,\ph}=\mathcal{B}_{0}^{\,\ph}(r)
\end{equation}
due to flux coordinates. \textbf{For further derivations see section
in cylindrical coordinates. Most likely this can be extended to special
cases in flux coordinates with $\mathcal{J}_{0}^{\,\ph}=\mathcal{J}_{0}^{\,\ph}(r)$.
This is the case in Hamada coordinates, but not necessarily in symmetry
flux coordinates. One could start looking for conditions under which
symmetry flux coordinates coincide with Hamada coordinates for testing.}
\end{enumerate}

\paragraph{Momentum equation with flow in flux coordinates}

Adding a finite flow velocity $\v v$ results in the following equation,
\begin{equation}
-i\omega\rho_{0}(r)B_{0}(\boldsymbol{x})\left(v_{1\parallel}(\v x)+\frac{v_{0\parallel}(\boldsymbol{x})}{\gamma p_{0}(r)}p_{1}(\v x)\right)+\v B_{0}(\boldsymbol{x})\cdot\nabla p_{1}(\boldsymbol{x})=-\v B_{1}(\boldsymbol{x})\cdot\nabla p_{0}(r).
\end{equation}
Here $v_{0\parallel}=\v B_{0}\cdot\v v_{0}$ and $v_{1\parallel}=\v B_{0}\cdot\v v_{1}$.
Written in flux coordinates we obtain
\begin{equation}
-i\omega\rho_{0}(r)\mathcal{B}_{0}^{\tht}(r)\left(v_{1\tht}(\v x)+q(r)v_{1\ph}(\v x)+\frac{v_{0\tht}(\v x)+q(r)v_{0\ph}(\v x)}{\gamma p_{0}(r)}p_{1}(\v x)\right)+\mathcal{B}_{0}^{\tht}(r)\left(\frac{\partial p_{1}(\boldsymbol{x})}{\partial\tht}+q(r)\frac{\partial p_{1}(\boldsymbol{x})}{\partial\ph}\right)=-\mathcal{B}_{1}^{r}(\boldsymbol{x})p_{0}^{\prime}(r).
\end{equation}
In the general case of non-zero equilibrium flow $v_{0\parallel}(\boldsymbol{x})$
we cannot expand this in angles locally on a flux surface, as a non-linear
term appears in its product with $p_{1}(\v x)$. Assuming $v_{0\tht}=v_{0\tht}(r)$
and $v_{0\ph}=v_{0\ph}(r)$ we can still use this method and obtain
a radially local Fourier expansion (skipping dependencies on $r$
in notation) with
\begin{equation}
-i\omega\rho_{0}\left(v_{mn\,\tht}+qv_{mn\,\ph}+\frac{v_{0\tht}+qv_{0\ph}}{\gamma p_{0}}p_{mn}\right)+i\left(m+nq\right)p_{mn}=-\frac{p_{0}^{\prime}}{\mathcal{B}_{0}^{\tht}}B_{mn}^{r}.
\end{equation}
The solution in $p_{mn}$ is given by
\begin{equation}
p_{mn}=\frac{i\frac{p_{0}^{\prime}}{\mathcal{B}_{0}^{\tht}}B_{mn}^{r}+\omega\rho_{0}\left(v_{mn\,\tht}+qv_{mn\,\ph}\right)}{m+nq-\omega\frac{\rho_{0}}{\gamma p_{0}}(v_{0\tht}+qv_{0\ph})}.
\end{equation}
Here the introduction of $v_{0\parallel}$ just shifts the resonance
radially and cannot resolve it. Only if the numerator tends towards
zero faster than the numerator tends to infinity, a finite solution
can follow in this model.

\paragraph{Solution in cylinder coordinates coordinates at zero frequency and/or
zero flow}

First we are going to approximate our torus by a periodic cylinder.
Before, we had used a different kind of cylindrical coordinates $(R,\ph,Z)$
with the cylinder axis coinciding with cartesian $Z$ axis in the
center of the torus. Now we take the $z$-axis as the distance along
the toroidal direction and define transformations
\begin{align}
R(r,\tht,z) & =R_{0}+r\cos\tht,\\
\ph(r,\tht,z) & =z/R_{0},\\
Z(r,\tht,z) & =r\sin\tht.
\end{align}
Non-vanishing derivatives are
\begin{align}
\partial_{r}R(r,\tht,z) & =\cos\tht,\\
\partial_{\tht}R(r,\tht,z) & =-r\sin\tht,\\
\\
\partial_{z}\ph(r,\tht,z) & =1/R_{0},\\
\\
\partial_{r}Z(r,\tht,z) & =\sin\tht,\\
\partial_{\tht}Z(r,\tht,z) & =r\cos\tht.
\end{align}
The metric tensor in original cylindrical coordinates was
\begin{equation}
g_{ij}=\left(\begin{array}{ccc}
1\\
 & R^{2}\\
 &  & 1
\end{array}\right).
\end{equation}
The metric tensor in the new \textquotedbl toroidal cylinder coordinates\textquotedbl{}
are
\begin{equation}
\bar{g}_{ij}=\frac{\partial x^{a}}{\partial\bar{x}^{i}}\frac{\partial x^{b}}{\partial\bar{x}^{j}}g_{ab}.
\end{equation}
Thus with $\cos^{2}\tht+\sin^{2}\tht=1$ we obtain
\begin{equation}
\bar{g}_{ij}=\left(\begin{array}{ccc}
1\\
 & r^{2}\\
 &  & \frac{R^{2}}{R_{0}^{\,2}}
\end{array}\right)=\left(\begin{array}{ccc}
1\\
 & r^{2}\\
 &  & (1+\frac{r}{R_{0}}\cos\tht)^{2}
\end{array}\right).
\end{equation}
At large aspect ratios $r\ll R_{0}$ it follows that $\bar{g}_{zz}\approx1$
and our new coordinates behave as usual cylinder coordinates, but
with the topology of a torus, as $z$ closes on itself toroidally.
Taking this approximation we now work in physical, rather than covariant
components. Our unit basis vectors are
\begin{equation}
\hat{\boldsymbol{e}}_{r}=\nabla r,\quad\hat{\v e}_{\tht}=r\nabla\tht,\quad\hat{\v e}_{z}=\nabla z.
\end{equation}
We model our magnetic field $\boldsymbol{B}_{0}$ and currents $\boldsymbol{J}_{0}$
winding around surfaces of $r=\text{const.}$ and physical components
depending only on $r$,
\begin{align}
\v B_{0} & =B_{0\tht}(r)\hat{\v e}_{\tht}+B_{0z}(r)\hat{\v e}_{z},\\
\v J_{0} & =J_{0\tht}(r)\hat{\v e}_{\tht}+B_{0z}(r)\hat{\v e}_{z}.
\end{align}
We can define the safety factor (TODO: check if this is correct) as
\begin{equation}
q(r)=\frac{B_{0z}(r)}{B_{0\tht}(r)}.
\end{equation}
Now we start with 
\begin{align}
\nabla p & =\frac{1}{c}\v J\times\boldsymbol{B},\\
\nabla\cdot\boldsymbol{B} & =0,\\
\nabla\cdot\boldsymbol{J} & =0.
\end{align}
The linear order equation is 
\begin{align}
\nabla p_{1} & =\frac{1}{c}(\v J_{0}\times\v B_{1}+\v J_{1}\times\v B_{0}),\\
\nabla\cdot\boldsymbol{B}_{1} & =0,\\
\nabla\cdot\boldsymbol{J}_{1} & =0.
\end{align}
Modeling quantities $p_{1},\boldsymbol{B}_{1}$ and $\boldsymbol{J}_{1}$
in wavenumber space on each flux surface as
\begin{equation}
f(r,\tht,z)=f_{\boldsymbol{k}}(r)e^{i\boldsymbol{k}\cdot\boldsymbol{x}}=f(r)e^{i(k_{\tht}\tht+k_{z}z)}
\end{equation}
with $\v k=k_{\tht}\hat{\v e}_{\tht}+k_{z}\hat{\v e}_{z}$ there is
no product of two harmonics, as unperturbed quantities depend only
on $r$. So we can write independent equations for each $\boldsymbol{k}$
with
\begin{align}
p_{\boldsymbol{k}}^{\prime}(r)\hat{\boldsymbol{e}}_{r}+i\boldsymbol{k}p_{\boldsymbol{k}}(r) & =\frac{1}{c}(\v J_{0}(r)\times\v B_{\boldsymbol{k}}(r)+\v J_{\boldsymbol{k}}(r)\times\v B_{0}(r)),\\
B_{\boldsymbol{k}r}^{\prime}(r)+i\boldsymbol{k}\cdot\boldsymbol{B}_{\boldsymbol{k}}(r) & =0,\\
J_{\boldsymbol{k}r}^{\prime}(r)+i\boldsymbol{k}\cdot\boldsymbol{J}_{\boldsymbol{k}}(r) & =0.
\end{align}
Here, vector quantities depending on $r$ mean that their components
in $(r,\tht,z)$ depend only on $r$, and primes mean radial derivatives.
Repeating the exercise from above we first take the parallel part
by a scalar product with $\boldsymbol{B}_{0}$
\begin{align}
i\boldsymbol{k}\cdot\boldsymbol{B}_{0}p_{\boldsymbol{k}} & =\frac{1}{c}(\boldsymbol{B}_{0}\cdot(\v J_{0}\times\v B_{\boldsymbol{k}}))\nonumber \\
 & =-\boldsymbol{B}_{\boldsymbol{k}}\cdot\nabla p_{0}=-\frac{1}{c}B_{\boldsymbol{k}r}p_{0}^{\prime}.
\end{align}
This means that Eq.~(\ref{eq:pmn}) is represented by
\begin{equation}
p_{\boldsymbol{k}}=i\frac{B_{\boldsymbol{k}r}p_{0}^{\prime}}{\boldsymbol{k}\cdot\boldsymbol{B}_{0}}.
\end{equation}
Resonance is obtained for $\boldsymbol{k}\parallel\boldsymbol{B}_{0}$.
\textbf{This is already one term inside Heyn08, Eq. (6), inside $F_{p}$.}

Now we look at Ampere's law for the perturbation with
\begin{equation}
\nabla\times\boldsymbol{B}_{1}=\frac{4\pi}{c}\boldsymbol{J}_{1}.
\end{equation}
Note that we don't assume it to be true for unperturbed quantities.
Then in $\boldsymbol{k}$-space
\begin{equation}
i\boldsymbol{k}\times\boldsymbol{B}_{\boldsymbol{k}}-B_{\boldsymbol{k}z}^{\prime}\hat{\boldsymbol{e}}_{\tht}+B_{\boldsymbol{k}\tht}^{\prime}\hat{\boldsymbol{e}}_{r}=\frac{4\pi}{c}\boldsymbol{J}_{\boldsymbol{k}}.
\end{equation}

\textbf{TODO: put everything together to obtain Heyn08, Eq. (6).}
\end{document}
