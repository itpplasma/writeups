%% LyX 2.3.2 created this file.  For more info, see http://www.lyx.org/.
%% Do not edit unless you really know what you are doing.
\documentclass[11pt,english,nofootinbib, tightenlines, notitlepage]{revtex4-1}
\usepackage{tgpagella}
\usepackage[T1]{fontenc}
\usepackage[latin9]{inputenc}
\setcounter{secnumdepth}{3}
\setlength{\parskip}{\smallskipamount}
\setlength{\parindent}{0pt}
\usepackage{babel}
\usepackage{amsmath}
\usepackage{amssymb}
\usepackage{cancel}
\usepackage{setspace}
\usepackage{esint}
\usepackage{microtype}
\usepackage[unicode=true,
 bookmarks=true,bookmarksnumbered=false,bookmarksopen=false,
 breaklinks=false,pdfborder={0 0 1},backref=false,colorlinks=false]
 {hyperref}
\hypersetup{pdftitle={Plasma Physics},
 pdfauthor={Christopher Albert}}
\begin{document}
\global\long\def\tht{\vartheta}%
\global\long\def\ph{\varphi}%
\global\long\def\balpha{\boldsymbol{\alpha}}%
\global\long\def\btheta{\boldsymbol{\theta}}%
\global\long\def\bJ{\boldsymbol{J}}%
\global\long\def\bGamma{\boldsymbol{\Gamma}}%
\global\long\def\bOmega{\boldsymbol{\Omega}}%
\global\long\def\d{\mathrm{d}}%
\global\long\def\t#1{\text{#1}}%
\global\long\def\m{\text{m}}%
\global\long\def\v#1{\boldsymbol{#1}}%

\global\long\def\t#1{\mathbf{#1}}%
\global\long\def\tT{\boldsymbol{\mathsf{T}}}%
\global\long\def\tI{\boldsymbol{\mathsf{I}}}%
\global\long\def\tP{\boldsymbol{\mathsf{P}}}%
\global\long\def\vv{\boldsymbol{v}}%

\title{Plasma Physics}
\author{Christopher Albert}
\date{\today}

\maketitle
This document is a write-up for the plasma physics course at TU Graz
with the goal of becoming the new lecture notes. While derivations
in the original lecture notes are kept short, they are listed more
extensively here for better understanding. Before going to the actual
problems, we summarize some general concepts that are required for
treating them. Most of those concepts have a much wider application
than plasma physics, so it's a good idea to be familiar with them.

\tableofcontents{}

\section*{Common Concepts}

\subsection*{Concept 1: Tensor calculus}

First we summarize some rules of tensor calculus that are required
for certain tasks within plasma physics. A summary of rules of vector
and tensor calculus is available in the NRL plasma formulary. Some
rules are

BAC-CAB rule: 
\begin{equation}
\v a\times(\v b\times\v c)=(\v c\times\v b)\times\v a=\v b(\v a\cdot\v c)-\v c(\v a\cdot\v b)
\end{equation}
The scalar product is not commutative for tensors:

\begin{equation}
\v a\cdot\nabla\v b=(\v a\cdot\nabla)\v b=\v a\cdot(\nabla\v b)\neq(\nabla\v b)\cdot\v a
\end{equation}

Cross-product with a curl:

\begin{equation}
\v a\times(\nabla\times\v b)=(\nabla\v b)\cdot\v a-\v a\cdot\nabla\v b\label{eq:eq3}
\end{equation}
Divergence of a dyad:
\begin{equation}
\nabla\cdot(\v a\v b)=(\v a\cdot\nabla)\v b+(\nabla\cdot\v a)\v b\label{eq:diaddiv}
\end{equation}

Those rules can be deduced from using index notation for the respective
operations. For application in research and engineering it is important
to note that tensor calculus can be formulated in a flexible and coordinate-independent
way within the co-contravariant formalism. More detailed derivations
on this very convenient way of treating curvilinear coordinates can
be found in the first chapters of the book of Callen/d'haeseleer \citep{Callen1991}. 

\subsubsection*{Taylor expansion of vector fields\label{subsec:Taylor-expansion-of}}

The second order expansion of a vector field $\v A(\v r)$ around
a point $\v R$ is given by
\begin{align}
\v A(\v R+\v{\rho}) & =\v A(\v R)+(\v{\rho}\cdot\nabla)\v A(\v R)+(\v{\rho}\v{\rho}:\nabla\nabla))\v A(\v R)+\mathcal{O}(\rho^{3}).
\end{align}
Here $\v{\rho}\v{\rho}$ and $\nabla\nabla$ are two dyades combined
by a scalar product. where differentiation doesn't act on $\v{\rho}$.
In index notation this means
\begin{equation}
A_{i}(\v R+\v{\rho})=A_{i}(\v R)+\rho_{j}\partial_{j}A_{i}(\v R)+(\rho_{k}\rho_{l}\partial_{k}\partial_{l})A_{i}(\v R)+\mathcal{O}(\rho^{3}).
\end{equation}
The convention is that a sum is taken automatically over indexes appearing
twice, and we skip the sum sign. If we consider the dyade
\begin{equation}
(\nabla\v A)_{kl}=\partial_{k}A_{l}.
\end{equation}
We see that the first order term can be written without brackets as
\begin{equation}
(\v{\rho}\cdot\nabla)\v A=\v{\rho}\cdot\nabla\v A.
\end{equation}

For the second order term we combine
\begin{equation}
\rho_{k}\rho_{l}\partial_{k}\partial_{l}=(\rho_{k}\partial_{k})(\rho_{l}\partial_{l}).
\end{equation}
Thus it can also be written as a dyad itself via
\begin{equation}
\v{\rho}\v{\rho}:\nabla\nabla=(\v{\rho}\cdot\nabla)(\v{\rho}\cdot\nabla),
\end{equation}
where derivatives don't act on $\v{\rho}$. As an example we can look
at Cartesian coordinates $(x^{1},x^{2})$ in 2D and a vector component
\begin{align*}
A_{i}(\v R+\v{\rho}) & =A_{i}(\v R)\\
 & +\rho^{1}\partial_{1}A_{i}(\v R)+\rho^{2}\partial_{2}A_{i}(\v R)\\
 & +(\rho^{1})^{2}\partial_{11}A_{i}(\v R)+2\rho^{1}\rho^{2}\partial_{12}A_{i}(\v R)+(\rho^{2})^{2}\partial_{22}A_{i}(\v R)\\
 & +\mathcal{O}(\rho^{3}).
\end{align*}
Here we used the shorthand notation $\partial_{k}\partial_{l}\equiv\partial_{kl}$.
Since $\rho^{1}\rho^{2}\partial_{12}=\rho^{2}\rho^{1}\partial_{21}$
a coefficient $2$ appears in front of this term in the second order
term of the Taylor expansion. Going to higher orders, such factors
will be binomial coefficients in 2D and multinomial coefficients in
$N$-D.

\subsection*{Concept 2: Fluid mechanics}

Let's look at the general form of fluid equations. The derivation
relies on the intuitive picture what a ``flux'' is. A rigorous derivation
of fluid equations can be performed from individual particles via
kinetic theory.

We start with the conservation law for mass density $\rho_{m}(\v r,t)$
known as the continuity equation. In a fluid moving with velocity
$\v v(\v r,t)$, mass is transported by the mass flux 
\begin{equation}
\boldsymbol{\Gamma}_{m}(\v r,t)=\rho_{m}(\v r,t)\v v(\v r,t).
\end{equation}
The mass flux is a vector field parallel to $\v v(\v r,t)$ and of
scales with the mass available to move in $\v x$ quantified by $\rho_{m}(\v r,t)$.
Now we consider a volume $\Omega$ fixed to the lab frame.\footnote{This is the Eulerian picture, as opposed to the Lagrangian picture,
where a volume is moved and distorted together with the fluid flow. } We compute the total mass inside $\Omega$ via a volume integral
over $\rho_{m}$. If the system's total mass should be conserved,
a change in mass in the volume $\Omega$ can only be caused by an
in-/outflux of mass via $\v{\Gamma}_{m}$ over its boundary surface
$\partial\Omega$. Mathematically we formulate this as a relation
between a partial time derivative over the total mass and a surface
integral over $\v{\Gamma}_{m}$ and obtain the law of\textbf{}\\
\textbf{Mass conservation (integral)}

\begin{equation}
\frac{\partial}{\partial t}\int_{\Omega}\rho_{m}(\v r,t)\d V+\int_{\partial\Omega}\boldsymbol{\Gamma}_{m}(\v r,t)\cdot\d\v S=0.
\end{equation}
Note that a positive flux $\boldsymbol{\Gamma}_{m}$ points outwards,
such that it needs to be compensated by a reduction of integral mass
via its partial time derivative. Now we can exchange derivative and
integral, as $\Omega$ is fixed during time and use the divergence
theorem to obtain
\begin{equation}
\int_{\Omega}\frac{\partial\rho_{m}(\v r,t)}{\partial t}+\nabla\cdot\boldsymbol{\Gamma}_{m}(\v r,t)\d V=0.
\end{equation}
Since this relation is true independent on the choice of $\Omega$,
it must hold for the integrand. Thus we obtain the law of\textbf{\smallskip{}
}\\
\textbf{Mass conservation (differential) / continuity equation}
\begin{equation}
\frac{\partial\rho_{m}(\v r,t)}{\partial t}+\nabla\cdot\boldsymbol{\Gamma}_{m}(\v r,t)=0.
\end{equation}
or explicitly
\begin{equation}
\frac{\partial\rho_{m}(\v r,t)}{\partial t}+\nabla\cdot(\rho_{m}(\v r,t)\v v(\v r,t))=0.
\end{equation}
For a single species of particles of mass $m$ we can write $\rho_{m}(\v r,t)=m\,n(\v r,t)$
via the number density $n(\v x,t)$ and write the law of\textbf{\smallskip{}
}\\
\textbf{Number density conservation}
\begin{equation}
\frac{\partial n(\v r,t)}{\partial t}+\nabla\cdot(\v v(\v r,t)n(\v r,t))=0.
\end{equation}

The second conserved quantity we are interested in is the momentum
density $\boldsymbol{p}$ in absence of external forces. Since momentum
is a vector, its flux $\boldsymbol{\Gamma}_{\v p}$ must be a rank-2
tensor. The result is still the same as for mass or any other conservation
law, yielding the\textbf{\smallskip{}
}\\
\textbf{Total momentum conservation} \textbf{law (differential)}
\begin{equation}
\frac{\partial\boldsymbol{p}(\v r,t)}{\partial t}+\nabla\cdot\boldsymbol{\Gamma}_{\v p}(\v r,t)=0.
\end{equation}
In case of external forces, they would appear as a \emph{source} on
the right-hand side as force densities $\boldsymbol{f}$. The overall
momentum density consists of \emph{kinematic fluid} $\v p^{\text{fl}}=\rho_{m}\v v$
and \emph{electromagnetic} momentum density $\v p^{\text{EM}}=\boldsymbol{S}/c^{2}$,
where $\boldsymbol{S}$ is the Poynting vector. The momentum flux
$\boldsymbol{\Gamma}_{\v p}$ consists of a fluid part\footnote{Here $\t T=\boldsymbol{v}\v w$ means the outer product $\boldsymbol{v}\otimes\boldsymbol{w}$,
being a rank-2 tensor with Cartesian components $T_{ij}=v_{i}w_{j}$.} $\boldsymbol{\Gamma}_{\v p}^{\text{fl}}=\v v\v p_{\text{fl}}$, but
is also influenced by \emph{thermodynamic} effects, namely the width
of the distribution around the average velocity $\boldsymbol{v}$.
We separate out \emph{electrodynamic} and \emph{thermodynamic} effects
in the total momentum conservation law, such that they appear as forces
on the right-hand side, leading to the\textbf{\smallskip{}
}\\
\textbf{Kinematic fluid momentum law}
\begin{equation}
\frac{\partial\v p^{\mathrm{fl}}(\v r,t)}{\partial t}+\nabla\cdot(\v v(\v r,t)\v p^{\mathrm{fl}}(\v r,t))=-\nabla p(\v r,t)+q\left(\v E(\v r,t)+\v v(\v r,t)\times\v B(\v r,t)\right).
\end{equation}
In Cartesian coordinates the divergence of a rank-2 tensor $\t T$
is a column vector whose components are the divergence of tensor columns,
so
\begin{equation}
\nabla\cdot\t T=\left(\begin{array}{c}
\nabla\cdot\boldsymbol{T}_{\cdot1}\\
\nabla\cdot\boldsymbol{T}_{\cdot2}\\
\nabla\cdot\boldsymbol{T}_{\cdot3}
\end{array}\right)=\left(\begin{array}{c}
\frac{\partial}{\partial x^{1}}T_{11}+\frac{\partial}{\partial x^{2}}T_{21}+\frac{\partial}{\partial x^{3}}T_{31}\\
\frac{\partial}{\partial x^{1}}T_{12}+\frac{\partial}{\partial x^{2}}T_{22}+\frac{\partial}{\partial x^{3}}T_{32}\\
\frac{\partial}{\partial x^{1}}T_{13}+\frac{\partial}{\partial x^{2}}T_{23}+\frac{\partial}{\partial x^{3}}T_{33}
\end{array}\right).
\end{equation}
Now we can further modify this using mass continuity by using a rule
from tensor calculus to write the flux vector $\nabla\cdot\boldsymbol{\Gamma}_{\v p}^{\text{fl}}$
as
\begin{align}
\nabla\cdot\boldsymbol{\Gamma}_{\v p}^{\mathrm{fl}}=\nabla\cdot(\v v\v p^{\mathrm{fl}}) & =\v p^{\mathrm{fl}}(\nabla\cdot\v v)+(\v v\cdot\nabla)\v p^{\mathrm{fl}}.
\end{align}
Here, $\boldsymbol{v}\cdot\nabla$ performs a convective derivative,
i.e. 
\begin{equation}
(\v v\cdot\nabla)\boldsymbol{w}=v_{1}\frac{\partial}{\partial x^{1}}\boldsymbol{w}+v_{2}\frac{\partial}{\partial x^{2}}\boldsymbol{w}+v_{3}\frac{\partial}{\partial x^{3}}\boldsymbol{w}.
\end{equation}
Now let's look back and see that \emph{incidentally }(well, not entirely)
the \emph{fluid momentum density} $\v p^{\mathrm{fl}}=\rho_{m}\v v=\boldsymbol{\Gamma}_{m}$
matches the \emph{mass flux} defined before. 

\section{Basics}

\subsection*{1.1 Criteria for a plasma}

\subsection*{Problem 1: Work needed to introduce charge density fluctuation}

see lecture notes

\subsection*{Problem 2: Debye shielding - the easy way}

see lecture notes

\subsection*{Problem 3: Debye shielding - the hard way}

see lecture notes

Helmholtz equation
\begin{equation}
\Delta\Phi(\v x)-\frac{1}{\lambda^{2}}\Phi(\v x)=-\frac{Q}{\varepsilon_{0}}\delta(\v x).
\end{equation}
Linear problem and infinite or periodic space: Use spatial Fourier
transform. Be careful about normalization with $1/\sqrt{2\pi}$ (symmetric,
unitary) or $1/(2\pi)$ (non-unitary) on one of either transform or
inverse transform.

1-dimensional Fourier transform of $\delta(x)$: 
\begin{align}
\mathcal{F}\delta(x) & =\frac{1}{\sqrt{2\pi}}\int_{-\infty}^{\infty}\d x\,\delta(x)e^{-ikx}\nonumber \\
 & =\frac{e^{-ik0}}{\sqrt{2\pi}}=\frac{1}{\sqrt{2\pi}}.
\end{align}
$N$-dimensional Fourier transform of $\delta(\v x)$:
\begin{align}
\mathcal{F}_{N}\delta(\v x) & =\frac{1}{(2\pi)^{N/2}}\int_{-\infty}^{\infty}\d^{N}x\,\delta(\v x)e^{-i\v k\cdot\v x}\nonumber \\
 & =\frac{e^{-i\v k\cdot\v 0}}{(2\pi)^{N/2}}=\frac{1}{(2\pi)^{N/2}}.
\end{align}
So in our 3D equation:
\begin{equation}
-k^{2}\Phi(\v k)-\frac{1}{\lambda^{2}}\Phi(\v k)=-\frac{Q}{(2\pi)^{3/2}\varepsilon_{0}}.
\end{equation}
Solution in $\v k$-space:
\begin{equation}
\Phi(\v k)=\frac{Q}{(2\pi)^{3/2}\varepsilon_{0}(k^{2}+1/\lambda_{D}^{\,2})}.
\end{equation}
Solution in real space:
\begin{align}
\Phi(\v x)=\mathcal{F}_{3}^{\,-1}\Phi(\v k) & =\frac{1}{(2\pi)^{3/2}}\int_{-\infty}^{\infty}\d^{3}k\,\Phi(\v k)\\
 & =\frac{1}{(2\pi)^{3/2}}\int_{-\infty}^{\infty}\d^{3}k\,\frac{Q}{(2\pi)^{3/2}\varepsilon_{0}(k^{2}+1/\lambda_{D}^{\,2})}e^{i\v k\cdot\v x}.
\end{align}
How do we represent solve this? Represent scalar product by
\begin{equation}
\v k\cdot\v x=|\v k|\,|\v x|\cos\tht_{(k)}=kr\cos\tht_{kx}
\end{equation}
where $k=|\v k|$, $r=|\v x|$ and $\tht_{kx}$ is the angle between
vectors $\v k$ and $\v x$. For the integral we first rotate the
coordinate system of $\v k$-space such that the $z$-axis coincides
with the vector $\v x$. The advantage of this coordinate frame is
that
\begin{equation}
k_{z}=\v k\cdot\frac{\v x}{|\v x|}.
\end{equation}

Now we switch from $(k_{x},k_{y},k_{z})$ to spherical coordinates
$(k,\tht_{(k)},\ph_{(k)})$ where
\begin{align}
k & =|\v k|,\\
k_{z} & =k\cos\tht_{(k)}.
\end{align}
We observe that the a
\begin{equation}
\v k\cdot\v x=kr\cos\tht_{(k)}=rk_{z}.
\end{equation}
With this trick, we can thus represent the term appearing in the complex
exponent as
\begin{equation}
e^{i\v k\cdot\v x}=e^{irk_{z}}=e^{ikr\cos\tht_{(k)}}.
\end{equation}
Now back to our original problem. We compute
\begin{align}
\Phi(\v x) & =\frac{Q}{(2\pi)^{3}\varepsilon_{0}}\int_{0}^{\infty}\d k\int_{0}^{\pi}\d\tht_{(k)}\int_{0}^{2\pi}\d\ph_{(k)}\,k^{2}\sin\tht\frac{k^{2}}{k^{2}+1/\lambda_{D}^{\,2}}e^{ikr\cos\tht_{(k)}}.
\end{align}
The next trick is to substitute
\begin{align}
u= & \cos\tht_{(k)},\\
\d u= & -\sin\tht_{(k)}\d\tht_{(k)}.
\end{align}
We cancel the negative sign in $\d u$ by swapping the integration
boundaries, which become $(-1,1)$ for $u$ where $\tht_{(k)}=(\pi,0)$.
As there are no dependencies on $\varphi_{(k)}$ integration just
adds a factor of $2\pi$.
\begin{align}
\Phi(\v x) & =\frac{Q}{(2\pi)^{2}\varepsilon_{0}}\int_{0}^{\infty}\d k\frac{k^{2}}{k^{2}+1/\lambda_{D}^{\,2}}\int_{-1}^{1}\d ue^{ikru}\\
 & =\frac{Q}{(2\pi)^{2}\varepsilon_{0}}\int_{0}^{\infty}\d k\frac{k^{2}}{k^{2}+1/\lambda_{D}^{\,2}}\frac{1}{ikr}(e^{ikr}-e^{-ikr})
\end{align}
The next trick is to replace the integral over the second term $e^{-ikr}$
by setting $k\rightarrow-k$, and include it by extending the remaining
integral in the range $(-\infty,0)$,
\begin{equation}
\Phi(\v x)=\frac{Q}{ir(2\pi)^{2}\varepsilon_{0}}\int_{-\infty}^{\infty}\d k\frac{k}{k^{2}+1/\lambda_{D}^{\,2}}e^{ikr}.
\end{equation}
We see that the integral is now an inverse 1D Fourier transform of
an expression of the form $k/(k^{2}+a^{2})$ that we can look up to
be
\begin{equation}
\mathcal{F}^{(-1)}\frac{a}{k^{2}+a^{2}}=\sqrt{\frac{\pi}{2}}e^{-a|x|},
\end{equation}
so taking one derivative
\begin{equation}
\mathcal{F}^{(-1)}\frac{ik}{k^{2}+a^{2}}=\sqrt{\frac{\pi}{2}}\frac{1}{a}\frac{\d}{\d x}e^{-a|x|}=-\sqrt{\frac{\pi}{2}}e^{-a|x|}.
\end{equation}
Finally, with $a=1/\lambda_{D}$, and $x=r>0$ we obtain 
\begin{align}
\Phi(\v x) & =\frac{Q}{ir(2\pi)^{2}\varepsilon_{0}}\frac{\sqrt{2\pi}}{i}\mathcal{F}^{(-1)}\frac{ik}{k^{2}+1/\lambda_{D}^{\,2}}\nonumber \\
 & =\frac{Q}{4\pi\varepsilon_{0}r}e^{-r/\lambda_{D}}.
\end{align}

\begin{itemize}
\item Summary: 
\begin{itemize}
\item We computed \emph{potential }of \emph{point charge} inside \emph{electron
fluid} with a certain \emph{temperature }where electrons are allowed
to \emph{move} and thus \emph{shield} the potential.
\item First we used vortex-free \emph{momentum law} (Euler/Navier-Stokes)
to compute \emph{local} \emph{Boltzmann law} for \emph{electron density}
according to $V$, $\Phi$ and $T$.
\item Then we inserted convection-free ($V=0$) case into \emph{Poisson
equation}\textbf{\emph{ }}and \emph{linearized} to obtain linear \emph{Helmholtz
equation}, where \emph{Debye length }$\lambda_{D}$ appeared.
\item We solved the resulting linear equation via a spatial \emph{Fourier
transform} and a few tricks.
\item The result is the usual potential of a point charge weighted by an
\emph{exponential decaying} term on the scale of $\lambda_{D}$.
\item The interpretation is, that outside of $\lambda_{D}$, perturbations
in charge density become \emph{invisible} $\rightarrow$ \emph{Debye
shielding}.
\end{itemize}
\item Remember: Fourier \emph{transform} works only in an infinite domain
$\mathbb{R}^{N}$ where no boundary conditions are imposed, since
we need to take integrals up to infinity. Fourier \emph{series} are
similar and work with periodic boundary conditions (lattice) or topology
(cylinder, torus, sphere). Both methods work only well for \emph{linear}
equations $\rightarrow$ no coupling between different Fourier harmonics.
\item Remark: Here we could see that notation introduces quite some ambiguity.
We often identify the coordinates of a point $\v x$ with the coordinate
system itself. If we have another point $\v k$ and switch to spherical
coordinates $(r,\tht,\ph)$ we cannot distinguish $r$ from $|\v x|$,
even if it could be the radius $|\v y|$ of another point $\v y$.
To resolve this, we gave different names to our coordinates $(k,\tht_{(k)},\ph_{(k)})$
in $k$-space. We could also have introduced the notation $r_{\v x}$
and $r_{\v k}$, which is non-standard and maybe even more confusing.
\end{itemize}

\subsection*{Problem 4: Electron plasma oscillations - the easier way}

see lecture notes

\subsection*{Problem 5: Electron plasma oscillations - the harder way}

We look at oscillations/waves in an unmagnetized plasma due to electrostatic
forces.

Approximations:
\begin{enumerate}
\item Ions are resting compared to fast electron dynamics.
\item Electrons modeled as fluid.
\item Electrostatic forces are stronger than thermodynamics forces (``cold
plasma'').
\item Linear expansion around equilibrium with subsonic flow and quasineutrality.
\end{enumerate}
Take mass continuity and momentum equation for electron fluid with
particles of mass $m_{e}$ and charge $q_{e}=-e$, as well as Poisson
equation for electric potential and constant and static ion density
$n_{i}$,
\begin{align}
m_{e}\frac{\partial n_{e}(\v x,t)}{\partial t}+\nabla\cdot\v p_{\text{fl}}(\v x,t) & =0,\label{eq:conti}\\
\frac{\partial\v p_{\text{fl}}(\v x,t)}{\partial t}+m_{e}\nabla\cdot(\v v(\v x,t)\v p_{\text{fl}}(\v x,t)) & =-\nabla p(\v x,t)+e\nabla\Phi(\v x,t),\\
\Delta\Phi(\v x,t) & =-\frac{e}{\varepsilon_{0}}(n_{i}-n_{e}(\v x,t)).\label{eq:poiss}
\end{align}
Here we could pull out $m_{e}$ as a constant, and replaced mass flux
$\v{\Gamma}_{m}$ by the identical kinematic momentum density $\v p^{fl}$.
In order to combine all three equations we now take the time derivative
of the continuity equation, and the divergence of the momentum equation
to obtain
\begin{align}
m_{e}\frac{\partial^{2}n_{e}(\v x,t)}{\partial t^{2}}+\frac{\partial}{\partial t}\nabla\cdot\v p_{\text{fl}}(\v x,t) & =0,\label{eq:conti-1}\\
\nabla\cdot\frac{\partial\v p_{\text{fl}}(\v x,t)}{\partial t}+\nabla\cdot(\nabla\cdot(\v v(\v x,t)\v p_{\text{fl}}(\v x,t))) & =-\Delta p(\v x,t)+en_{e}(\v x,t)\Delta\Phi(\v x,t),\\
\Delta\Phi(\v x,t) & =-\frac{e}{\varepsilon_{0}}(n_{i}-n_{e}(\v x,t)).\label{eq:poiss-1}
\end{align}
Exchanging divergence and time derivative in front of $\v p^{fl}$
we can eliminate the respective terms by subtracting the momentum
equation from the continuity equation. The Laplacian $\Delta\Phi$
is substituted from Poisson's equation to yield a single equation
\begin{align}
\frac{\partial^{2}n_{e}(\v x,t)}{\partial t^{2}}+\frac{e^{2}n_{e}(\v x,t)}{m_{e}\varepsilon_{0}}(n_{e}(\v x,t)-n_{i}(\v x))-\nabla\cdot(\nabla\cdot(\v v(\v x,t)\v v(\v x,t))) & =\frac{\Delta p(\v x,t)}{m_{e}}.
\end{align}
Now we start with a stationary equilibrium without time derivatives,
and flow velocity small enough (``subsonic'') such that the term
$\nabla\cdot(\nabla\cdot(\v v(\v x,t)\v v(\v x,t)))$ can be neglected.
The equilibrium should fulfill quasineutrality,
\begin{equation}
n_{0}(\v x)=n_{e}^{0}(\v x)=n_{i}^{0}(\v x),
\end{equation}
so $\Delta p^{0}=0$ in the subsonic approximation.

Now we introduce a perturbation in $n_{e}(\v x,t)=n_{e}^{0}(\v x)+\delta n_{e}(\v x,t)$
and $\v v=\v v^{0}+\delta\v v(\v x,t)$, but neglect changes in $p$
(``cold plasma''). Keeping only first order terms (remember, $n_{e}^{0}-n_{i}^{0}=0$),
we obtain
\begin{align}
\frac{\partial^{2}\delta n_{e}(\v x,t)}{\partial t^{2}}+\frac{e^{2}n_{0}(\v x)}{m_{e}\varepsilon_{0}}\delta n_{e}(\v x,t) & =0.
\end{align}
Here the electron plasma frequency appears locally at position $\v x$
as
\begin{equation}
\omega_{\text{pe}}^{\,2}(\v x)=\frac{e^{2}n_{0}(\v x)}{m_{e}\varepsilon_{0}}.
\end{equation}


\section{Collisions}

\subsection*{2.1 Binary Coulomb Collisions}

see lecture notes

\section{Charged Particle Dynamics}

\subsection*{3.2 Motion in static and homogeneous fields}

Our \textbf{goal} is to solve
\begin{align}
m\ddot{\v r}(t) & =\frac{e}{c}\dot{\v r}(t)\times\v B+e\v E.\label{eq:Lorentz}
\end{align}
or written in terms of a differential operator describing Lorentz
force:
\begin{equation}
\hat{L}\v r\equiv\left(\frac{\d^{2}}{\d t^{2}}+\left(\text{\ensuremath{\frac{e\v B}{mc}\frac{\d}{\d t}}}\right)\times\right)\v r(t)=\frac{e}{m}\v E\label{eq:LorentzOperator}
\end{equation}
and identify\emph{ gyromotion} circling perpendicular around $\v B$
with $v_{\perp}$,

\emph{parallel }motion along $\v B$ with $v_{\parallel}$ and

average \emph{drift} motion perpendicular to $\v B$ with $\v v_{D}$.\\

Our \textbf{strategy} is to separate out gyromotion in $\v B$-field
alone and solve its equations first. Then we could solve parallel
and drift motion with the constraint that $\v v_{D}$ remains constant.
In the combined solution we verified there are 6 free parameters that
can reproduce any initial conditions.\\
~

The\textbf{ assumption}\textbf{\emph{ }}is that we considere only
static and homogeneous fields. Under this simplification the solution
describes particle motion exactly, as $\hat{L}$ in Eq.~(\ref{eq:LorentzOperator})
is \emph{linear} in \textbf{$\v r(t)$}, so the overall solution can
be described as a linear superposition of two solutions.

Splitting Eq.~(\ref{eq:Lorentz}) into a sum of two solutions $\v R+\v{\rho}$
with $\v r=\v R+\v{\rho}$ yields 
\begin{align}
m(\ddot{\v R}+\ddot{\v{\rho}}) & =e\v E+\frac{e}{c}(\dot{\v R}+\dot{\v{\rho}})\times\v B.\label{eq:split}
\end{align}
Since we introduced 3 more variables, we can also introduce 3 more
equations, as long as they are not in conflict with Eq. (\ref{eq:split}).
We choose
\begin{align}
m\ddot{\v{\rho}} & =\frac{e}{c}\dot{\v{\rho}}\times\v B,\\
m\ddot{\v R} & =e\v E+\frac{e}{c}\dot{\v R}\times\v B.
\end{align}
This way, the gyromotion in $\v{\rho}$ is only affected by the magnetic,
but not the electric field.

\subsubsection*{Solution for gyration}

First we solve $m\ddot{\v{\rho}}=\frac{e}{c}\dot{\v{\rho}}\times\v B$.
We choose our coordinate system such that field $\v B=B\v b=B\v e_{1}\times\v e_{2}$.
(or: $\v e_{3}=\v b$) Then
\begin{align}
m\ddot{\v{\rho}} & =\frac{e}{c}\dot{\v{\rho}}\times(B\v b)=\frac{eB}{c}(\dot{\rho}_{1}\v e_{1}+\dot{\rho}_{2}\v e_{2})\times\v e_{3}\nonumber \\
 & =\frac{eB}{c}(-\dot{\rho}_{1}\v e_{2}+\dot{\rho}_{2}\v e_{1}).
\end{align}
We combine this to
\begin{align}
\ddot{\rho}_{1} & =\frac{eB}{mc}\dot{\rho}_{2}=\omega_{c}\dot{\rho}_{2},\\
\ddot{\rho}_{2} & =-\frac{eB}{mc}\dot{\rho}_{1}=-\omega_{c}\dot{\rho}_{1}.
\end{align}
Taking another time derivative and denoting $v_{1}\equiv\dot{\rho}_{1}$,
$v_{2}\equiv\dot{\rho}_{2}$ we obtain
\begin{equation}
\ddot{v}_{1}=\omega_{c}\dot{v}_{2}=-\omega_{c}^{\,2}v_{1}.
\end{equation}
In this second order equation we have six free constants. We choose
one to be the \textbf{perpendicular velocity} $v_{\perp}$, which
is the amplitude of the gyration velocity and another one as a phase-shift
$\phi_{0}$. Then we obtain the solution of position, velocity, and
acceleration
\begin{align}
\v{\rho} & =\frac{v_{\perp}}{\omega_{c}}(\sin(\omega_{c}t+\phi_{0})\,\v e_{1}+\cos(\omega_{c}t+\phi_{0})\,\v e_{2}),\\
\dot{\v{\rho}} & =v_{\perp}(\cos(\omega_{c}t+\phi_{0})\,\v e_{1}-\sin(\omega_{c}t+\phi_{0})\,\v e_{2}),\\
\ddot{\v{\rho}} & =-v_{\perp}\omega_{c}(\sin(\omega_{c}t+\phi_{0})\,\v e_{1}+\cos(\omega_{c}t+\phi_{0})\,\v e_{2}).
\end{align}
From here we define the \textbf{gyroradius} $\rho_{c}=v_{\perp}/\omega_{c}$
and the \textbf{gyrophase} $\phi=\omega_{c}t+\phi_{0}$.\\
\\
Additional four integration constants are a translational shift by
$\v{\rho}_{0}$ plus a constant parallel velocity $\dot{\v{\rho}}_{0}\cdot\v b$
which we both set to zero! This means that we didn't include any motion
parallel to the magnetic field, or drift across the field intro $\v{\rho}$
and need to account for them in $\v R$.\\
\\
(Have $6\times2$ integration constants in $\v{\rho},\v R$ but need
only $6$ of them for overall problem in $\v r$).

\subsubsection*{Solution for parallel and drift motion}

Now we split
\begin{equation}
\dot{\v R}=v_{\parallel}\v b+\v v_{D},\qquad\v v_{D}\cdot\v b=0.
\end{equation}
We need to solve
\begin{equation}
m\ddot{\v R}=\dot{v}_{\parallel}\v b+\dot{\v v}_{D}=e\v E+\frac{e}{c}\dot{\v R}\times\v B.
\end{equation}
For parallel motion, obtain immediately
\begin{equation}
m\dot{v}_{\parallel}=eE_{\parallel}\Rightarrow v_{\parallel}(t)=v_{\parallel0}+eE_{\parallel}t.
\end{equation}
For perpendicular drift use
\begin{equation}
m\dot{\v v}_{D}=e\v E_{\perp}+\frac{e}{c}\v v_{D}\times\v B
\end{equation}
take $\times\v B$ and use bac-cab rule (NRL)
\begin{equation}
m\dot{\v v}_{D}\times\v B=e\v E\times\v B-\frac{e}{c}B^{2}v_{D}.
\end{equation}

Consistent solution is possible if $\dot{\v v}_{D}\times\v B$ remains
zero. This way we do not include any gyration in the perpendicular
motion. Solution for this drift motion (\textbf{ExB drift}):
\begin{equation}
\v v_{D}=c\frac{\v E\times\v B}{B^{2}}.
\end{equation}
\emph{... independent of charge and mass! }This also means setting
initial condition $v_{D0}=c\frac{\v E\times\v B}{B^{2}}$ which fixes
another integration constant. 

The solution for parallel and drift motion together is thus

\begin{equation}
\v R(t)=\v R_{0}+c\frac{\v E\times\v B}{B^{2}}t+\left(v_{\parallel0}t+\frac{eE_{\parallel}}{2}t^{2}\right)\v b.
\end{equation}


\subsubsection*{Combined solution}

The combined exact solution for particle motion in static, homogeneous
$\v E$ and $\v B$ is
\begin{align}
\v r(t) & =\v R(t)+\v{\rho}(t)\nonumber \\
 & =\v R_{0}+c\frac{\v E\times\v B}{B^{2}}t+\left(v_{\parallel0}t+\frac{eE_{\parallel}}{2}t^{2}\right)\v b+\frac{v_{\perp}}{\omega_{c}}\left(\sin(\omega_{c}t+\phi_{0})\,\v e_{1}+\cos(\omega_{c}t+\phi_{0})\,\v e_{2}\right).
\end{align}

~

Let's count integration constants: 
\begin{itemize}
\item 1,2,3: $\v R_{0}$ contains 3 constants of initial position of Larmor
circle center (\textbf{gyrocenter}).
\item 4: Constant $v_{\parallel0}$ sets initial gyrocenter parallel velocity.
\item 5,6: Constants $v_{\perp}$ and $\phi_{0}$ fix the gyromotion ($v_{\perp}$
can optionally be replaced by $\rho_{c}$).\\
\end{itemize}
Thus we could find a consistent solution that separates \emph{gyromotion}
from \emph{parallel} and \emph{drift} motion in \emph{static} and
\emph{homogeneous} fields. In more complicated fields we will rely
on methods of averaging that work as long as gyromotion i.e. $\omega_{c}$
is fast, and $\rho_{c}$ is small, compared to field scales.

\subsection*{3.4 Motion in weakly inhomogeneous fields\label{subsec:Inhomo}}

Now the \textbf{goal} is to solve
\begin{align}
m\ddot{\v r} & =\frac{e}{c}\dot{\v r}\times\v B(\v r)+e\v E(\v r).
\end{align}
or written in terms of a \emph{non-linear} Lorentz operator:
\begin{equation}
\hat{L}\v r\equiv\left(\frac{\d^{2}}{\d t^{2}}+\frac{e\v B(\v r)}{mc}\frac{\d}{\d t}\times\right)\v r=\frac{e}{m}\v E(\v r).
\end{equation}

\textbf{Strategy:} Instead of an exact splitting as before, separate
into \emph{fast} gyromotion for $\v{\rho}$ and \emph{gyro-averaged}
motion in $\v R$. The latter means to take time averages over\emph{
gyroperiod} $\tau_{c}=2\pi/\omega_{c}$ and spatial averages within
\emph{gyroradius $\rho_{c}=v_{\perp}/\omega_{c}$. }Perform a series
expansion of fields around average $\v R$ in direction of $\v{\rho}$.\\
~

We\textbf{ assume}\textbf{\emph{ }}that relative variations of $\v E$
and $\v B$ in time and space are small over $\tau_{c}$ and within
$\rho_{c}$. This means that the plasma must be \emph{strongly magnetized
}to get reasonably large $\omega_{c}=eB/(mc)$. Due to the expansion
the solution is only \textbf{approximate}\emph{.}

\subsubsection*{Calculation}

A detailed calculation can be found in the lecture notes, chapter
3.4. For details how to do an expansion of tensor fields, see p.~\pageref{subsec:Taylor-expansion-of}.

\subsubsection*{Result}

The orbital motion is split into the sum of the gyrocenter position
$\v R$ and the Larmor gyration vector $\v{\rho}$, where

\begin{align}
\v{\rho} & =\frac{v_{\perp}}{\omega_{c}(\v R)}(\sin(\phi(\v R,t))\,\v e_{1}(\v R)+\cos(\phi(\v R,t))\,\v e_{2}(\v R)),\\
m\ddot{\v R} & =e\left(1+\frac{\rho_{c}^{2}}{4}\nabla_{\perp}^{\,2}\right)\v E(\v R)+\frac{e}{c}\dot{\v R}\times\v B(\v R)-\mu\nabla B(\v R).\label{eq:R}
\end{align}
Fields are evaluated in the gyrocenter position $\v R$ that can be
computed ignoring $\v{\rho}$. The magnetic moment $\mu$ is a conserved
quantity and related to $v_{\perp}$ via
\begin{equation}
\mu=\frac{mv_{\perp}^{\,2}}{2B(\v R)}=\frac{ev_{\perp}}{2\pi\rho_{c}}\pi\rho_{c}^{\,2}.\label{eq:magmoment}
\end{equation}
This is the ring current over one gyration period and $\mu=I\cdot A$
over the gyration circle with surface area $A=\pi\rho_{c}^{2}$. If
$E_{\parallel}=\v b\cdot\v E$ then $\mu$ is approximately conserved
-- an \textbf{adiabatic invariant}. 
\begin{align}
\frac{\d\mu}{\d t} & =\frac{1}{B}\frac{\d}{\d t}\left(\frac{mv_{\perp}^{\,2}}{2}\right)-\frac{mv_{\perp}^{\,2}}{2B^{2}}\dot{\v R}\cdot\nabla B\nonumber \\
 & \approx\frac{1}{B}\frac{\d}{\d t}\left(\frac{mv_{\perp}^{\,2}}{2}\right)-\frac{\mu}{B}v_{\parallel}\v b\cdot\nabla B.
\end{align}
Compare this with total energy conservation
\begin{align}
0 & =\frac{\d}{\d t}\left(\frac{mv_{\parallel}^{\,2}}{2}+\frac{mv_{\perp}^{\,2}}{2}+e\Phi\right)\nonumber \\
 & \approx\frac{\d}{\d t}\left(\frac{mv_{\perp}^{\,2}}{2}\right)-v_{\parallel}\mu\v b\cdot\nabla B.
\end{align}
Here $v_{\parallel}\v b\cdot\nabla\Phi$ has vanished due to $E_{\parallel}=0$
and $\dot{v}_{\parallel}$ has been replaced by its expression from
the equations of motion (\ref{eq:R}) with $\dot{\v R}\approx v_{\parallel}\v b$.
Together we see that the total time derivative of $\mu$ vanishes
in this first order calculation. A subtle point that should concern
us is the actual nature of $v_{\perp}$ and $\phi$. If we assume
$\mu$ to be exactly conserved, Eq.~(\ref{eq:magmoment}) becomes
a definition for $v_{\perp}=v_{\perp}(\v R)$ rather than a quantity
resulting from it. While the motion of the gyrocenter $\v R$ can
be computed from Eq.~(\ref{eq:R}) without referring to the gyromotion,
the definition what $\v R$ actually describes is still somewhat dubious.
For this reason we will perform a more ``watertight'' treatment
of the problem using Lagrangian mechanics.

\section{Guiding-Center Dynamics}

A detailed derivation of the guiding-center Lagrangian can also be
found in Lectures 3-4 of 2017 and https://itp.tugraz.at/\textasciitilde ert/assets/internal/plasma/2017/A1\_Guiding\_Center\_Lagrangian.pdf
. More information on asymptotic expansions with a small parameter
$\varepsilon$ can be found in the lecture of Stefan Possanner at
TUM. Here we treat the case where fields are not time-dependent.

We transform from the \textbf{particle} coordinates $\v z_{P}=(\v r,\v v)$
in phase-space to new \textbf{guiding-center} coordinates $\v z=(\v R,\phi,v_{\parallel},v_{\perp})$
related to particle position $\v r$ and velocity $\v v$ via
\begin{align}
\v r(\v z) & =\v R+\v{\rho}(\v R,\phi,v_{\perp}),\text{ where }\v{\rho}(\v R,\phi,v_{\perp})\equiv\frac{mcv_{\perp}}{eB(\v R)}\hat{\v{\rho}}(\v R,\phi)\equiv\rho_{c}(v_{\perp},\v R)\hat{\v{\rho}}(\v R,\phi),\label{eq:rz}\\
\v v(\v z) & =v_{\parallel}\hat{\v b}(\v R)+v_{\perp}\hat{\v n}(\v R,\phi).\label{eq:vz}
\end{align}
Here $\hat{\v b}(\v R)=\v B(\v R)/B(\v R)$ and forms a tripod with
some pair of \textbf{fixed }but possibly space-dependent unit vectors
$\hat{\v e}_{1}(\v R)$ and $\hat{\v e}_{2}(\v R)$, and definitions
for \textbf{dynamic} (gyrating) unit vectors $\hat{\v{\rho}}$ and
$\hat{\v n}$ are
\begin{align}
\hat{\v{\rho}}(\v R,\phi) & =\cos(\phi)\,\hat{\v e}_{1}(\v R)-\sin(\phi)\,\hat{\v e}_{2}(\v R)=-\frac{\partial\hat{\v n}(\v R,\phi)}{\partial\phi},\\
\hat{\v n}(\v R,\phi) & =\hat{\v{\rho}}(\v R,\phi)\times\hat{\v b}(\v R)=-\sin(\phi)\,\hat{\v e}_{1}(\v R)-\cos(\phi)\,\hat{\v e}_{2}(\v R)=\frac{\partial\hat{\v{\rho}}(\v R,\phi)}{\partial\phi}.
\end{align}
The transformation of Eqs.~(\ref{eq:rz}-\ref{eq:vz}) is still exact!
We call $\v R$ the guiding-center position. Both, $(\hat{\v e}_{1},\hat{\v e}_{2},\hat{\v b})$
and $(\hat{\v n},\hat{\v{\rho}},\hat{\v b})$ form an orthonormal
basis, where $\hat{\v e}_{1},\hat{\v e}_{2}$ depend only on $\boldsymbol{R}$,
but $\hat{\v n},\hat{\v{\rho}}$ also on $\phi$. Variables have suspicious
names $\phi$, $v_{\parallel}$ and $v_{\perp}$. Indeed, for the
case of a homogeneous $\v B$-field those variables are equal to gyrophase,
parallel and perpendicular velocity of the particle. If the field
becomes weakly inhomogeneous, they still approximate the particle
values averaged over one gyro-period. Similarly, $\v R$ describes
the center of the Larmor circle only in the homogeneous limiting case,
and remains close to it in a field with sufficiently weak gradients.
To sum up one could say that $(\phi,v_{\parallel},v_{\perp})$ are
values we would get if we pretend that $\v B$ is evaluated in the
guiding-center position $\v R$ rather than the particle position
$\v R+\v{\rho}$. In contrast to the treatment in section~\ref{subsec:Inhomo},
all variables in $\v z=(\v R,\phi,v_{\parallel},v_{\perp})$ have
a well-defined relation to the particle position $\v r$ and velocity
$\v v$ if Eqs.~(\ref{eq:rz}-\ref{eq:vz}) are solved implicitly
in $\v z$. 

The phase-space Lagrangian of a particle in an electromagnetic field
in terms of canonical $(\v r,\v p$) is given by
\begin{equation}
L(\v r,\dot{\v r},\v p,\cancel{\dot{\v p}})=\v p\cdot\dot{\v r}-\left(\frac{(\v p-\frac{e}{c}\v A(\v r))^{2}}{2m}+e\Phi(\v r)\right).\label{eq:Lag1-1}
\end{equation}
Using non-canonical variables $\v v=\frac{\v p}{m}-\frac{e}{mc}\v A$
instead of the momenta $\v p$ the particle phase-space Lagrangian
is
\begin{equation}
L(\v r,\dot{\v r},\v v,\cancel{\dot{\v v}})=\left(m\v v+\frac{e}{c}\v A(\v r)\right)\cdot\dot{\v r}-\left(\frac{m\v v^{2}}{2}+e\Phi(\v r)\right).\label{eq:Lag1-1-1}
\end{equation}
In terms of $\v z=(\v R,\phi,v_{\parallel},v_{\perp})$ we have
\begin{equation}
L(\v R,\dot{\v R},\phi,\dot{\phi},v_{\parallel},v_{\perp})=\left(mv_{\parallel}\hat{\v b}(\v R)+mv_{\perp}\hat{\v n}(\v R,\phi)+\frac{e}{c}\v A(\v R+\v{\rho})\right)\cdot(\dot{\v R}+\dot{\v{\rho}})-\left(\frac{m(v_{\parallel}^{\,2}+v_{\perp}^{\,2})}{2}+e\Phi(\v R+\v{\rho})\right),\label{eq:Lag1-1-1-1}
\end{equation}
with gyration vector $\v{\rho}=\v{\rho}(\v R,\phi,v_{\perp})=\v r-\v R$
from Eq.~(\ref{eq:rz}). This Lagrangian is still exact, as are transformations
to guiding-center variables. We will now make approximations to the
Lagrangian, affecting the \emph{dynamics} of motion in guiding-center
variables. As in the previous chapter we use a Taylor expansion which
we break at the first order. For higher order computations, Lie transform
methods are better suited (see Littlejohn, 1983). Here the small parameter
compared to typical length scales is the gyroradius
\begin{equation}
\rho_{c}=|\v{\rho}|=\frac{mcv_{\perp}}{eB(\v R)}\sim\varepsilon.
\end{equation}
Furthermore we define the cyclotron frequency at the guiding-center
position
\begin{equation}
\omega_{c}=\frac{eB(\v R)}{mc}\sim\varepsilon^{-1},
\end{equation}
as a large parameter compared to typical time scales. This corresponds
to the physical conditions where the guiding-center approximation
is applicable, leaves the perpendicular velocity $v_{\perp}$ being
a quantity of interest with order $\varepsilon^{0}=1$ and thus prevents
it from being swallowed by a perturbation expansion. The parameter
$\varepsilon$ is called an ordering parameter that allows us to collect
terms of similar magnitude. Vector and scalar fields are expanded
in the following way,
\begin{align}
\v A(\v r) & =\v A(\v R)+\varepsilon\v{\rho}\cdot\nabla\v A(\v R)+\mathcal{O}(\varepsilon^{2}),\\
\Phi(\v r) & =\Phi(\v R)+\varepsilon\v{\rho}\cdot\nabla\Phi(\v R)+\mathcal{O}(\varepsilon^{2}),
\end{align}
where $\varepsilon$ acts as a marker for intermediate calculations
and is set to one only in the final result after neglecting all terms
of order $\varepsilon$ or higher. We mark each small quantity with
an $\varepsilon$ in Eq.~\ref{eq:Lag1-1-1-1} to obtain
\begin{equation}
L(\v R,\dot{\v R},\phi,\dot{\phi},v_{\parallel},v_{\perp})=\left(mv_{\parallel}\hat{\v b}(\v R)+mv_{\perp}\hat{\v n}(\v R,\phi)+\varepsilon^{-1}\frac{e}{c}\v A(\v R+\varepsilon\v{\rho})\right)\cdot(\dot{\v R}+\dot{\v{\rho}})-\left(\frac{m(v_{\parallel}^{\,2}+v_{\perp}^{\,2})}{2}+e\Phi(\v R+\varepsilon\v{\rho})\right).\label{eq:Lag1-1-1-1-1}
\end{equation}

We cannot simply write $\varepsilon$ in front of $\dot{\boldsymbol{\rho}}$
just because the magnitude of $\v{\rho}$ is small, since gyration
is fast. This becomes apparent by using the chain rule,
\begin{align}
\dot{\v{\rho}} & =\frac{\d}{\d t}\v{\rho}(\v R,\phi,v_{\perp})=\dot{\v R}\cdot\nabla\v{\rho}+\dot{\phi}\frac{\partial\v{\rho}}{\partial\phi}+\dot{v}_{\perp}\frac{\partial\v{\rho}}{\partial v_{\perp}}\nonumber \\
 & =\varepsilon\dot{\v R}\cdot\nabla\v{\rho}+\varepsilon^{-1}\dot{\phi}\varepsilon\rho_{c}\hat{\v n}+\varepsilon\frac{\dot{v}_{\perp}}{v_{\perp}}\v{\rho}=\dot{\phi}\rho_{c}\hat{\v n}+\varepsilon(\dot{\v R}\cdot\nabla\v{\rho}+\frac{\dot{v}_{\perp}}{v_{\perp}}\v{\rho}).
\end{align}
Here in the first term the fast gyration $\dot{\phi}$ and the small
$\rho_{c}$ cancel to order $\varepsilon^{0}$, and the last term
should be even smaller than order $\varepsilon$ as long as $v_{\perp}$
doesn't change too fast due to strongly inhomogeneous fields. Now
we look at the first term in~(\ref{eq:Lag1-1-1-1-1}) up to order
$\varepsilon^{0}$,

\begin{align}
\left(mv_{\parallel}\hat{\v b}(\v R)+mv_{\perp}\hat{\v n}(\v R,\phi)+\varepsilon^{-1}\frac{e}{c}\v A(\v R+\v{\rho})\right)\cdot(\dot{\v R}+\dot{\v{\rho}}) & =\left(mv_{\parallel}\hat{\v b}+mv_{\perp}\hat{\v n}+\varepsilon^{-1}\frac{e}{c}\v A+\frac{e}{c}\v{\rho}\cdot\nabla\v A\right)\cdot\dot{\v R}\nonumber \\
 & +mv_{\perp}\dot{\phi}\rho_{c}+\varepsilon^{-1}\frac{e}{c}\v A\cdot\dot{\v{\rho}}+\frac{e}{c}(\v{\rho}\cdot\nabla\v A)\cdot\dot{\v{\rho}}+\mathcal{O}(\varepsilon).
\end{align}
Here the term with $\hat{\v b}\cdot\hat{\v n}$ vanished and we took
a scalar product $\hat{\v n}\cdot\hat{\v n}=1$ to obtain the first
term in the second line. Now we use the fact that we can add/subtract
an arbitrary total time derivative to the Lagrangian without changing
its dynamics. We will use this trick two times in a row.

First we take
\begin{align}
\frac{\d}{\d t}(\v{\rho}\cdot\v A(\v R)) & =\dot{\v{\rho}}\cdot\v A+\v{\rho}\cdot\dot{\v A}\nonumber \\
 & =\dot{\v{\rho}}\cdot\v A+\v{\rho}\cdot(\dot{\v R}\cdot\nabla\v A).\label{eq:dt}
\end{align}
This we can combine with the term $(\v{\rho}\cdot\nabla\v A)\cdot\dot{\v R}=\v{\rho}\cdot((\nabla\v A)\cdot\dot{\v R})$
via tensor algebra rule (\ref{eq:eq3}),

\begin{equation}
(\nabla\v A)\cdot\dot{\v R}-\dot{\v R}\cdot\nabla\v A=\dot{\v R}\times(\nabla\times\v A)=\dot{\v R}\times\boldsymbol{B}.\label{eq:eq3-1}
\end{equation}
So 
\begin{align}
(\v{\rho}\cdot\nabla\v A)\cdot\dot{\v R}-\v{\rho}\cdot(\dot{\v R}\cdot\nabla\v A) & =\v{\rho}\cdot(\dot{\v R}\times\v B).
\end{align}
and together with~(\ref{eq:dt}) we obtain
\begin{equation}
(\v{\rho}\cdot\nabla\v A)\cdot\dot{\v R}=\v{\rho}\cdot(\dot{\v R}\times\v B+\dot{\v R}\cdot\nabla\v A)=\v{\rho}\cdot(\dot{\v R}\times\v B)+\frac{\d}{\d t}(\v{\rho}\cdot\v A)-\dot{\v{\rho}}\cdot\v A.
\end{equation}
Inserting everything into $L$ cancels the term $\v A\cdot\dot{\v{\rho}}$
and we obtain

\begin{align}
L-\frac{e}{c}\frac{\d}{\d t}(\v{\rho}\cdot\v A)+\mathcal{O}(\varepsilon) & =\left(mv_{\parallel}\hat{\v b}+mv_{\perp}\hat{\v n}+\varepsilon^{-1}\frac{e}{c}\v A\right)\cdot\dot{\v R}\nonumber \\
 & +\frac{e}{c}\v{\rho}\cdot(\dot{\v R}\times\v B)+mv_{\perp}\dot{\phi}\rho_{c}+\frac{e}{c}(\v{\rho}\cdot\nabla\v A)\cdot\dot{\v{\rho}}\nonumber \\
 & -\left(\frac{m(v_{\parallel}^{\,2}+v_{\perp}^{\,2})}{2}+e\Phi\right),
\end{align}
where $\v A,\v B,$ and $\Phi$ are now evaluated in the guiding-center
position $\v R$ instead of the particle position $\v r$. Two terms
cancel since by rotating the triple-product and extracting basis vectors,
\begin{equation}
\frac{e}{c}\v{\rho}\cdot(\dot{\v R}\times\v B)=\frac{e}{c}\rho_{c}B\dot{\boldsymbol{R}}\cdot(\hat{\v b}\times\hat{\v{\rho}})=-mv_{\perp}\dot{\boldsymbol{R}}\cdot\hat{\v n}.
\end{equation}

To eliminate the term $(\v{\rho}\cdot\nabla\v A)\cdot\dot{\v{\rho}}=\v{\rho}\cdot\nabla\v A\cdot\dot{\v{\rho}}$
we use another total time derivative (see also review paper of Cary
and Brizard, 2009),
\begin{align}
\frac{1}{2}\frac{\d}{\d t}\left(\varepsilon\v{\rho}\cdot\varepsilon^{-1}\frac{e}{c}\nabla\v A\cdot\varepsilon\v{\rho}\right) & =\frac{1}{2}\left(\dot{\v{\rho}}\cdot\nabla\v A\cdot\v{\rho}+\v{\rho}\cdot\nabla\v A\cdot\dot{\v{\rho}}\right)\nonumber \\
 & +\frac{\varepsilon}{2}\v{\rho}\cdot\left(\frac{\d\nabla\v A}{\d t}\right)\cdot\v{\rho}.
\end{align}
Despite being a total time derivative one could also argue that this
term is of order $\varepsilon$ for the leading order approximation.
Since it contains terms of higher order due to the time derivative
and fast gyration one should be careful with this. Anyway, we can
use the expression for the term
\begin{align}
(\v{\rho}\times\dot{\v{\rho}})\cdot\v B & =(\v{\rho}\times\dot{\v{\rho}})\cdot(\nabla\times\v A)\nonumber \\
 & =(\varepsilon_{ijk}\rho_{j}\dot{\rho}_{k})(\varepsilon_{imn}\partial_{m}A_{n})\nonumber \\
 & =(\delta_{jm}\delta_{kn}-\delta_{jn}\delta_{km})(\rho_{j}\dot{\rho}_{k}\partial_{m}A_{n})\nonumber \\
 & =\v{\rho}\cdot\nabla\v A\cdot\dot{\v{\rho}}-\dot{\v{\rho}}\cdot\nabla\v A\cdot\v{\rho}.
\end{align}
Thus,
\begin{align}
\v{\rho}\cdot\nabla\v A\cdot\dot{\v{\rho}} & =\v{\rho}\cdot\nabla\v A\cdot\dot{\v{\rho}}=-\dot{\v{\rho}}\cdot\nabla\v A\cdot\v{\rho}\nonumber \\
 & =\frac{1}{2}\underbrace{\left(\v{\rho}\cdot\nabla\v A\cdot\dot{\v{\rho}}-\dot{\v{\rho}}\cdot\nabla\v A\cdot\v{\rho}\right)}_{(\v{\rho}\times\dot{\v{\rho}})\cdot\v B}+\frac{1}{2}\underbrace{\left(\dot{\v{\rho}}\cdot\nabla\v A\cdot\v{\rho}+\v{\rho}\cdot\nabla\v A\cdot\dot{\v{\rho}}\right)}_{\mathcal{O}(\varepsilon)-\frac{1}{2}\frac{\d}{\d t}\dots}.
\end{align}
Further,
\begin{equation}
(\v{\rho}\times\dot{\v{\rho}})\cdot\v B=\left(\rho_{c}\hat{\v{\rho}}\times\dot{\phi}\rho_{c}\hat{\v n}\right)\cdot\boldsymbol{B}=-\dot{\phi}\rho_{c}^{\,2}B=-\frac{mcv_{\perp}}{e}\rho_{c}\dot{\phi}.
\end{equation}
and we can replace $(\v{\rho}\cdot\nabla\v A)\cdot\dot{\v{\rho}}$
in the Lagrangian to obtain to order $\varepsilon$ (formally setting
$\varepsilon=1$) 
\begin{align}
L-\frac{e}{c}\frac{\d}{\d t}\left(\v{\rho}\cdot\v A-\frac{1}{2}\v{\rho}\cdot\nabla\v A\cdot\v{\rho}\right)+\mathcal{O}(\varepsilon) & =\left(mv_{\parallel}\hat{\v b}+\frac{e}{c}\v A\right)\cdot\dot{\v R}\nonumber \\
 & +mv_{\perp}\dot{\phi}\rho_{c}-\frac{mv_{\perp}}{2}\rho_{c}\dot{\phi}\nonumber \\
 & -\left(\frac{m(v_{\parallel}^{\,2}+v_{\perp}^{\,2})}{2}+e\Phi\right).
\end{align}
In the bracket under the total time derivative we can see that there
is some system to the way the gauge transform in phase-space is done.
More generally the technique is known as the \textbf{Lie-transform
method}. To leading order in $\varepsilon$ the \textbf{guiding-center
Lagrangian} follows as
\[
L_{\mathrm{gc}}(\v R,\dot{\v R},\cancel{\phi},\dot{\phi},v_{\parallel},v_{\perp})=\left(mv_{\parallel}\hat{\v b}(\v R)+\frac{e}{c}\v A(\v R)\right)\cdot\dot{\v R}+\frac{m^{2}cv_{\perp}^{\,2}}{2eB(\v R)}\dot{\phi}-\left(\frac{m(v_{\parallel}^{\,2}+v_{\perp}^{\,2})}{2}+e\Phi(\v R)\right).
\]
Here all fields are evaluated in $\v R$ and dependencies on $\phi$
have vanished. Thus we can identify a conserved generalized momentum
\begin{equation}
\frac{\partial L_{\mathrm{gc}}}{\partial\dot{\phi}}=\frac{m^{2}cv_{\perp}^{\,2}}{2eB(\v R)}\equiv J_{\perp}=\frac{mc}{e}\mu,
\end{equation}
known as the perpendicular invariant with $\dot{J}_{\perp}=0$ along
the trajectory. The perpendicular invariant is proportional to the
magnetic moment $\mu$ of the guiding-center orbit, which is thus
also conserved. Again, $\mu$ is not the exact magnetic moment of
the particle, but rather an approximation up to order $\varepsilon$.
If we replace $v_{\perp}$ by the variable $\mu$ we can write
\begin{equation}
L_{\mathrm{gc}}(\v R,\dot{\v R},\dot{\phi},v_{\parallel},\mu)=\left(mv_{\parallel}\hat{\v b}(\v R)+\frac{e}{c}\v A(\v R)\right)\cdot\dot{\v R}-\frac{mc}{e}\mu\dot{\phi}-\left(\frac{mv_{\parallel}^{\,2}}{2}+\mu B(\v R)+e\Phi(\v R)\right).
\end{equation}
The Hamiltonian is the last part in brackets,
\begin{equation}
H=\frac{mv_{\parallel}^{\,2}}{2}+\mu B(\v R)+e\Phi(\v R),
\end{equation}
and is equal to the total energy. For equations of motion, it is of
limited use, since we use non-canonical variables. We rather need
Euler-Lagrange equations of the guiding-center Lagrangian that follow
as
\begin{align}
\dot{\v R} & =\frac{v_{\parallel}\v B^{\star}(\v R,v_{\parallel})+c\v E^{\star}(\v R,\mu)\times\hat{\v b}(\v R)}{\hat{\v b}(\v R)\cdot\v B^{\star}(\v R,v_{\parallel})},\quad\dot{v}_{\parallel}=\frac{e}{m}\frac{\v E^{\star}(\v R,\mu)\cdot\v B^{\star}(\v R,v_{\parallel})}{\hat{\v b}(\v R)\cdot\v B^{\star}(\v R,v_{\parallel})},\label{eq:rdot-1}\\
\dot{\phi} & =-\omega_{c}(\v R)\equiv\frac{e\v B(\v R)}{mc},\quad\dot{\mu}=0.\label{eq:phidot}
\end{align}
where $\hat{\v b}(\v R)=\v B(\v R)/B(\v R)$ and
\begin{align}
\v E^{\star}(\v R,\mu) & =-\frac{1}{e}\nabla H(\v R,v_{\parallel},\mu)=-\frac{\mu}{e}\nabla B(\v R)-\nabla\Phi(\v R),\\
\v B^{\star}(\v R,v_{\parallel}) & =\nabla\times\v A^{\star}(\v R,v_{\parallel}),\\
\v A^{\star}(\v R,v_{\parallel}) & =\v A(\v R)+\frac{mc}{e}v_{\parallel}\hat{\v b}(\v R).
\end{align}

\textbf{Derivation of equations of motion}

From now on we will call the guiding-center Lagrangian $L_{\mathrm{gc}}$
just $L$ to reduce notational clutter. Take
\begin{equation}
L(\v R,v_{\parallel},\mu,\dot{\v R},\dot{\phi})=\left(mv_{\parallel}\hat{\v b}(\v R)+\frac{e}{c}\v A(\v R)\right)\cdot\dot{\v R}-\frac{mc}{e}\mu\dot{\phi}-\left(\frac{mv_{\parallel}^{\,2}}{2}+\mu B(\v R)+e\Phi(\v R)\right).
\end{equation}

Derivatives are

\begin{align}
\frac{\partial L}{\partial\v R} & =mv_{\parallel}(\nabla\hat{\v b})\cdot\dot{\v R}+\frac{e}{c}(\nabla\v A)\cdot\dot{\v R}-\left(\mu\nabla B+e\nabla\Phi\right)\\
\frac{\partial L}{\partial\phi} & =0\\
\frac{\partial L}{\partial v_{\parallel}} & =m\hat{\v b}\cdot\dot{\v R}-mv_{\parallel}\label{eq:vpar}\\
\frac{\partial L}{\partial\mu} & =-\frac{mc}{e}\dot{\phi}-B\label{eq:mu}
\end{align}
Generalized velocities:
\begin{align}
\frac{\d}{\d t}\frac{\partial L}{\partial\dot{\v R}} & =\frac{\d}{\d t}(mv_{\parallel}\hat{\v b}(\v R(t))+\frac{e}{c}\v A(\v R(t)))=m\dot{v}_{\parallel}\hat{\v b}+mv_{\parallel}\dot{\v R}\cdot\nabla\hat{\v b}+\frac{e}{c}\dot{\v R}\cdot\nabla\v A.\\
\frac{\d}{\d t}\frac{\partial L}{\partial\dot{\phi}} & =\frac{\d}{\d t}(-\frac{mc}{e}\mu)=-\frac{mc}{e}\dot{\mu}
\end{align}
Euler-Lagrange euqations in $\v R$ are
\begin{align}
0=\frac{\d}{\d t}\frac{\partial L}{\partial\dot{\v R}}-\frac{\partial L}{\partial\v R} & =m\dot{v}_{\parallel}\hat{\v b}+mv_{\parallel}(\nabla\hat{\v b})\cdot\dot{\v R}+\frac{e}{c}(\nabla\v A)\cdot\dot{\v R}\nonumber \\
 & -\mu\nabla B-e\nabla\Phi-mv_{\parallel}\dot{\v R}\cdot\nabla\hat{\v b}-\frac{e}{c}\dot{\v R}\cdot\nabla\v A\nonumber \\
 & =m\dot{v}_{\parallel}\hat{\v b}+mv_{\parallel}\dot{\v R}\times(\nabla\times\hat{\v b})+\frac{e}{c}\dot{\v R}\times(\nabla\times\v A)\nonumber \\
 & -\mu\nabla B+e\v E-mv_{\parallel}\dot{\v R}\cdot\nabla\hat{\v b}\nonumber \\
 & =m\dot{v}_{\parallel}\hat{\v b}+\frac{e}{c}\dot{\v R}\times(\nabla\times\v A^{\star})-\mu\nabla B+e\v E-mv_{\parallel}\dot{\v R}\cdot\nabla\hat{\v b}\nonumber \\
 & =m\dot{v}_{\parallel}\hat{\v b}+\frac{e}{c}\dot{\v R}\times\v B^{\star}-\mu\nabla B+e\v E-mv_{\parallel}\dot{\v R}\cdot\nabla\hat{\v b}.
\end{align}
Here we have used $\v a\times(\nabla\times\v b)=(\nabla\v a)\cdot\v b-\v b\cdot\nabla\v a$
from (\ref{eq:eq3}) and defined
\begin{align}
\v A^{\star}(\text{\ensuremath{\v R}},v_{\parallel}) & =\v A(\text{\ensuremath{\v R}})+\frac{mc}{e}v_{\parallel}\hat{\v b}(\text{\ensuremath{\v R}}),\\
\v B^{\star}(\text{\ensuremath{\v R}},v_{\parallel}) & =\nabla\times\v A^{\star}(\text{\ensuremath{\v R}},v_{\parallel})
\end{align}
as a modified vector potential and magnetic field. The latter depend
on the parallel velocity $v_{\parallel}$ that can take both, positive
or negative sign and vary during the orbit. The Euler-Lagrange equation
in the ignorable $\phi$ yield conservation of magnetic moment $\dot{\mu}=0$.
The remaining equations have no generalized velocity terms in $L_{\text{gc}}$
being a phase-space Lagrangian. Thus we can set terms (\ref{eq:vpar}-\ref{eq:mu})
to zero right away. To sum up our equations of motion become
\begin{align}
m\dot{v}_{\parallel}\hat{\v b}(\v R)+\frac{e}{c}\dot{\v R}\times\v B^{\star}(\text{\ensuremath{\v R}},v_{\parallel}) & =\mu\nabla B(\v R)-e\v E(\v R),\label{eq:elg1}\\
\hat{\v b}(\v R)\cdot\dot{\v R} & =v_{\parallel},\label{eq:elg2}\\
\dot{\phi} & =-\frac{eB(\v R)}{mc}\equiv-\omega_{c}(\v R).\label{eq:elg3}
\end{align}
Eq.~(\ref{eq:elg1}) describes drifts due to $\v E$ and gradient
and curvature of $\v B$. Eq.~(\ref{eq:elg2}) confirms the meaning
of $v_{\parallel}$ as the parallel velocity, and Eq.~(\ref{eq:elg3})
of $\phi$ as the gyrophase with regard to the guiding-center position
$\v R$. Now we would like to decouple Eqs.~(\ref{eq:elg1}-\ref{eq:elg2})
to get explicit expressions for $\dot{\v R}$ and $\dot{v}_{\parallel}$.
Taking a cross product of $\hat{\v b}$ with Eq.~\ref{eq:elg1} removes
the first term and together with the BAC-CAB rule and $\v b\cdot\dot{\v R}=v_{\parallel}$
from Eq.~(\ref{eq:elg2}) yields
\begin{equation}
\frac{e}{c}((\hat{\v b}\cdot\v B^{\star})\dot{\v R}-v_{\parallel}\v B^{\star})=\hat{\v b}\times(\mu\nabla B-e\v E).
\end{equation}
Finally we obtain
\begin{align}
\dot{\v R} & =\frac{v_{\parallel}\v B^{\star}+c\v E^{\star}\times\hat{\v b}}{\hat{\v b}\cdot\v B^{\star}},\label{eq:rdot}\\
\dot{v}_{\parallel} & =\frac{e}{m}\frac{\v B^{\star}\cdot\v E^{\star}}{\hat{\v b}\cdot\v B^{\star}}.
\end{align}
Here (convention taken from Lanthaler et al, PPCF 2017 vol: 59 (4)
pp: 044014) we defined
\begin{equation}
\v E^{\star}=-e^{-1}\nabla H=-\frac{\mu}{e}\nabla B-\nabla\Phi.
\end{equation}


\section{Plasma Models}

The goal of this chapter is to show a clear path from single particle
motion via a kinetic intermediate stage. Details on the kinetic part,
including an alternative derivation via Liouville's theorem leading
to a well-defined collision term are presented in the course \emph{Plasma
Kinetic Theory} of Winfried Kernbichler. More extensive descriptions
of the derivations made here can be found at \url{http://homepages.cae.wisc.edu/~callen/book.html}
as well as ``Introduction to Plasma Theory'' by D.D. Nicholson,
``Fundamentals of Plasma Physics'' by P.M. Bellan and ``Ideal MHD''
by J.P. Freidberg. 

\subsection{From Newton's law to the Klimontovich equation}

We consider an ensemble of $N$ particles of this species and introduce
a kinetic picture: The normalized number density of discrete particles
at position $\boldsymbol{z}_{k}(t)$ in phase-space is given by placing
Dirac $\delta$s wherever a particle lies at time $t$,
\begin{equation}
f^{{\rm m}}(\text{\ensuremath{\v z}},t)=\frac{1}{N}\sum_{k=1}^{N}\delta(\boldsymbol{z}-\boldsymbol{z}_{k}(t)).\label{eq:fm}
\end{equation}
The superscript ${\rm m}$ stands for microscopic here as this \textbf{microscropic
distribution function} $f^{{\rm m}}$ contains all details about single
particle states (classically positions and velocities/momenta). Roughly
speaking $f^{\mathrm{m}}=0$ where there is no particle and $\infty$
where there is a particle. It is important to distinguish the position
in phase-space $z$ as an independent variable for $f^{\mathrm{m}}$
from the orbit curves $\v z_{k}(t)$ of the individual particles,
that introduce the time-dependency in $f^{\mathrm{m}}$. 

It is normalized, In particular the choice of the normalization by
$1/N$ together with the properties of the Dirac $\delta$ leads to
the \textbf{normalization property}
\begin{equation}
\int_{\Pi}\d^{6}z\,f^{{\rm m}}(\v z,t)=1.\label{eq:norm1}
\end{equation}

Here $\Pi$ is the domain in phase-space where particles exist. If
we perform an integral over a region $\Pi_{k}\subset\Pi$ which contains
only one particle $k$ we obtain instead
\begin{equation}
\int_{\Pi_{k}}\d^{6}z\,f^{{\rm m}}(\boldsymbol{r},\boldsymbol{v},t)=\frac{1}{N}.\label{eq:norm1-1}
\end{equation}
This can be interpreted as the probability to find a predefined particle
amongst our set of $N$ particles near phase-space location $k$.

To evaluate the partial time derivative of $f^{\mathrm{m}}$ we use
the chain rule,
\begin{equation}
\frac{\partial}{\partial t}f^{{\rm m}}(\text{\ensuremath{\v z}},t)=-\frac{1}{N}\sum_{k=1}^{N}\dot{\boldsymbol{z}}_{k}(t)\cdot\nabla_{\v z}\delta(\boldsymbol{z}-\boldsymbol{z}_{k}(t)).\label{eq:fm-1}
\end{equation}
From the equations of motion we know relations between phase-space
velocities $\dot{\v z}_{k}$ and phase-space positions $\v z_{k}$
of the form $\dot{\boldsymbol{z}}_{k}(t)=\v w(\v z_{k}(t),t)$. In
particular charged plasma particles are subject to electromagnetic
forces excerted by electric field $\boldsymbol{E}$ and magnetic field
$\boldsymbol{B}$. Here we use real-space position $\v r$ and velocity
$\v v$ to parameterize phase-space via $\v z=(\v r,\v v)$ and apply
the usual Lorentz force law. For developing other variants like gyrokinetics
one could start with the guiding-center Lagrangian formalism. 

\subsubsection*{Non-interacting particles in external fields}

We first start with an idealized set of \emph{non-interacting}\textbf{
}particles in \emph{external} fields $\v E$ and $\v B$. The motion
of a single particle $k$ of mass $m$ and charge $e$ using CGS units
is given by 
\begin{align}
\dot{\boldsymbol{r}}_{k}(t) & =\boldsymbol{v}_{k}(t),\label{eq:ma}\\
\dot{\boldsymbol{v}}_{k}(t) & =\frac{\boldsymbol{F}_{k}(\boldsymbol{r}_{k}(t),\boldsymbol{v}_{k}(t),t)}{m}=\frac{e}{m}(\boldsymbol{E}(\boldsymbol{r}_{k}(t),t)+\frac{1}{c}\boldsymbol{v}_{k}(t)\times\boldsymbol{B}(\boldsymbol{r}_{k}(t),t)).
\end{align}
This is of the required form $\dot{\boldsymbol{z}}_{k}(t)=\v w(\v z_{k}(t),t)$
with
\begin{equation}
\dot{\boldsymbol{z}}_{k}(t)\equiv(\dot{\boldsymbol{r}}_{k}(t),\dot{\boldsymbol{v}}_{k}(t)),\qquad\v w(\v z_{k}(t),t)=(\boldsymbol{v}_{k}(t),\frac{\boldsymbol{F}_{k}(\boldsymbol{r}_{k}(t),t)}{m}).\label{eq:newton}
\end{equation}
Now we can replace $\dot{\boldsymbol{z}}_{k}(t)$ by $\v w(\v z_{k}(t),t)$
and use the rule $g(\v z_{0})\delta(\v z-\v z_{0})=g(\v z)\delta(\v z-\v z_{0})$
for the $\delta$. Then Eq.~(\ref{eq:fm-1}) changes to
\begin{align}
\frac{\partial}{\partial t}f^{{\rm m}}(\text{\ensuremath{\v z}},t) & =-\frac{1}{N}\sum_{k=1}^{N}\v w(\v z_{k}(t),t)\cdot\nabla_{\v z}\delta(\boldsymbol{z}-\boldsymbol{z}_{k}(t))\nonumber \\
 & =-\frac{1}{N}\sum_{k=1}^{N}\v w(\v z,t)\cdot\nabla_{\v z}\delta(\boldsymbol{z}-\boldsymbol{z}_{k}(t))\nonumber \\
 & =-\v w(\v z,t)\cdot\nabla_{\v z}f^{{\rm m}}(\text{\ensuremath{\v z}},t).\label{eq:kineq}
\end{align}
This is a convective derivative in phase-space where $f^{\mathrm{m}}$
is transported along the flow $\v w$ produced by the movement of
particles via equations of motion evaluated in $\v z$. In particular
for $\v z=(\v r,\v v)$ and Newton's equations of motion with force
$\v F$ as in Eq.~(\ref{eq:newton}) we obtain the \textbf{kinetic
equation for non-interacting particles}
\begin{equation}
\frac{\partial}{\partial t}f^{{\rm m}}(\text{\ensuremath{\v r}},\v v,t)+\v v\cdot\nabla_{\v r}f^{{\rm m}}(\text{\ensuremath{\v r}},\v v,t)+\frac{\v F(\v r,\v v,t)}{m}\cdot\nabla_{\v v}f^{{\rm m}}(\text{\ensuremath{\v r}},\v v,t)=0.\label{eq:kin1}
\end{equation}
While one might say this is not a big surprise, we have achieved the
complete elimination of individual particle orbits with dependent
variables $\v z_{k}(t)=(\v r_{k}(t),\v v_{k}(t))$ given by $2N$
\emph{ordinary }differential equations of motion by a single \emph{partial}
differential equation in the microscopic distribution function $f^{\mathrm{m}}$
with independent variables $\v z=(\v r,\v v)$ and $t$. They are
connected via the fact that orbits are the \emph{characteristics}
of Eq.~(\ref{eq:kin1}) along which $f^{\mathrm{m}}$ propagates.
By smoothing over a regions of phase-space with many particles we
can now also find solutions of (\ref{eq:kin1}) with a distribution
function that is less ``spiky'' than the original $f^{\mathrm{m}}\propto\delta$.
This will be our goal in the following sections. 

\subsubsection*{Interacting particles}

Up to now we have made our life easy by ignoring particle interactions.
Before proceeding we have to resolve the issue that in reality the
force $\v F_{k}$ acting on particle $k$ doesn't depend on $\v r_{k}$
and $\v v_{k}$ alone, but also on other particles' positions and
velocities (currents generating $\v B$ fields). In a plasma the motion
of particle number $k$ is affected the microscopic fields $\boldsymbol{E}^{{\rm m}}(\boldsymbol{r}_{k},t;\{\boldsymbol{r}_{l}\},\{\boldsymbol{v}_{l}\})$,
$\boldsymbol{B}^{m}(\boldsymbol{r}_{k},t;\{\boldsymbol{r}_{l}\},\{\boldsymbol{v}_{l}\})$
that are generated by\footnote{or linked to, to be more precise}
charges and currents dependent on position and velocity of all the
other particles $l\neq k$ (denoted by curly brackets here). Also
(possibly explicitly time-dependent) external fields excerted on the
plasma are included in this full description. We rewrite Eq. \ref{eq:ma}
to 
\begin{align}
\dot{\boldsymbol{v}}_{k} & =\boldsymbol{F}^{{\rm m}}(\boldsymbol{r}_{k},\boldsymbol{v}_{k},t;\{\boldsymbol{r}_{l}\},\{\boldsymbol{v}_{l}\})/m\nonumber \\
 & =\frac{e}{m}\boldsymbol{E}^{{\rm m}}(\boldsymbol{r}_{k},t;\{\boldsymbol{r}_{l}\},\{\boldsymbol{v}_{l}\})+\frac{e}{mc}\boldsymbol{v}_{k}\times\boldsymbol{B}^{{\rm m}}(\boldsymbol{r}_{k},t;\{\boldsymbol{r}_{l}\},\{\boldsymbol{v}_{l}\}).
\end{align}
Following the same steps as in the non-interacting case we arrive
at the \textbf{kinetic equation for interacting particles},
\begin{equation}
\frac{\partial}{\partial t}f^{{\rm m}}(\text{\ensuremath{\v r}},\v v,t)+\v v\cdot\nabla_{\v r}f^{{\rm m}}(\text{\ensuremath{\v r}},\v v,t)+\frac{\v F^{\mathrm{m}}(\v r,\v v,t;\{\boldsymbol{r}_{l}(t)\},\{\boldsymbol{v}_{l}(t)\})}{m}\cdot\nabla_{\v v}f^{{\rm m}}(\text{\ensuremath{\v r}},\v v,t)=0.\label{eq:kin1-1}
\end{equation}
While formally similar to the non-interacting result \ref{eq:kin1},
the interacting variant here contains all orbits $\{\boldsymbol{r}_{l}(t)\},\{\boldsymbol{v}_{l}(t)\}$
resulting in a complicated time-dependency of microscopic forces $\v F^{\mathrm{m}}$.
This dependency cannot even be stated without knowing the solutions
of orbits in the first place, thereby eliminating any practical value
of the kinetic reformulation. For electromagnetic fields Eq.~(\ref{eq:kin1-1})
is known as the \textbf{Klimontovich equation}, 
\begin{align}
\frac{\partial f^{{\rm m}}}{\partial t} & +\boldsymbol{v}\cdot\nabla_{\boldsymbol{r}}f^{{\rm m}}+\frac{q}{m}(\boldsymbol{E}^{{\rm {\rm m}}}+\frac{1}{c}\boldsymbol{v}\times\boldsymbol{B}^{{\rm m}})\cdot\nabla_{\boldsymbol{v}}f^{{\rm m}}=0,\label{eq:Klimontovich}
\end{align}
Here $\v E^{\mathrm{m}}$ and $\v B^{\mathrm{m}}$ could in principle
be found from either solving Maxwell's equations self-consistently
together with (\ref{eq:Klimontovich}), or finding the Li�nard--Wiechert
potentials for all particles ($N\approx10^{19}$ in laboratory plasmas).
While this is of maybe of philosophical value, we are usually not
interested in every single particle, but rather in their average statistical
behaviour.

\subsubsection*{Smoothing the distribution function}

In order to make equation (\ref{eq:Klimontovich}) manageable one
has to rely on some averaging procedure. Since we can only observe
macroscopic quantities such as temperature, density and pressure,
we average the $f^{{\rm m}}$ on an intermediate scale by
\begin{align}
\left\langle f^{m}\right\rangle  & :=\frac{\int_{V_{x}}d^{3}x^{\prime}\int_{V_{v}}d^{3}v^{\prime}\,f^{{\rm m}}(\boldsymbol{r}^{\prime},\boldsymbol{v}^{\prime})}{\int_{V_{x}}d^{3}x^{\prime}\int_{V_{v}}d^{3}v^{\prime}}=\frac{\int_{V_{x}}d^{3}x^{\prime}\int_{V_{v}}d^{3}v^{\prime}\,f^{{\rm m}}(\boldsymbol{r}^{\prime},\boldsymbol{v}^{\prime})}{\Delta V_{\mathrm{p}}}.\label{eq:average}
\end{align}
The phase space volume $\Delta V_{\mathrm{p}}$ over which the average
is taken is chosen much larger than the distance between individual
particles and much smaller than a scale such as the Debye length in
a plasma. This means that the macroscopic distribution function $f=\left\langle f^{\mathrm{m}}\right\rangle $
and other quantities can be treated as continuous functions with respect
to even larger scales such as the whole plasma dimensions. The macroscopic
distribution function retains the normalization property of Eq.~(\ref{eq:norm1})
when integrating over phase-space.

We split the microscopic distribution function $f^{{\rm m}}$ as well
as the fields into the average part and the fluctuations around the
average,
\begin{equation}
f^{{\rm m}}=f+\delta f,\,\,\,\,\,\,\boldsymbol{E}^{{\rm m}}=\boldsymbol{E}+\delta\boldsymbol{E},\,\,\,\,\,\,\boldsymbol{B}^{{\rm m}}=\boldsymbol{B}+\delta\boldsymbol{B}.
\end{equation}
Most importantly, $\boldsymbol{E}$ and $\boldsymbol{B}$ do not depend
on the single particle positions anymore, but can still depend on
local densities of charges $\rho_{e}(\v r,t)$ and currents $\v j(\v r,t)$
on the averaging scale. Applying the average (\ref{eq:average}) to
Eq. (\ref{eq:Klimontovich}) we obtain the \textbf{plasma kinetic
equation}
\begin{align}
\frac{\partial f}{\partial t}+\boldsymbol{v}\cdot\nabla_{\boldsymbol{r}}f+\frac{e}{m}(\boldsymbol{E}+\frac{1}{c}\boldsymbol{v}\times\boldsymbol{B})\cdot\nabla_{\boldsymbol{v}}f & =-\left\langle \frac{e}{m}(\delta\boldsymbol{E}+\frac{1}{c}\boldsymbol{v}\times\delta\boldsymbol{B})\cdot\nabla_{\boldsymbol{v}}\delta f\right\rangle \equiv\left(\frac{\partial f}{\partial t}\right)_{\mathrm{c}}.\label{eq:Klimontovich-1}
\end{align}
Here we have used the fact that averages containing a single fluctuation
term such as $\left\langle \delta f\right\rangle $ are zero. The
remaining term containing the products of two of such terms has been
moved to the right side. All the influences from microscopic fluctuations
on the time evolution of the distribution function $f$ are hidden
there. Because interactions between point-like particles apart from
their collective behavior can be viewed as collisions it is called
the collision term and labelled $(\partial f/\partial t)_{\mathrm{c}}$.
Methods to treat this collision term by different approximations are
subject to plasma kinetic theory. A consistent derivation is possible
via the Lenard-Balescu equation, where cumulative small-angle collisions
allow to write the collision term as a Fokker-Planck-type differential
operator acting on $f$ (Landau collision term). For details, see
book ``Introduction to Plasma Theory'' by Nicholson, or ``Classical
transport theory'' by Balescu.

\subsection{From the plasma kinetic equation to magnetohydrodynamic (fluid) equations}

With the distribution function $f$ defined, it is still difficult
to observe it directly in experiment. Our goal is to get rid of velocity
space dependencies in order to define macroscopic (fluid) observables.
This is commonly done for three reasons. Firstly, we can observe distributions
of quantities over position space $\v r$ more easily and inuitively
than over velocity space $\v v$. Secondly, in a thermodynamic equilibrium
the velocity space distribution is fixed by a Maxwellian. Finally,
macroscopic conservation laws in position space follow automatically
from conservation laws in phase-space and can be of interest on their
own, no matter what the distribution in $\v v$ is.

\subsubsection*{Conservation laws in phase-space}

In order to find conservation laws, we first need to transform the
plasma kinetic equation (\ref{eq:Klimontovich-1}) from its convective
form into a conservative form. This means pulling the gradients of
$f$ to the front so we get the divergence of a phase-flux. For the
second term this is easy, since $\v v$ doesn't depend on $\v r$
as both are independent variables,
\begin{equation}
\boldsymbol{v}\cdot\nabla_{\boldsymbol{r}}f=\nabla_{\v r}\cdot(\v vf)-(\nabla_{\v r}\cdot\v v)f=\nabla_{\v r}\cdot(\v vf).
\end{equation}
For the second term it works too, since $\v E$ doesn't depend on
$\v v$, and
\begin{align}
\nabla_{\v v}\cdot(\v v\times\v B) & =\v B\cdot(\nabla_{\v v}\times\v v)-\v v\cdot(\nabla_{\v v}\times\v B).
\end{align}
Taking a curl of $\v v=(v_{x},v_{y},v_{z})$ in velocity space yields
zero, and $\v B$ doesn't even depend on $\v v$. Thus,
\begin{equation}
\frac{e}{m}(\boldsymbol{E}+\frac{1}{c}\boldsymbol{v}\times\boldsymbol{B})\cdot\nabla_{\boldsymbol{v}}f=\nabla_{\boldsymbol{v}}\cdot\left(\frac{e}{m}(\boldsymbol{E}+\frac{1}{c}\boldsymbol{v}\times\boldsymbol{B})f\right).
\end{equation}
Thus we can write Eq.~(\ref{eq:Klimontovich-1}) in conservative
form
\begin{equation}
\frac{\partial f}{\partial t}+\nabla_{\boldsymbol{r}}\cdot(\boldsymbol{v}f)+\nabla_{\boldsymbol{v}}\cdot\left(\frac{e}{m}(\boldsymbol{E}+\frac{1}{c}\boldsymbol{v}\times\boldsymbol{B})f\right)=\left(\frac{\partial f}{\partial t}\right)_{\mathrm{c}}.\label{eq:falpha-2}
\end{equation}
This result is only a special case of a more general principle. Let's
consider the convective (Lagrangian) form of any kinetic equation
in phase space with variables $\v z$,
\begin{equation}
\frac{Df}{Dt}=\frac{\partial f(\v z,t)}{\partial t}+\boldsymbol{w}(\v z,t)\cdot\nabla_{\boldsymbol{z}}f(\v z,t)=\left(\frac{\partial f(\v z,t)}{\partial t}\right)_{\mathrm{c}}.\label{eq:kincon}
\end{equation}
Again it is important to remember that $\v w$ and $\dot{\v z}(t)$
are related but different objects. As mentioned before, $\v w(\v z,t)$
are characeristics of Eq.~(\ref{eq:kincon}), related to orbits via
$\dot{\v z}_{\alpha}(t)=\v w_{\alpha}(\v z_{\alpha}(t),t)$. If $\v w$
follows from Hamiltonian equations of motion (including phase-space/guiding-center
Lagrangian), it is called a \textbf{Hamiltonian vector field} and
always has the property of divergence-freeness
\begin{equation}
\nabla_{\v z}\cdot\v w(\v z,t)=0,
\end{equation}
known as \textbf{Liouville's theorem}, or the conservation of phase-space
volume. This immediately allows to write the conservative form of
Eq.~(\ref{eq:kincon}) as
\begin{equation}
\frac{\partial f(\v z,t)}{\partial t}+\nabla_{\boldsymbol{z}}\cdot(\boldsymbol{w}(\v z,t)f(\v z,t))=\left(\frac{\partial f(\v z,t)}{\partial t}\right)_{\mathrm{c}}.\label{eq:kincon-1}
\end{equation}
Eq.~(\ref{eq:kincon-1}) is completely analogous to the conservative
Eulerian picture in fluid mechanics. The change of $f$ over time
at a fixed $\v z$ comes from a divergence of the phase flux $\Gamma_{f}=\v wf$
transporting $f$ and a source terms on the right-hand side resulting
from collisions.

\subsubsection*{Moments}

In order to take influences from electrons and ions on macroscopic
observables into account, we define an individual distribution function
$f_{\alpha}$ for each plasma species $\alpha$. Macroscopic observables
can then be calculated from the distribution functions $f_{\alpha}$
of either one species $\alpha$ or a combination of them by removing
velocity dependencies via an integral over the whole velocity space.
The fluid observable $\t Q(\v r,t)$ is defined via its kinetic counterpart
$\t q(\v r,\v v,t)$ by
\begin{equation}
\t Q(\v r,t)\equiv\int d^{3}v\,\t q(\v r,\v v,t)f_{\alpha}(\boldsymbol{r},\boldsymbol{v},t).\label{eq:intfluid}
\end{equation}

where $\t q$ and $\t Q$ can be a scalar, vector, or tensor. The
usual way to proceed is to define a number density from $\t q_{0}=N_{\alpha}$,
momentum density $\t q_{1}=m_{\alpha}N_{\alpha}\v v$ and energy density
$\t q_{2}=m_{\alpha}N_{\alpha}v^{2}/2$. Here $N_{\alpha}$ is the
total number of particles of species $\alpha$ in the spatial domain
$\Omega$ defined large enough that it will always contain all particles
(or even $\Omega=\mathbb{R}^{3}$). Integrals of the kind of (\ref{eq:intfluid})
when applied to another expression $g(\v r,\v v,t)$,
\begin{equation}
\t M\equiv\int d^{3}v\,\t q(\v r,\v v,t)g(\v r,\v v,t)f_{\alpha}(\boldsymbol{r},\boldsymbol{v},t).\label{eq:intfluid-1}
\end{equation}
are called $\t q$-weighted \textbf{moments }of $g$. We speak of
a $K$-th moment, if $\v v$ appears $K$ times in the overall expression
in the integral \ref{eq:intfluid-1}. For our $\t q_{k}$ defined
above, moments will be of order $K=k$ if no $\v v$ appears inside
$g$.

In the following section we are going to take $\t q_{k}$-weighted
moments of the plasma kinetic equation (\ref{eq:Klimontovich-1})
\begin{equation}
\underset{(I)}{\frac{\partial f_{\alpha}}{\partial t}}+\underset{(II)}{\nabla_{\boldsymbol{r}}\cdot(\boldsymbol{v}f_{\alpha})}+\underset{(III)}{\nabla_{\boldsymbol{v}}\cdot\left(\frac{\v F_{\alpha}}{m_{\alpha}}f_{\alpha}\right)}=\underset{(C)}{\left(\frac{\partial f_{\alpha}}{\partial t}\right)_{\mathrm{c}}}.\label{eq:falpha-2-1}
\end{equation}
where we have labeled single terms for later reference and denoted
the Lorentz force as $\v F_{\alpha}=\v F_{\alpha}(\v r,\v v,t)$.
Already here we see that because of $\v v$ appearing in $(II)$ the
order of its moments will be of order $k+1$.

\subsubsection*{Continuity equation for number density}

We start with the definition of the \textbf{number density} $n_{\alpha}$
that should give the average\emph{ local} particle density at $\v r$
at time $t$ via Eq.~\ref{eq:intfluid} with $\t q=N_{\alpha}$
\begin{align}
n_{\alpha}(\boldsymbol{r},t) & \equiv\int d^{3}v\,N_{\alpha}f_{\alpha}(\boldsymbol{r},\boldsymbol{v},t)=N_{\alpha}\int d^{3}v\,f_{\alpha}(\boldsymbol{r},\boldsymbol{v},t).
\end{align}
This equation is a zeroeth moment. Since $f_{\alpha}$ is normalized
with 
\begin{equation}
\int_{\Omega}\d^{3}r\int\d^{3}vf_{\alpha}(\v r,\v v,t)=1,
\end{equation}
we can easily check that 
\begin{equation}
\int_{\Omega}\d^{3}r\,n_{\alpha}(\v r,t)=N_{\alpha}
\end{equation}
yields the number of particles. Mass density $\rho_{m\alpha}(\v r,t)=m_{\alpha}n_{\alpha}(\v r,t)$
and charge density $\rho_{e\alpha}(\v r,t)=e_{\alpha}n_{\alpha}(\v r,t)$
can be computed from $n_{\alpha}$. Due to quasineutrality $\sum_{\alpha}e_{\alpha}n_{\alpha}(\v r,t)=0$
we will often assume $n_{i}=n_{e}$ in a two-species plasma of hydrogen/deuterium/tritium
ions with charge $+e$ and electrons with charge $-e$. Now we start
taking moments $N_{\alpha}\int\d^{3}v$ of Eq.~\ref{eq:falpha-2-1}.
Application on $(I)$ yields the zeroth moment
\begin{equation}
N_{\alpha}\int\d^{3}v\,(I)=N_{\alpha}\int\d^{3}v\frac{\partial f_{\alpha}}{\partial t}=\frac{\partial}{\partial t}(N_{\alpha}\int\d^{3}vf_{\alpha})=\frac{\partial n_{\alpha}}{\partial t}.
\end{equation}
Here we could pull out the time derivative because integral boundaries
are at $\infty$ and independent of time,\footnote{otherwise one needs the Leibniz rule}
and $N_{\alpha}$ is a constant, provided that the plasma is hot enought
that no recombination occurs, and nuclear fusion is so inefficient
that the relative change in $N_{\alpha}$ is negligible.

Acting on $(II)$ yields
\begin{equation}
N_{\alpha}\int\d^{3}v\,(II)=N_{\alpha}\int\d^{3}v\nabla_{\boldsymbol{r}}\cdot(\boldsymbol{v}f_{\alpha})=\nabla\cdot(N_{\alpha}\int\d^{3}v\,\boldsymbol{v}f_{\alpha})\equiv\nabla\cdot\v{\Gamma}_{n\alpha}.
\end{equation}
Here we have defined the \textbf{particle flux} 
\begin{equation}
\v{\Gamma}_{n\alpha}(\v r,t)\equiv N_{\alpha}\int\d^{3}v\,\boldsymbol{v}f_{\alpha}(\v r,\v v,t)
\end{equation}
for species $\alpha$ being a \emph{first} rather than a zeroth moment.
Note also that from here on we denote $\nabla\equiv\nabla_{\v r}$
as the usual Nabla operator in position space.

Term $(III)$ can be transformed to a surface integral in velocity
space using Gauss' divergence theorem,
\begin{equation}
N_{\alpha}\int\d^{3}v\,(III)=N_{\alpha}\int\d^{3}v\nabla_{\boldsymbol{v}}\cdot\left(\frac{\v F_{\alpha}}{m_{\alpha}}f_{\alpha}\right)=N_{\alpha}\oint\,\d\v S_{\v v}\cdot\frac{\v F_{\alpha}}{m_{\alpha}}f_{\alpha}=0.
\end{equation}
Here the surface integral vanishes when taking the limit to infinite
velocities, that are never reached by any particle.

Finally, the collisional term $(C)$ becomes
\begin{equation}
N_{\alpha}\int\d^{3}v\,(C)=N_{\alpha}\int\d^{3}v\left(\frac{\partial f_{\alpha}}{\partial t}\right)_{\mathrm{c}}=0.\label{eq:collnum}
\end{equation}
This term must vanish, since collisions cannot change species, or
even create or destroy particles, but only reshuffle them in velocity
space. Every collision that removes a particle of species $\alpha$
at a certain velocity $v$, must add it at another velocity $v$.
Since there are no further velocity-dependent weights in the integral
(\ref{eq:collnum}), each of these contributions cancels each out
identically. 

With the above-made definitions and simplifications, we can write
the conservation law for the particle number density $n_{\alpha}$
as
\begin{equation}
\frac{\partial n_{\alpha}(\v r,t)}{\partial t}+\nabla\cdot\v{\Gamma}_{n\alpha}(\v r,t)=0.\label{eq:falpha-2-1-1}
\end{equation}
Re-writing this in the commonly used form with \textbf{species fluid
velocity} $\v V_{\alpha}$ and
\begin{equation}
\frac{\partial n_{\alpha}(\v r,t)}{\partial t}+\nabla\cdot(\v V_{\alpha}(\v r,t)n_{\alpha}(\v r,t))=0,\label{eq:falpha-2-1-1-1}
\end{equation}
yields the \emph{definition} of $\v V_{\alpha}$ as
\begin{equation}
\v V_{\alpha}(\v r,t)\equiv\frac{\v{\Gamma}_{n\alpha}(\v r,t)}{n_{\alpha}(\v r,t)}=\frac{\cancel{N_{\alpha}}\int\d^{3}v\,\boldsymbol{v}f_{\alpha}(\v r,\v v,t)}{\cancel{N_{\alpha}}\int d^{3}v\,f_{\alpha}(\boldsymbol{r},\boldsymbol{v},t)}.
\end{equation}
The fluid velocity is thus an average over the velocity distribution
in phase-space, given by a first moment, normalized by the local particle
density. As $\v V_{\alpha}$ is an intrinsic quantity it doesn't depend
on how many particles are there to determine how fast they move. An
increase in particle flux $\v{\Gamma}_{n\alpha}$ can be made either
by increasing the local density $n_{\alpha}$ while keeping $\v V_{\alpha}$
constant, or by increasing $\v V_{\alpha}$ at constant $n_{\alpha}$.
Thus, our family of related but different velocities has grown to
$\dot{\v r}_{k}(t),\v v_{k}(t),\v v,\v V_{\alpha}(\v r,t)$. This
is a good point to rest and meditate on what you have learned.

\subsubsection*{Fluid momentum law for each single species}

From the continuity equation (\ref{eq:falpha-2-1-1}) resulting from
a \emph{zeroth} moment of the plasma kinetic equation (\ref{eq:kincon-1})
we remember that a \emph{first} moment popped up in form of a flux
that we could translate into a fluid velocity $\v V_{\alpha}$ of
species $\alpha$. Going back to a single species we now take a moment
with the kinematic momentum $m_{\alpha}N_{\alpha}\v v$ as a weight,
so $\int\d^{3}vm_{\alpha}N_{\alpha}\v v$. This is a \emph{first}
moment, as the velocity appears once. Applying it on $(I)$ we obtain
\begin{equation}
m_{\alpha}N_{\alpha}\int\d^{3}v\v v\frac{\partial f_{\alpha}}{\partial t}=\frac{\partial}{\partial t}\left(m_{\alpha}N_{\alpha}\int\d^{3}v\v vf_{\alpha}\right)=\frac{\partial(m_{\alpha}n_{\alpha}\v V_{\alpha})}{\partial t}.
\end{equation}
This is the change in time of the \textbf{fluid} \textbf{momentum
density} of species $\alpha$,\textbf{ }
\begin{equation}
\v p_{\alpha}^{\mathrm{fl}}(\v r,t)\equiv m_{\alpha}n_{\alpha}(\v r,t)\v V_{\alpha}(\v r,t).
\end{equation}
(remember, there was also additional electromagnetic momentum, see
concepts). We notice that it is equal to the \textbf{mass flux $\v{\Gamma}_{m\alpha}(\v r,t)$}
of species $\alpha$.

From $(II)$ we now get 
\begin{equation}
m_{\alpha}N_{\alpha}\int\d^{3}v\v v(II)=m_{\alpha}N_{\alpha}\int\d^{3}v\v v\nabla_{\v r}\cdot(\v vf_{\alpha})=\nabla\cdot(m_{\alpha}N_{\alpha}\int\d^{3}v\v v\v vf_{\alpha}).\label{eq:momflux1}
\end{equation}
representing the spatial divergence of a \emph{second }moment (you
see this will never end if we don't stop it). This term contains a
\emph{dyad} $\v v\text{\ensuremath{\v v}}\equiv\v v\otimes\v v$ with
components $v_{i}v_{j}$ (see concepts), making it a second order
tensor. The divergence of such a second-order tensor yields a vector
field, so we can add this to the change of the kinematic momentum
density $\partial\v p_{\alpha}^{\mathrm{kin}}/\partial t$ from $(I)$,
being also a vector field. We leave it as it is for now.

To resolve the mass-weighted second moment of $(III)$
\begin{equation}
m_{\alpha}N_{\alpha}\int\d^{3}v\v v(III)=m_{\alpha}N_{\alpha}\int\d^{3}v\v v\left(\nabla_{\boldsymbol{v}}\cdot\frac{\v F_{\alpha}f_{\alpha}}{m_{\alpha}}\right),
\end{equation}
we use the identity (\ref{eq:diaddiv})
\begin{equation}
\nabla\cdot(\v a\v b)=\v a\cdot\nabla\v b+(\nabla\cdot\v a)\v b
\end{equation}
with $\v a=\v F_{\alpha}f_{\alpha}$ and $\v b=\v v$ , so
\begin{equation}
\left(\nabla_{\boldsymbol{v}}\cdot(\v F_{\alpha}f_{\alpha})\right)\v v=\nabla_{\boldsymbol{v}}\cdot\left(f_{\alpha}\v F_{\alpha}\v v\right)-f_{\alpha}\v F_{\alpha}\cdot\underbrace{\nabla_{\v v}\v v}_{\v I}.
\end{equation}
Again the integral over the first term vanishes if we convert it to
a boundary integral via the divergence theorem for tensors (one can
always write it in components to see this, if unsure). The second
term just yields a velocity-space average of the force,
\begin{align}
m_{\alpha}N_{\alpha}\int\d^{3}v\v v(III) & =-\cancel{m_{\alpha}}N_{\alpha}\int\d^{3}v\frac{\v F_{\alpha}f_{\alpha}}{\cancel{m_{\alpha}}}\nonumber \\
 & =-N_{\alpha}e_{\alpha}\int\d^{3}v(\v E+\frac{1}{c}\v v\times\v B)f_{\alpha}\nonumber \\
 & =-e_{\alpha}(n_{\alpha}\v E+\frac{1}{c}\v{\Gamma}_{n\alpha}\times\v B)\nonumber \\
 & =-e_{\alpha}n_{\alpha}(\v E+\frac{1}{c}\v V_{\alpha}\times\v B),
\end{align}
This is the negative Lorentz force acting on the fluid of species
$\alpha$ with charge density $\rho_{e\alpha}=e_{\alpha}n_{\alpha}$.

The collisional term $(C)$ yields 
\begin{equation}
m_{\alpha}N_{\alpha}\int\d^{3}v\v v(C)=m_{\alpha}N_{\alpha}\int\d^{3}v\v v\left(\frac{\partial f_{\alpha}}{\partial t}\right)_{\mathrm{c}}\equiv\v R_{\alpha}.
\end{equation}
This contribution doesn't vanish as before, because collisions with
other species can insert or extract momentum to the fluid of species
$\alpha$. Contribution $\v R_{\alpha\alpha}$ from self-collisions
of species $\alpha$ must not change its fluid momentum, so we can
write
\begin{equation}
\v R_{\alpha}=\sum_{\beta\neq\alpha}\v R_{\alpha\beta}.
\end{equation}
As global momentum conservation is still valid for elastic collisions,
each collision can only add momentum that it removed from one species
to another species, but not create or destroy it. Thus the sum of
collision terms over all species must vanish,
\begin{equation}
\sum_{\alpha}\v R_{\alpha}=\sum_{\alpha}\sum_{\beta\neq\alpha}\v R_{\alpha\beta}=0.
\end{equation}

Now we can compare Eq.~(\ref{eq:momflux1}) to the divergence of
the \textbf{fluid momentum flux }defined by
\begin{equation}
\v{\Gamma}_{\v p\alpha}^{\mathrm{fl}}(\v r,t)\equiv\v V_{\alpha}(\v r,t)\v p_{\alpha}^{\mathrm{fl}}(\v r,t)=m_{\alpha}n_{\alpha}(\v r,t)\v V_{\alpha}(\v r,t)\v V_{\alpha}(\v r,t).
\end{equation}
At first sight the order of vectors in the dyad plays a role and we
have to write $\v V_{\alpha}$ that transports the vector field $\v p_{\alpha}^{\mathrm{fl}}$
in the front. In the result we see however that it doesn't matter.
\begin{align*}
m_{\alpha}N_{\alpha}\int\d^{3}v\v v\v vf_{\alpha} & =m_{\alpha}N_{\alpha}\int\d^{3}v\v V_{\alpha}\v vf_{\alpha}+m_{\alpha}N_{\alpha}\int\d^{3}v(\v v-\v V_{\alpha})\v vf_{\alpha}\\
 & =\underbrace{m_{\alpha}n_{\alpha}\v V_{\alpha}\v V_{\alpha}}_{\v{\Gamma}_{\v p\alpha}^{\mathrm{fl}}}+m_{\alpha}N_{\alpha}\int\d^{3}v(\v v-\v V_{\alpha})\v vf_{\alpha}.
\end{align*}
The remaining term apart from $\v{\Gamma}_{\v p\alpha}^{\mathrm{fl}}$
is called the \textbf{species pressure tensor }
\begin{align}
\t P_{\alpha} & \equiv m_{\alpha}N_{\alpha}\int\d^{3}v(\v v-\v V_{\alpha})(\v v-\v V_{\alpha})f_{\alpha}\nonumber \\
 & =m_{\alpha}N_{\alpha}\int\d^{3}v\v v\v vf_{\alpha}-\v{\Gamma}_{\v p\alpha}^{\mathrm{fl}}.
\end{align}
Here we could add ``$-\v V_{\alpha}$'' in the second term since
\begin{equation}
N_{\alpha}\int\d^{3}v(\v v-\v V_{\alpha})\v V_{\alpha}f_{\alpha}=\left(N_{\alpha}\int\d^{3}v(\v v-\v V_{\alpha})f_{\alpha}\right)\v V_{\alpha}
\end{equation}
and
\begin{equation}
N_{\alpha}\int\d^{3}v(\v v-\v V_{\alpha})f_{\alpha}=N_{\alpha}\int\d^{3}v\v vf_{\alpha}-N_{\alpha}\v V_{\alpha}\int\d^{3}vf_{\alpha}=n_{\alpha}\v V_{\alpha}-n_{\alpha}\v V_{\alpha}=0.
\end{equation}

The pressure tensor quantifies the average deviation $(\v v-\v V_{\alpha})$
away from the fluid velocity. Finally, we write the \textbf{conservative
momentum law for species }$\alpha$,
\begin{equation}
\frac{\partial(m_{\alpha}n_{\alpha}\v V_{\alpha})}{\partial t}+\nabla\cdot(m_{\alpha}n_{\alpha}\v V_{\alpha}\v V_{\alpha})=-\nabla\cdot\t P_{\alpha}+e_{\alpha}n_{\alpha}(\v E+\frac{1}{c}\v V_{\alpha}\times\v B)+\v R_{\alpha}.\label{eq:moma}
\end{equation}
Here we have moved $(III)$ and $\nabla\cdot\t P_{\alpha}$ already
to the right-hand side, as it appears as a source term. 

\subsubsection*{Single-fluid magnetohydrodynamics including multiple species}

As collisions don't change masses or charges of particles either,
we can easily extend the continuity equation of number density \ref{eq:falpha-2-1-1}
to variants for charge and mass density conservation. For this purpose,
it is more convenient to sum over species, as currents are a result
of opposite charges moving in opposite directions, and the total mass
density consists of all species. This approach is known as the \textbf{single-fluid}
formulation of magnetohydrodynamics, as opposed to the more complicated
\textbf{two-fluid} approach (see e.g. book of Freidberg for more information).

Taking a sum of Eq.~(\ref{eq:falpha-2-1-1}) over $\alpha$ weighted
by charges $e_{\alpha}$, we obtain \textbf{charge continuity}
\begin{equation}
\frac{\partial\rho(\v r,t)}{\partial t}+\nabla\cdot\v J(\v r,t)=0,\label{eq:falpha-2-1-1-2}
\end{equation}
where the charge density in the partial time derivative 
\begin{equation}
\rho(\v r,t)\equiv\sum_{\alpha}e_{\alpha}n_{\alpha}(\v r,t)
\end{equation}
effectively vanishes due to quasineutrality, and the \textbf{single-fluid
current density}
\begin{equation}
\v J(\v r,t)\equiv\sum_{\alpha}e_{\alpha}\v{\Gamma}_{n\alpha}(\v r,t)=\sum_{\alpha}e_{\alpha}n_{\alpha}(\v r,t)\v V_{\alpha}(\v r,t)
\end{equation}
therefore becomes divergence-free. Since dynamics of electrons are
usually faster than the one of ions due to their mass difference,
$\v j$ will be mainly governed by electron motion $\v V_{e}$.

Weighting by $m_{\alpha}$ instead we obtain \textbf{single-fluid
mass continuity}
\begin{equation}
\frac{\partial\rho_{m}(\v r,t)}{\partial t}+\nabla\cdot\v{\Gamma}_{m}(\v r,t)=0,\label{eq:falpha-2-1-1-2-1}
\end{equation}
where the mass density is
\begin{equation}
\rho_{m}(\v r,t)\equiv\sum_{\alpha}m_{\alpha}n_{\alpha}(\v r,t).
\end{equation}
Since electrons are much lighter than ions, this is mainly governed
by the ion mass, $\rho_{m}\approx\rho_{mi}$. The mass flux
\begin{align}
\v{\Gamma}_{m}(\v r,t) & =\sum_{\alpha}\v{\Gamma}_{m\alpha}(\v r,t)\equiv\sum_{\alpha}m_{\alpha}\v{\Gamma}_{n\alpha}(\v r,t)\nonumber \\
 & =\sum_{\alpha}m_{\alpha}n_{\alpha}(\v r,t)\v V_{\alpha}(\v r,t)=\rho_{m}(\v r,t)\v V(\v r,t).
\end{align}
has the dimension of a momentum density (see concepts). This allows
us to define the ``effective'' \textbf{single-fluid velocity 
\begin{equation}
\v V(\v r,t)\equiv\frac{\sum_{\alpha}m_{\alpha}n_{\alpha}(\v r,t)\v V_{\alpha}(\v r,t)}{\rho_{m}(\v r,t)}.\label{eq:vel}
\end{equation}
}This fluid velocity is the one responsible for mass transport and
is a combination of ion and electron velocity due to $m_{i}\gg m_{e}$
but $|\v V_{e}|\gg|\v V_{i}|$. Be careful that the \textbf{single-fluid
MHD} \textbf{pressure tensor} $\t P$ is now defined via the deviation
with respect to $\v V$ rather than $\v V_{\alpha}$, so
\begin{equation}
\t P\equiv\sum_{\alpha}m_{\alpha}N_{\alpha}\int\d^{3}v(\v v-\v V)(\v v-\v V)f_{\alpha}.
\end{equation}

Performing the same derivation as for (\ref{eq:moma}) via second
moments, but now using this $\t P$ and taking a sum over all species
$\alpha$ yields the \textbf{conservative single-fluid momentum law}
\begin{equation}
\frac{\partial(\rho_{m}\v V)}{\partial t}+\nabla\cdot(\rho_{m}\v V\v V)=-\nabla\cdot\t P+\rho\v E+\frac{1}{c}\v J\times\v B,\label{eq:momentum}
\end{equation}
where the collisional term on the right-hand side has vanished and
the single-fluid velocity $\v V$ enters via its definition in Eq.~(\ref{eq:vel}).
Note that now the total current $\v J$ appears in the Lorentz force
term, but it is independent from the flow velocity $\v V$, since
$\v J$ is a difference and $\v V$ is a weighted sum over electron
and ion velocities $\v V_{\alpha}$. As mentioned before the electrostatic
force $\rho_{e}\v E$ is usually negligible due to quasineutrality.

Often the left-hand side of Eq.~\ref{eq:momentum} is modified with
help of mass continuity (\ref{eq:falpha-2-1-1-2-1}). Namely, using
Leibniz' product rule and the law for the divergence of dyads (\ref{eq:diaddiv})
for $\nabla\cdot((\rho_{m}\v V)\v V)$ we obtain
\begin{align}
\frac{\partial(\rho_{m}\v V)}{\partial t}+\nabla\cdot(\rho_{m}\v V\v V) & =\frac{\partial\rho_{m}}{\partial t}\v V+\rho_{m}\frac{\partial\v V}{\partial t}+(\rho_{m}\v V\cdot\nabla)\v V+(\nabla\cdot(\rho_{m}\v V))\v V\nonumber \\
 & =\rho_{m}\frac{\partial\v V}{\partial t}+\rho_{m}\v V\cdot\nabla\v V+\left(\frac{\partial\rho_{m}}{\partial t}+\nabla\cdot(\rho_{m}\v V)\right)\v V.
\end{align}
The term in brackets vanishes due to Eq.~(\ref{eq:falpha-2-1-1-2-1})
and we are left with the \textbf{convective single-fluid momentum
law}
\begin{equation}
\rho_{m}\frac{\partial\v V}{\partial t}+\rho_{m}\v V\cdot\nabla\v V=-\nabla\cdot\t P+\rho_{e}\v E+\frac{1}{c}\v J\times\v B.
\end{equation}

\begin{spacing}{0.9}
This corresponds to \textbf{Euler's equation} of inviscious fluid
mechanics with a scalar perssure if $\t P$ if it is a diagonal tensor
field,
\begin{equation}
\t P=\left(\begin{array}{ccc}
p & 0 & 0\\
0 & p & 0\\
0 & 0 & p
\end{array}\right)=p\t I
\end{equation}
then $\nabla\cdot\t P=\nabla p$. More generally we define the \textbf{scalar
pressure} $p$ as the isotropic part of $\t P$ over the trace
\begin{equation}
p\equiv\frac{1}{3}\mathrm{Tr}\,\t P.
\end{equation}
The rest we call the \textbf{anistropic part of the pressure tensor}
\begin{equation}
\v{\Pi}\equiv\t P-p\t I,
\end{equation}
which is responsible for viscosity. It can e.g. arise from gyration
(\textbf{gyroviscosity}), but for the remaining lecture we set $\v{\Pi}=0$,
as it is often done in practice. Instead of going on to equations
for higher moments above the second moment $p$, we make the ad-hoc
assumption that
\begin{align}
\frac{\d}{\d t}\left(\frac{p}{\rho_{m}^{\,\gamma}}\right) & =0,
\end{align}
being the \textbf{adiabatic equation}.\textbf{ }Take $\gamma=5/3$
for a monoatomic gas. 
\end{spacing}

This is justified if the process of interest happens much faster than
heat can flow in the system which is usually the case for waves and
instabilities. One can see this approximation as anaalogue to the
starting point to derive acoustic waves, but with more wave types,
including electromagnetic interaction. A more complete derivation
is done via the energy flux law and its limiting cases (see e.g. the
book ``Fundamentals of Plasma Physics'' by Paul Bellan for a good
explanation). The opposite to the adiabatic law is the isothermal
condition that can be used for processes much slower than the time-scale
of heat condution across the system.

\section{Plasma Electrodynamics}

\subsection*{6.1 The Vlasov-Maxwell-system}

Before we continue entirely in the magnetohydrodynamic/fluid picture
we look at how one would generally derive a \textbf{plasma response}
to an externally imposed electromagnetic field including multi-species
collisionless kinetic effects. This is known as the \textbf{Vlasov-Maxwell
system}. It is given by the following equations,

\begin{spacing}{0.3}
\begin{align}
\text{Vlasov:}\qquad\frac{\partial f_{\alpha}}{\partial t}+\boldsymbol{v}\cdot\nabla_{\boldsymbol{r}}f_{\alpha}+\frac{e_{\alpha}}{m_{\alpha}}(\v E+\frac{1}{c}\v v\times\v B)\cdot\nabla_{\boldsymbol{v}}f_{\alpha} & =0.\label{eq:falpha-2-1-2-1-1-1}
\end{align}
\begin{align}
\text{Induced charge density:}\qquad\rho^{\mathrm{ind}} & =\sum_{\alpha}e_{\alpha}n_{\alpha},\\
\text{Induced current density:}\qquad\boldsymbol{J}^{\mathrm{ind}} & =\sum_{\alpha}e_{\alpha}n_{\alpha}\v V_{\alpha}.
\end{align}
\begin{align}
\text{Gauss \textbf{E}:}\qquad\nabla\cdot\v E & =4\pi(\rho^{\mathrm{ind}}+\rho^{\mathrm{ext}}),\\
\text{Amp�re/Maxwell:}\qquad\nabla\times\v B & =\frac{4\pi}{c}(\boldsymbol{J}^{\mathrm{ind}}+\v J^{\mathrm{ext}})+\frac{1}{c}\frac{\partial\v E}{\partial t},\\
\text{Faraday:}\qquad\nabla\times\v E & =-\frac{1}{c}\frac{\partial\v B}{\partial t},\\
\text{Gauss \textbf{B}:}\qquad\nabla\cdot\v B & =0.
\end{align}

\end{spacing}

An interesting detail is the question of the continuity equation.
From taking moments of the Vlasov equation we know that continuity
holds for induced charges and currents with.
\begin{equation}
\frac{\partial\rho^{\mathrm{ind}}}{\partial t}+\nabla\cdot\boldsymbol{J}^{\mathrm{ind}}=0.
\end{equation}
Continuity of total $\rho=\rho^{\mathrm{ind}}+\rho^{\mathrm{ext}}$
and $\v J=\boldsymbol{J}^{\mathrm{ind}}+\v J^{\mathrm{ext}}$ is \emph{required}
in order to be compatible with Maxwell (combine Gauss E with $\nabla\cdot$Amp/Max),
so
\begin{equation}
\frac{\partial\rho}{\partial t}+\nabla\cdot\boldsymbol{J}=0.
\end{equation}
As a consequence, there \emph{must} be continuity of external charges
and currents,
\begin{equation}
\frac{\partial\rho^{\mathrm{ext}}}{\partial t}+\nabla\cdot\boldsymbol{J}^{\mathrm{ext}}=0,
\end{equation}
if we want to call them as such. This is clearly fulfilled if also
$\rho^{\mathrm{ext}}$ and $\boldsymbol{J}^{\mathrm{ext}}$ ultimately
come from particles carrying charges and currents. The deeper conceptual
fact is though that Maxwell's equations couldn't be fulfilled by a
more ``unnatural'' type of source fields. One could say that the
continuity equation in $\rho$ and $\v J$ is a compatibility condition
for Maxwell's equations and can (must) always be assumed together
with them.

As a side note, if one takes the \textbf{stationary} limit, Maxwell's
equations reduce to
\begin{align}
\text{Gauss \textbf{E}:}\qquad\nabla\cdot\v E & =4\pi\rho,\\
\text{Amp�re/Maxwell:}\qquad\nabla\times\v B & =\frac{4\pi}{c}\boldsymbol{J},\\
\text{Faraday:}\qquad\nabla\times\v E & =0,\\
\text{Gauss \textbf{B}:}\qquad\nabla\cdot\v B & =0.
\end{align}
In that case, the continuity condition is reduced to divergence-freeness
of currents,
\begin{equation}
\nabla\cdot\boldsymbol{J}=0.
\end{equation}
For numerical approximations one must choose an adequate discrete
basis ($H^{\mathrm{div}}$ or Raviart-Thomas Elements) to correctly
represent $\v J$ for this purpose -- it is not enough to just discretize
each vector component individually.

\subsection*{6.2 Linear plasma response}

The \textbf{goal} of this section is to derive the \textbf{response
fields }$\v E$ and $\v B$ inside a plasma subject to externally
imposed \textbf{source fields }$\rho^{\mathrm{ext}},\v J^{\mathrm{ext}}$,
taking also the reaction of induced charges and currents $\rho^{\mathrm{ind}},\boldsymbol{J}^{\mathrm{ind}}$
into account.

Our \textbf{strategy} will be to relate $\rho^{\mathrm{ind}},\boldsymbol{J}^{\mathrm{ind}}$
to $\v E$ and $\v B$ and move the resulting expressions to the left-hand
side of Maxwell's equations, leaving only $\rho^{\mathrm{ext}},\v J^{\mathrm{ext}}$
as source terms. This is similar to splitting induced and free charges
in the classical derivation of Maxwell's equation in a material with
permittivity $\varepsilon$ and permeability $\mu$. To relate $\boldsymbol{J}^{\mathrm{ind}}$
to $\rho^{\mathrm{ind}}$ we already know that we can use \textbf{charge
continuity}. A relation between $\boldsymbol{J}^{\mathrm{ind}}$ and
$\v E$ can be made without specific assumptions via a \textbf{generalized
Ohm's law}.

The main \textbf{assumption }we make is that $\v E$ and $\v B$ of
the plasma are linearly related to $\rho^{\mathrm{ind}},\boldsymbol{J}^{\mathrm{ind}}$
such that we can assume a generalized Ohm's law at the first place.
This would have to be justified from analyzing the solution of the
plasma kinetic (Vlasov) equation in the given circumstances, which
is not a trivial task. As a consequence, since Maxwell's equations
are linear, we also obtain a linear response in $\rho^{\mathrm{ext}},\v J^{\mathrm{ext}}$.
To obtain an explicit solution, we have to make the assumption\textbf{\emph{
}}that our plasma can be treated as an infinite and homogeneous system.
In that case, we can work in the frequency/wavenumber space where
differential operators and convolution integrals turn into simple
products.

\subsubsection*{Generalized Ohm's law and plasma response}

We would like to generalize Ohm's law with conductivity $\sigma$,
\begin{equation}
\boldsymbol{J}^{\mathrm{ind}}=\sigma\v E.
\end{equation}

The response of a medium (here: plasma) is said to be \textbf{linear}
if it follows 

\begin{equation}
\boldsymbol{J}^{\mathrm{ind}}(\v r,t)=\int_{t_{0}}^{t}\d t^{\prime}\int_{\Omega}\d^{3}r^{\prime}\,\v{\sigma}(\v r,t,\v r^{\prime},t^{\prime})\cdot\v E(\v r^{\prime},t^{\prime})\equiv\hat{\v{\sigma}}\v E.\label{eq:Jind}
\end{equation}
with a rank-2 \textbf{conductivity tensor} $\v{\sigma}(t,\tau,\v r,\v{\xi})$
that may introduce \textbf{anisotropy}. In the short-hand notation
the law is a \textbf{linear} integral operator 
\begin{equation}
\hat{\v{\sigma}}\equiv\int_{t_{0}}^{t}\d t^{\prime}\int_{\Omega}\d^{3}r^{\prime}\,\v{\sigma}(\v r,t,\v r^{\prime},t^{\prime})\cdot
\end{equation}
acting on $\v E$. Linearity means that
\begin{equation}
\hat{\v{\sigma}}(\alpha\v E_{1}+\beta\v E_{2})=\alpha\hat{\v{\sigma}}\v E_{1}+\beta\hat{\v{\sigma}}\v E_{2},
\end{equation}
where $\alpha$ and $\beta$ are constants. The response is \textbf{non-local
}in space and time since it relates \textbf{response} current $\boldsymbol{J}^{\mathrm{ind}}(\v r,t)$
to \textbf{excitation }field $\v E(\v r^{\prime},t^{\prime})$ at
any other point $(\v r^{\prime},t^{\prime})$ in space/time. It is
\textbf{causal} since integration over $t^{\prime}$ goes until $t$,
but not in the future. Formally we can relate $\boldsymbol{J}^{\mathrm{ind}}$
and $\rho^{\mathrm{ind}}$ via continuity
\begin{align}
\rho^{\mathrm{ind}}(\v r,t) & =\rho^{\mathrm{ind}}(\v r,t_{0})-\int_{t_{0}}^{t}\nabla\cdot\boldsymbol{J}^{\mathrm{ind}}(\v r,t)\,\d t^{\prime}\nonumber \\
 & =\rho^{\mathrm{ind}}(\v r,t_{0})-\nabla\cdot\int_{t_{0}}^{t}\boldsymbol{J}^{\mathrm{ind}}(\v r,t)\,\d t^{\prime}\nonumber \\
 & \equiv\rho^{\mathrm{ind}}(\v r,t_{0})+\nabla\cdot(\hat{\rho}\boldsymbol{J}^{\mathrm{ind}})=\rho^{\mathrm{ind}}(\v r,t_{0})+\nabla\cdot(\hat{\rho}\hat{\v{\sigma}}\v E).
\end{align}
We can assume $\rho^{\mathrm{ind}}(\v r,t_{0})=0$ because perturbation
is switched off before a certain time $t_{0}$. Then $\rho^{\mathrm{ind}}$
is again given by a linear operator
\begin{equation}
\rho^{\mathrm{ind}}(\v r,t)=\nabla\cdot\hat{\rho}\hat{\v{\sigma}}\v E
\end{equation}
acting on $\v E$ by combining divergence $\nabla\cdot$, charge accumulation
$\hat{\rho}$ and conductivity $\hat{\v{\sigma}}$. Inserting these
relations into the first two Maxwell equations yields
\begin{align}
\text{Gauss \textbf{E}:}\qquad\nabla\cdot(\t I-4\pi\hat{\rho}\hat{\v{\sigma}})\v E & =4\pi\rho^{\mathrm{ext}},\\
\text{Amp�re/Maxwell:}\qquad\nabla\times\v B-\frac{1}{c}\left(\frac{\partial}{\partial t}+4\pi\hat{\v{\sigma}}\right)\v E & =\frac{4\pi}{c}\v J^{\mathrm{ext}}.
\end{align}
To obtain a generalized Gauss \textbf{E} law $\nabla\cdot(\hat{\v{\varepsilon}}\v E)=4\pi\rho^{\mathrm{ext}}$
for plasma seen as a dielectric materials we can define a tensorial
permittivity operator
\begin{equation}
\hat{\v{\varepsilon}}\equiv\t I-4\pi\hat{\rho}\hat{\v{\sigma}}.\label{eq:permi}
\end{equation}
Taking a time derivative of Amp�re/Maxwell and inserting Faraday for
$\frac{\partial\v B}{\partial t}$ yields
\begin{equation}
-c^{2}\nabla\times(\nabla\times\v E)-\left(\frac{\partial^{2}}{\partial t^{2}}+4\pi\frac{\partial}{\partial t}\hat{\v{\sigma}}\right)\v E=4\pi\frac{\partial}{\partial t}\v J^{\mathrm{ext}}.
\end{equation}
Combining everything on the left-hand side to a single operator yields
our final system
\begin{align}
\nabla\cdot(\t I-4\pi\hat{\rho}\hat{\v{\sigma}})\v E & =4\pi\rho^{\mathrm{ext}},\label{eq:gauss}\\
\left(\frac{\partial^{2}}{\partial t^{2}}+c^{2}\nabla\times\nabla\times+4\pi\frac{\partial}{\partial t}\hat{\v{\sigma}}\right)\v E & =-4\pi\frac{\partial}{\partial t}\v J^{\mathrm{ext}}.\label{eq:wave}
\end{align}
Compare this result to the derivation of electromagnetic waves in
a homogeneous, charge- and current-free vacuum where $\rho^{\mathrm{ext}}=0$,
$\v J^{\mathrm{ext}}=0$,$\hat{\v{\sigma}}=0$. There we obtain divergence-free
$\v E$ and the electromagnetic wave equation,
\begin{align}
\nabla\cdot\v E & =0,\\
\left(\frac{\partial^{2}}{\partial t^{2}}+c^{2}\nabla\times\nabla\times\right)\v E & =0.
\end{align}
There the curl-curl operator is usually replaced by a vector Laplacian
defined as $\Delta\v E\equiv\nabla(\nabla\cdot\v E)-\nabla\times(\nabla\times\v E)$,
since the first term vanishes in vacuum due to charge-freeness in
Gauss \textbf{E}. In the plasma, even for vanishing excitations $\rho^{\mathrm{ext}}$
and $\v J^{\mathrm{ext}}$ this is not necessarily the case due to
the additional plasma response terms
\begin{equation}
-4\pi\hat{\rho}\hat{\v{\sigma}},\quad\text{and}\quad4\pi\frac{\partial}{\partial t}\hat{\v{\sigma}}.
\end{equation}
Those will modify the usual electromagnetic waves depending on the
plasma conductivity encoded inside $\hat{\v{\sigma}}$. Such linear
response laws with modified electromagnetic wave equations are also
applicable to other materials besides plasma.

\subsubsection*{Solution in Fourier space for a homogeneous plasma}

If our plasma is infinite and homogeneous, there can be no material
parameters with dependencies on \emph{absolute} $\v r,t$ but only
\emph{relative} distances $(\v r-\v r^{\prime},t-t^{\prime})$ we
can write Eq.~(\ref{eq:Jind}) as

\begin{equation}
\boldsymbol{J}^{\mathrm{ind}}(\v r,t)=\int_{t_{0}}^{t}\d t^{\prime}\int_{\mathbb{R}^{3}}\d^{3}r^{\prime}\,\v{\sigma}(\v r-\v r^{\prime},t-t^{\prime})\cdot\v E(\v r^{\prime},t^{\prime}).\label{eq:Jconv}
\end{equation}
This is a \textbf{convolution} integral. In case we have an infinite
homogeneous space we can \textbf{Fourier transform }where quantities
are set up as $f(\v r,t)\propto f_{\omega\v k}e^{i(\v k\cdot\v r-\omega t)}$
(watch the minus, if something is finite and periodic, use a Fourier
series instead). This makes life easier with products instead of differential
operators, 
\begin{align}
\frac{\partial}{\partial t}\, & \rightarrow\,-i\omega,\qquad\nabla\,\rightarrow\,i\v k,\\
\nabla\cdot\, & \rightarrow\,i\v k\cdot,\qquad\nabla\times\,\rightarrow\,i\v k\times.
\end{align}
Convolution integrals transform like
\begin{equation}
\int_{-\infty}^{\infty}\d t^{\prime}\int_{\mathbb{R}^{3}}f(\v r-\v r^{\prime},t-t^{\prime})g(\v r^{\prime},t^{\prime})\quad\rightarrow\quad f_{\omega\v k}g_{\omega\v k}.
\end{equation}
Looking more carefully at Ohm's law (\ref{eq:Jconv}), we should treat
causality correctly in time, which can start at $t_{0}=-\infty$ but
should end at $t$ and not $+\infty$. Actually we need a \emph{half-sided}
Fourier transform or \emph{complex Laplace transform}. This concerns
mainly the back-transformation but all other rules for Fourier transforms
that we use remain the same. This is why we ignore this peculiarity
in the frequency domain for now, but should keep it in mind for later
($\rightarrow$ Landau damping). The generalized non-local \textbf{Ohm's
law} for a homogeneous (not necessarily isotropic) medium in $\omega\v k$
space thus becomes

\begin{equation}
\boldsymbol{J}_{\omega\v k}^{\mathrm{ind}}=\v{\sigma}_{\omega\v k}\cdot\v E_{\omega\v k}.\label{eq:Jconv-1}
\end{equation}

\textbf{Continuity} becomes

\begin{equation}
i\omega\rho_{\omega\v k}^{\mathrm{ind}}=i\v k\cdot\boldsymbol{J}_{\omega\v k}^{\mathrm{ind}}.\label{eq:conti-2}
\end{equation}

\textbf{Maxwell's equations }become (leaving out $\omega\v k$ subscripts
for readiliby)
\begin{align}
\text{Gauss \textbf{E}:}\qquad i\v k\cdot\v E & -4\pi\rho^{\mathrm{ind}}=4\pi\rho^{\mathrm{ext}},\\
\text{Amp�re/Maxwell:}\qquad i\v k\times\v B & -\frac{4\pi}{c}\boldsymbol{J}^{\mathrm{ind}}+\frac{i\omega}{c}\v E=\frac{4\pi}{c}\v J^{\mathrm{ext}},\\
\text{Faraday:}\qquad i\v k\times\v E & =\frac{i\omega}{c}\v B,\\
\text{Gauss \textbf{B}:}\qquad i\v k\cdot\v B & =0.
\end{align}
In $\omega\v k$ space Eqs.~(\ref{eq:gauss}-\ref{eq:wave}) can
either be derived from those results or directly transformed to become
the \textbf{linear plasma response in frequency space}
\begin{align}
i\v k\cdot(\t I-\frac{4\pi}{i\omega}\v{\sigma}\cdot)\v E & =4\pi\rho^{\mathrm{ext}},\\
i\omega\left(1+\frac{c^{2}}{\omega^{2}}\v k\times\v k\times-\frac{4\pi}{i\omega}\v{\sigma}\cdot\right)\v E & =4\pi\v J^{\mathrm{ext}}.
\end{align}
The second equation being a wave equation in $\omega\v k$ space is
often written in a different form where the double cross-product is
replaced by
\begin{equation}
\v k\times\v k\times\v E=\v{kk}\cdot\v E-k^{2}\v E.
\end{equation}
Defining further the \textbf{wave vector}
\begin{equation}
\v N\equiv\frac{c\v k}{\omega}
\end{equation}
we write the \textbf{modified EM wave equation}
\begin{equation}
i\omega\left(\v{NN}-N^{2}\t I+\t I-\frac{4\pi}{i\omega}\v{\sigma}\right)\cdot\v E=4\pi\v J^{\mathrm{ext}}.\label{eq:wave-1}
\end{equation}
Transforming the permittivity operator~\ref{eq:permi} into $\omega\v k$
space we obtain the \textbf{dielectric tensor}
\begin{equation}
\v{\varepsilon}\equiv\t I-\frac{4\pi}{i\omega}\v{\sigma}.
\end{equation}
The total expression in front of $\v E$ in the wave equation called
the \textbf{Maxwell tensor}
\begin{equation}
\v{\Lambda}\equiv\v{NN}-N^{2}\t I+\v{\varepsilon}
\end{equation}
We call modes that follow purely from Gauss \textbf{E }in the plasma
the \textbf{longitudinal modes }
\begin{equation}
i\v k\cdot\v{\varepsilon}\cdot\v E=4\pi\rho^{\mathrm{ext}}.
\end{equation}
Those from the modified EM wave equation are the \textbf{transverse
modes}
\begin{equation}
i\omega\v{\Lambda}\cdot\v E=4\pi\v J^{\mathrm{ext}}.
\end{equation}
In a region without sources the problem reduces to an eigenvalue problem
of the \textbf{dispersion equation} 
\begin{equation}
\mathrm{det}\v{\Lambda}=0.
\end{equation}
Since $\v{\Lambda}$ depends on both, $\omega$ and $\v k$, this
gives a dispersion relation
\begin{equation}
\omega=\omega^{\sigma}(\v k)
\end{equation}
between the two for each mode $\sigma$, determining the \textbf{polarization}
of the mode.
\begin{thebibliography}{99}
\bibitem{Callen1991} D\textquoteright haeseleer, W. D., Hitchon,
W. N. G., Callen, J. D., \& Shohet, J. L. (1991). Flux Coordinates
and Magnetic Field Structure. Springer Berlin Heidelberg. https://doi.org/10.1007/978-3-642-75595-8
\end{thebibliography}

\end{document}
