%% LyX 2.3.2 created this file.  For more info, see http://www.lyx.org/.
%% Do not edit unless you really know what you are doing.
\documentclass[english,notitlepage]{revtex4-1}
\usepackage{tgpagella}
\usepackage[T1]{fontenc}
\usepackage[latin9]{inputenc}
\setcounter{secnumdepth}{3}
\usepackage{amsbsy}

\makeatletter
%%%%%%%%%%%%%%%%%%%%%%%%%%%%%% User specified LaTeX commands.
\usepackage{tikz}

\makeatother

\usepackage{babel}
\begin{document}
\global\long\def\tht{\vartheta}%
\global\long\def\ph{\varphi}%
\global\long\def\balpha{\boldsymbol{\alpha}}%
\global\long\def\btheta{\boldsymbol{\theta}}%
\global\long\def\bJ{\boldsymbol{J}}%
\global\long\def\bGamma{\boldsymbol{\Gamma}}%
\global\long\def\bOmega{\boldsymbol{\Omega}}%
\global\long\def\d{\text{d}}%
\global\long\def\t#1{\text{#1}}%
\global\long\def\m{\text{m}}%
\global\long\def\v#1{\boldsymbol{#1}}%
\global\long\def\u#1{\underline{#1}}%

\global\long\def\t#1{\mathbf{#1}}%
\global\long\def\bA{\boldsymbol{A}}%
\global\long\def\bB{\boldsymbol{B}}%
\global\long\def\c{\mathrm{c}}%
\global\long\def\difp#1#2{\frac{\partial#1}{\partial#2}}%
\global\long\def\xset{{\bf x}}%
\global\long\def\zset{{\bf z}}%
\global\long\def\qset{{\bf q}}%
\global\long\def\pset{{\bf p}}%

\title{Neoclassical Toroidal Viscous Torque}
\author{Christopher Albert}
\date{\today}

\maketitle
Neoclassical toroidal viscous torque, often called neoclassical toroidal
viscosity or NTV, is a result from non-axisymmetric perturbations
in an originally axisymmetric plasma equilibrium of a tokamak. The
underlying theory of neoclassical transport relies on distorted but
still nested flux surfaces, i.e. \emph{non-resonant }magnetic perturbations.
In case of \emph{resonant} magnetic perturbations (RMPs) one must
exclude resonant surfaces from the analysis that yield a different
mechanism of \emph{resonant torque}. At sufficient distance from the
resonant regions NTV theory is then applicable, and the overall torque
is the sum of non-resonant NTV torque and resonant torque, respectively
dominant in different radial regions of the plasma. Apart from additional
resonant torque contributions an important point affecting the expected
accuracy of NTV results is the reliance on a on a given perturbed
plasma equilibrium for neoclassical computations. Finally, both, resonant
and NTV torque affect this equilibrium, so it must be found in a self-consistent
way. Such an effort including only NTV has been undertaken in~\citep{Park2017}.
A full treatment must however also include resonant torque, requiring
a full Monte-Carlo kinetic computation coupled to Maxwell's equations
necessary~\citep{Albert2016}. Analysis of NTV is still useful, as
it allows to study and quantify the influence of the non-resonant
part of the perturbation on its own and is much less computationally
intensive than the full kinetic treatment of perturbed equilibria.

Generally we speak of toroidal torque as the driving term affecting
\emph{kinematic }toroidal angular momentum, as opposed to the total
toroidal momentum including electromagnetic effects that is always
conserved. Apart from a possible kinematic momentum source term from
neutral beam injection, internal torque affecting toroidal plasma
rotation is always electromagnetic, i.e. linked to the Lorentz $\v J\times\v B$
force. Within ideal MHD equilibria, currents $\v J$ and magnetic
field $\v B$ are parallel to flux surfaces and do not produce any
forces, since radial Lorentz forces balanced by $\nabla p$ via
\begin{equation}
\nabla p=\v J\times\v B.
\end{equation}
The MHD force balance is also valid in non-axisymmetric fields, where
ideal treatment is only possible away from resonant regions where
surfaces of $p=p_{0}+\delta p=\mathrm{const.}$ coincide with perturbed
flux surfaces $r=r_{0}+\delta r=\mathrm{const}$. When taking neoclassical
transport into account one can show that it is radially \emph{ambipolar}
in axisymmetric devices to leading order, i.e. there is no net radial
charge transport by different species. This changes in a perturbed
configuration, where the transport across the perturbed flux surfaces
(see Fig.~\ref{fig:Perturbed-and-unperturbed}) becomes \emph{non-ambipolar}
and generates a net radial current.

\begin{figure}
\input{pertunpert.tpx}

\caption{Perturbed and unperturbed flux surfaces. The projection of perturbed
currents $J^{r_{0}}=\protect\v J\cdot\nabla r_{0}$ to the original
radial direction $\nabla r_{0}$ is of the order one and periodic
in flux surface angles. For NTV we are rather interested in the small
$\delta J^{r}=\protect\v J\cdot\nabla r$ produced by non-ambipolar
transport across perturbed flux surfaces $r=\mathrm{const.}$\label{fig:Perturbed-and-unperturbed}}

\end{figure}

The resulting toroidal torque density is defined as the covariant
component with respect to the toroidal angle $\ph$ parametrizing
the \emph{perturbed} flux surfaces,
\begin{equation}
\pi_{\ph}=(\v J\times\v B)_{\ph}=\frac{1}{\sqrt{g}}\varepsilon\delta J^{r}B^{\tht}.\label{eq:lag}
\end{equation}
Here we have marked $\delta J^{r}=\v J\cdot\nabla r$ as a small quantity
resulting from $\v J=\v J_{0}+\delta\v J$. This is not to be confused
with the torque with respect to \emph{unperturbed} flux surfaces having
different coordinates $(r_{0},\tht_{0},\ph_{0})$,
\begin{equation}
\pi_{\ph_{0}}=(\v J\times\v B)_{\ph_{0}}=\frac{1}{\sqrt{g}}(\delta J^{r_{0}}B^{\tht_{0}}-J^{\tht_{0}}\delta B^{r_{0}}).\label{eq:eul}
\end{equation}
Here also the radial component $\delta B^{r_{0}}=\v B\cdot\nabla r_{0}$
from $\v B=\v B_{0}+\delta\v B$ enters, where $\v B_{0}\cdot\nabla r_{0}=0$
on unperturbed flux surfaces. The contribution 
\begin{equation}
J^{r_{0}}=\v J\cdot\nabla r_{0}=\v J\cdot\nabla(r-\delta r)=\delta J^{r}-\v J\cdot\nabla\delta r.
\end{equation}
now contains not only small currents from $\delta J^{r}$ non-ambipolar
transport , but also currents parallel to perturbed flux surfaces,
containing angular components of the overall (order of $\v J_{0}$)
current $\v J$ that is not parallel to surfaces $r_{0}=\mathrm{const.}$
but rather to $r=\mathrm{const.}$ due to the distortion. Since the
perturbation is periodic in angles, those additional periodic contribution
cancel out when computing the flux-surface averaged radial torque
density, and the two results become equivalant to leading order in
the perturbation. In case of destroyed flux surfaces it is no longer
possible to use Eq.~(\ref{eq:lag}) and one can only compute overall
(including resonant) torque within Eq.~\ref{eq:eul} and flux-surface
averaging. In contrast, NTV torque corresponds to the toroidal component
of the torque generate by radial currents of non-ambipolar neoclassical
transport with respect to perturbed flux surfaces and can be most
compactly formulated in the Lagrangian picture working \emph{on }those
surfaces in Eq.~(\ref{eq:lag}). Later we can make use of the assumption
that the separation of perturbed from unperturbed flux surfaces is
infintesimal, and use flux coordinates and orbits with regard to the
unperturbed surfaces while evaluating toroidal torque on perturbed
surfaces.
\begin{thebibliography}{99}
\bibitem{Park2017}J.-K. Park and N. C. Logan, \textquotedblleft Self-consistent
perturbed equilibrium with neoclassical toroidal torque in tokamaks,\textquotedblright{}
Phys. Plasmas, vol. 24, no. 3, p. 032505, Mar. 2017.

\bibitem{Albert2016}C. G. Albert, M. F. Heyn, S. V Kasilov, W. Kernbichler,
A. F. Martitsch, and A. M. Runov, \textquotedblleft Kinetic modeling
of 3D equilibria in a tokamak,\textquotedblright{} J. Phys. Conf.
Ser., vol. 775, no. 1, p. 012001, Nov. 2016.
\end{thebibliography}

\end{document}
