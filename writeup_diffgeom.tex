%% LyX 2.3.2 created this file.  For more info, see http://www.lyx.org/.
%% Do not edit unless you really know what you are doing.
\documentclass[english]{elsarticle}
\usepackage{tgpagella}
\usepackage[T1]{fontenc}
\usepackage[latin9]{inputenc}
\usepackage{geometry}
\geometry{verbose,tmargin=3cm,bmargin=3cm,lmargin=3cm,rmargin=3cm}
\usepackage{babel}
\usepackage{url}
\usepackage{amsmath}
\usepackage{amssymb}
\usepackage{graphicx}
\usepackage[unicode=true]
 {hyperref}

\makeatletter

%%%%%%%%%%%%%%%%%%%%%%%%%%%%%% LyX specific LaTeX commands.
%% Because html converters don't know tabularnewline
\providecommand{\tabularnewline}{\\}

\makeatother

\begin{document}
\global\long\def\tht{\vartheta}%
\global\long\def\ph{\varphi}%
\global\long\def\balpha{\boldsymbol{\alpha}}%
\global\long\def\btheta{\boldsymbol{\theta}}%
\global\long\def\bJ{\boldsymbol{J}}%
\global\long\def\bGamma{\boldsymbol{\Gamma}}%
\global\long\def\bOmega{\boldsymbol{\Omega}}%
\global\long\def\d{\text{d}}%
\global\long\def\t#1{\text{#1}}%
\global\long\def\m{\text{m}}%
\global\long\def\v#1{\boldsymbol{#1}}%
\global\long\def\u#1{\underline{#1}}%

\global\long\def\t#1{\mathbf{#1}}%
\global\long\def\bA{\boldsymbol{A}}%
\global\long\def\bB{\boldsymbol{B}}%
\global\long\def\c{\mathrm{c}}%
\global\long\def\difp#1#2{\frac{\partial#1}{\partial#2}}%
\global\long\def\xset{{\bf x}}%
\global\long\def\zset{{\bf z}}%
\global\long\def\qset{{\bf q}}%
\global\long\def\pset{{\bf p}}%
\global\long\def\wset{{\bf w}}%
\global\long\def\rset{{\bf r}}%
\global\long\def\yset{\mathbf{y}}%


\title{A short introduction to differential geometry\\
from the perspective of tensor calculus}

\author{Christopher Albert}

\address{Max-Planck-Institut f�r Plasmaphysik, Boltzmannstra�e 2, 85748 Garching,
Germany\\
<albert@alumni.tugraz.at>}
\begin{abstract}
In this text a short explanation on the treatment of space as a smooth
manifold and phase-space as a symplectic manifold for readers acquainted
to terminology from tensor calculus in co-/contravariant notation
is given. The main aim of this text is to provide a fast track to
working knowledge of the discussed concepts. For more details, the
three textbooks of Arnold, Jose/Saletan and Marsden should be helpful
for further clarifications and deeper understanding. For a thorough
introduction into smooth manifolds, the video lectures of Frederic
Schuller from the WE-Heraeus International Winter School on Gravity
and Light are highly recommended. A review concerned with guiding-center
dynamics and in more traditional notation is given by Cary/Brizard
at \href{https://journals.aps.org/rmp/pdf/10.1103/RevModPhys.81.693}{https://journals.aps.org/rmp/pdf/10.1103/RevModPhys.81.693}
. A main feature distinguishing differential geometry from tensor
calculus is the treatment of covectors as first-class objects in addition
to vectors. As a concequence in many derivations no metric tensor
has to be specified, and the construction of structure-preserving
numerical methods becomes simpler and more transparent. Apart from
conceptual differences, the practical use of the formalisms is similar,
in particular the notation with upper and lower indexes for contravariant
and covariant components and the convention to sum over indexes appearing
once up and once down.
\end{abstract}
\date{\today}
\maketitle

\makeatletter 
\def\ps@pprintTitle{ 
 \let\@oddhead\@empty 
 \let\@evenhead\@empty                         \def\@oddfoot{\footnotesize\itshape\hfill\today} \def\@evenfoot{\thepage\hfill}
}
\makeatother

\section{Notation}

Here we fix some notational conventions for the following text. Scalars
and scalar functions are denoted in lowercase italic letters $a,f$
and for points $X,Q,Z$ we use uppercase italic letters. We use italic
bold letters to denote ensor fields, in particular vectors $\dot{\v q},\dot{\v z}$
and covectors $\v p,\v A$. Coordinate tuples are denoted by bold
upright letters $\xset,\zset,\qset,\pset$. Spatial coordinates $\xset$
have components $x^{k}$, canonical coordinates $\qset,\pset$ consist
of tuples $q^{i},p_{j}$, and generally non-canonical coordinates
in phase-space $\zset$ have entries $z^{\alpha}$. Lower index notation
for canonical momenta $p_{j}$ is used since in the classical construction
$p_{j}$ are not only coordinates in phase-space but also components
of covectors with respect to configuration space charted by coordinates
$q^{i}$. Latin indices $i,j,k,l$ run from $1$ to $3$ for spatial
coordinates $x^{k}$, or from $1$ to $N$ for canonical coordinates
in a system with $N$ degrees of freedom. Greek indices like $\alpha,\beta$
are used for general phase-space coordinates $z^{\alpha}$ and run
from $1$ to $2N$. We use the usual convention from tensor algebra
to sum over indices appearing once up and once down in formulas, i.e.
$p_{i}\dot{q}^{i}\equiv\sum_{i}p_{i}\dot{q}^{i}$, where indexes in
the denominator of derivatives switch their position, e.g. $\partial H/\partial q^{j}=p_{j}$
has a lower index. A common convention of physics literature is used
where functional dependencies such as $\zset(\qset,\pset),\zset(t)$
are denoted by the same letters as $\zset$ alone. If arguments are
not written explicitly inside functions the meaning should become
clear from the respective context.

\section{Basics}

Physics can be treated on different levels of abstraction. Here we
will start from the simplest picture and progress to the most advanced
one, expressing our dissatisfaction with the result in each step,
to justify an increase in the level of abstraction. The names of the
sub-sections are chosen from the author's experience and can of course
vary from school to school. The last subsection on the mathematical
theory of differential geometry is at the highest level of abstraction
and clarity and contains all other section as special cases and/or
more sloppy interpretations. This does not necesserily mean that one
should start on the top. In the view of the author it is rather convenient
to start from the most basic levels and see where conceptual problems
arise. Once the highest level is reached, one can take a step back
to the sometimes more practical levels, with the power of understanding
what objects we are actually talking about, and how they are interrelated.

\subsection{High-school physics}

The usual picture of high-school mathematics and physics is the following:
The world of classical physics can be described by points $X$ given
in Cartesian coordinates $\xset=(x,y,z)$, and time $t$ that is running
steadily. A vector is constructed as an arrow pointing from $X_{1}$
to $X_{2}$ with
\begin{equation}
\v u_{12}=\left(\begin{array}{c}
u_{12\,x}\\
u_{12\,y}\\
u_{12\,z}
\end{array}\right)\equiv\left(\begin{array}{c}
x_{2}-x_{1}\\
y_{2}-y_{1}\\
z_{2}-z_{1}
\end{array}\right).
\end{equation}
It is said to exists independently from its spreading point $P$ and
can be parallely translated in space. The length of a vector can be
computed by its norm
\begin{equation}
|\v u|=\sqrt{u_{x}^{\,2}+u_{y}^{\,2}+u_{z}^{\,2}}.
\end{equation}
To each point $P$ we can assign a vector from the origin point ``$0$'',
\begin{equation}
\v r(P)\equiv\v u^{0P}=\left(\begin{array}{c}
x\\
y\\
z
\end{array}\right),
\end{equation}
whose components are the same as the coordinates of $P$. The inner
product between vectors $\v u$ and $\v v$ is given by
\begin{equation}
\v u\cdot\v v=u_{x}v_{x}+u_{y}v_{y}+u_{z}v_{z}.
\end{equation}
By geometrical means we see that can be related to the angle $\theta^{\v{uv}}$
between the vectors via
\begin{equation}
\v u\cdot\v v=|\v u||\v v|\cos\theta^{\v u\v v}.
\end{equation}
We also notice that the norm is related to the inner product via $|\v u|=\sqrt{\v u\cdot\v u}$.
In addition we introduce the cross product
\begin{equation}
\v u\times\v v\equiv\left(\begin{array}{c}
u_{y}u_{z}-u_{z}u_{y}\\
u_{z}u_{x}-u_{x}u_{z}\\
u_{x}u_{y}-u_{y}u_{x}
\end{array}\right),
\end{equation}
yielding a vector as a result.

\subsubsection*{Mechanics}

A particle that starts at time $t=0$ at point $X$ with a constant
velocity vector $\v v$ will have a time-dependent position vector
\begin{equation}
\v r(t)=\v r(0)+\v vt
\end{equation}
that we can then translate back to a point $X(t)$. If a constant
vectorial force $\v F$ acts on a particle of mass $m$ we obtain
an accelaration vector 
\begin{equation}
\v a\equiv\frac{\d\v v(t)}{\d t}=\frac{\v F}{m}
\end{equation}
according to Newton's law. We then first compute the time-dependent
velocity vector by integrating once,
\begin{equation}
\v v(t)=\v v(0)+\frac{\v F}{m},
\end{equation}
and integrate another time to obtain
\begin{equation}
\v r(t)=\v r(0)+\v v(0)t+\frac{\v F}{2m}t^{2}.
\end{equation}


\subsubsection*{Field theory}

A vector fields $\v E(X)$ means to attach a vector to each point
$X$ in space, i.e. $\v E$ is like a function on components $\xset=(x,y,z)$
that has vectors as an output. In that sense we can view $\v r(X)$
as a vector field that yields Cartesian components of $X$ in each
point. For electromagnetism and gravity, the intuitive picture of
a \emph{flow} of a vector field through a surface is introduced. There
we take the normal components of a vector field to the surface, and
sum them up. Gauss' law can be derived by looking at the vectors in
the direction of a point charge and how they flow through a spherical
surface.

\subsubsection*{Why we are dissatisfied}

First of all, we could only use Cartesian coordinates. We practically
identified points, vectors, and their Cartesian components, despite
noticing that the inner product would work the same, if we translated
or rotated our coordinate frame. In mechanics and field theory we
were limited to constant forces and simple fields. For treating cylindrical
and spherical symmetry we had to rely on ad-hoc solutions by drawing
arrows, since using Cartesian coordinates there is very cumbersome.
The questions what space and time actually are, and why we can describe
them by numbers in the given way, remains completely open.

\subsection{Undergraduate physics}

In the undergraduate physics courses we keep the picture of vector
calculus and classical physics from high-school, but more advanced
concepts are introduced on several levels. A vector is generalized
from a geometric quantity to an abstract object in a linear vector
space that allows for addition and multiplication by scalars. Newton's
equations of motion are generalized to Lagrange's and Hamilton's equations
in generalized coordinates, and solved for space- and time-dependent
forces. In field theory we solve partial differential equations in
possibly curvilinear coordinates. Such a coordinate change is given
by a set of transformation equations
\begin{align}
x & =x(\xset)\equiv x(x^{1},x^{2},x^{3}),\\
y & =y(\xset)\equiv y(x^{1},x^{2},x^{3}),\\
z & =z(\xset)\equiv z(x^{1},x^{2},x^{3}),
\end{align}
or more compactly $\v r=\v r(\xset)$, with the position vector $\v r$
still identified with its Cartesian coordinates $(x,y,z)$ via
\begin{align}
\v r & \equiv x\hat{\v e}_{x}+y\hat{\v e}_{y}+z\hat{\v e}_{z}
\end{align}
even though we already anticipate that the concept of a position ``vector''
becomes problematic due to the possibility of coordinate changes.
If we want to be more precise and distinguish $X$ from $\v r$, we
look at the position vector field $\v r(X)$ and represent $X(\xset)$
via coordinates $\xset$, so when we write $\v r$, we actually mean
the vector-valued function $\v r(\xset)\equiv\v r(X(\xset))$ representing
the vector field $\v r(X)$ in coordinates $\xset$. Unit vectors
in curvilinear coordinates are given as derivatives of this function
via
\begin{equation}
\hat{\v e}_{k}=\frac{1}{|\partial\v r/\partial x^{k}|}\frac{\partial\v r}{\partial x^{k}},
\end{equation}
where we require an orthonormal set, i.e. $\hat{\v e}_{k}\cdot\hat{\v e}_{l}=\delta_{kl}$,
so $0$ if $k\neq l$ and $1$ if $k=l$. This is the case for many
sets of coordinates, including cylindrical and spherical coordinates,
that make our life much easier in electromagnetic field theory. To
manipulate vector fields we introduce the nabla symbol $\nabla$ as
a vector differential operator, producing gradient, curl, and divergence
via
\begin{align}
\nabla\Phi & \equiv\frac{\partial\Phi}{\partial x}\v e_{x}+\frac{\partial\Phi}{\partial y}\v e_{y}+\frac{\partial\Phi}{\partial z}\v e_{z},\\
\nabla\times\v A & \equiv\left(\frac{\partial A_{z}}{\partial y}-\frac{\partial A_{y}}{\partial z}\right)\v e_{x}+\left(\frac{\partial A_{x}}{\partial z}-\frac{\partial A_{z}}{\partial x}\right)\v e_{y}+\left(\frac{\partial A_{y}}{\partial x}-\frac{\partial A_{y}}{\partial x}\right)\v e_{z},\\
\nabla\cdot\v B & \equiv\frac{\partial B_{x}}{\partial x}+\frac{\partial B_{y}}{\partial y}+\frac{\partial B_{z}}{\partial z}.
\end{align}
We proof differential theorems of vector calculus, namely
\begin{equation}
\nabla\times(\nabla\Phi)=0,\quad\nabla\cdot(\nabla\times\v A)=0,
\end{equation}
that aid us in the construction of potential fields in electrodynamics.
Related integral theorems of Gauss and Stokes are
\begin{equation}
\int_{\Omega_{3}}\nabla\cdot\v D\,\d V=\int_{\partial\Omega_{3}}\v D\cdot\d\v S,\quad\int_{\Omega_{2}}(\nabla\times\v D)\cdot\d\v S=\int_{\partial\Omega_{2}}\v D\cdot\d\v l,
\end{equation}
where $\Omega_{3}$ and $\Omega_{2}$ are 3- and 2-dimensional domains,
and $\Omega_{3},\Omega_{2}$ their respective boundaries with surface
normal $\d\v S$ and line element $\d\v l$ being infinitesimal vectors.

\subsubsection*{Why we are dissatisfied}

Transforming differential operators to curvilinear coordinates is
complicated and we are limited to normalized basis vectors in orthogonal
coordinate systems.

\subsection{Graduate level physics}

Still the focus of tensor calculus in curvilinear geometry is on the
components of a tensor in a certain basis. This is reflected in the
statement that a tensor is a ``matrix with transformation properties
under coordinate changes''.

\subsubsection*{Why we are dissatisfied}

The fundamental building block of the co-contravariant treatment in
classical physics is still based on a position vector field $\v r(x,y,z)$
given in Cartesian coordinates. In general relativity the absence
of such a vector field confuses matter further, since tensors are
now only given in terms of their components fulfilling transformation
properties ``out of thin air'', without a global reference frame.
While we know what quantities are better represented in co- or contravariant
form, there is no clear distinction between the two.

\subsection{Differential geometry}

One of the features of a mathematical description is the exact specification
of what objects we are dealing with, and how to trace them back to
already known mathematical objects, going down to fundamental axioms
that are chosen to make intuitive concepts work, while giving them
a solid foundation. Similarly the connection to the physical world
relies on postulates. In contrast to axioms they can be justified
by the correctness of predictions made by the theory constructed in
terms of these objects. As common in mathematical descriptions, differential
geometry strives to be economic with additional structure, such as
metrizability or differentiability, requiring only the bare minimum
for the according task. On first sight this looks like this means
departing from reality and application, but the contrary is the case.
The mathematical approach is more honest in the sense that we do not
claim to know what our system actually is, but rather distill out
of it the bare minimum properties we require it to have in order to
make sense. This allows us to separate topological from metric features
and gives a clear guideline on structure-preserving discretizations
such as symplectic time-stepping schemes and vector finite elements.
Here we go back to treating vectors in their geometrical sense rather
than abstract vectors, and give a more precise definition to what
that means.

\subsubsection*{Space}

Here we describe the mathematical foundation to describe physical
space, which applies also to spacetime and phase-space with slight
modifications. In the philosophy of the bare minimum we start with
modeling space as a \emph{set} $\mathbb{M}$ of points $X\in\mathbb{M}$.
Our goal is to add a sufficient amount of mathematical structure to
this set to be able to describe motion and fields in different sets
of coordinates and formulate dynamical laws as differential equations.
\begin{enumerate}
\item Since we would like to know what points are adjacent to each other,
we require a topology $\mathcal{O}$ that yields the notion of \emph{open
neighbourhoods} containing adjacent points. This introduces the concept
of \emph{continuous maps} between topological spaces that keep neighbouring
points adjacent to each other also in their image. Adding this structure
to $\mathbb{M}$ makes it a \emph{topological space $(\mathbb{M},\mathcal{O})$.}
\item In order to be able to translate points to numbers, we require $\mathbb{M}$
to be at least locally similar to Euclidian space $\mathbb{R}^{d}$
of dimension $d$ equiped with the usual open sets as a topology based
on the Euclidian distance. This means that for each open set of points
$\Omega\subset\mathbb{M}$ we can find a continuous and invertible
map to $\mathbb{R}^{d}$ which yields points as functions $X(\xset)$
of $d$ coordinates $\xset=(x^{1},\dots,x^{d})$. Such a map is called
a \emph{chart.} A collection of charts is called an \emph{atlas }$\mathcal{A}$
that contains at least one chart for each $\Omega$. With $d$ the
minimum required chart dimension\footnote{We cannot chart the surface of the earth in $\mathbb{R}^{1}$, and
$\mathbb{R}^{3}$ would be too much, so we use $\mathbb{R}^{2}$} we call $(\mathbb{M},\mathcal{O},\mathcal{A})$ a $d$\emph{-dimensional
topological manifold}.
\item To define what differentiation means on the manifold, we require the
\emph{transition maps} representing coordinate transformations between
regions where charts overlap to be differentiable in $\mathbb{R}^{d}$.
It can be proven that already requiring $C^{1}$ differentiability
in an atlas induces a $C^{\infty}$ atlas as a subset of the original
atlas. This is why we call a manifold with a differentiable atlas
a \emph{smooth manifold}.
\item If we would like to measure distances on $\mathbb{M}$, we introduce
a \emph{metric} $g:\mathbb{M}\times\mathbb{M}\rightarrow[0,\infty)$
that assigns a distance to each pair of points that remains identical
when the arguments are swapped, is zero if and only if they are the
same point, and fulfills the triangle inequality. A manifold with
a metric is called a \emph{Riemannian manifold}\footnote{The concept can be generalized to a \emph{pseudo-Riemannian} manifold
for spacetime allowing negative ''distances''} $(\mathbb{M},\mathcal{O},\mathcal{A},g)$.
\end{enumerate}
Depending on the required features we can define mathematical objects
that represent physical quantities at different structural levels.
Most of the time we require a smooth manifold after step 3. If we
can avoid the use of the metric structure in step 4 to a bare minimum
we can gain insight into laws of physics and guidance for discretized
numerical modelling. This will imply keeping vectors and covectors
as distinct objects rather than converting between the two using the
metric tensor.

\subsubsection*{Vectors and covectors}

\textbf{TODO: mention curves}

On a $d$-dimensional smooth manifold we can introduce \emph{vectors}
$\v v\in T_{X}\mathbb{M}$ of the \emph{tangent space }$T_{X}\mathbb{M}$.
Intuitively $T_{X}\mathbb{M}$ can be imagined as the tangent plane
to a curved surface representing the manifold. The subscript $X$
in $T_{X}\mathbb{M}$ indicates that a vector $\v v$ only has a meaning
with respect to a specific point $X$, since $T_{X}\mathbb{M}$ is
generally not the same at different points. A further abstraction
concerns the basis of the tangent space. While we are used to a geometrical
definition based on Cartesian coordinates, the mathematical definition
is more abstract and based on a question what a vector can \emph{do}.
This is closely related to the high-school definition of a vector
$\v v$, defining it as an arrow pointing from point $X_{A}$ to $X_{B}$.
In fact we are asking for the difference of a scalar field $f$ if
we go along $\v v$ from $X_{A}$ to $X_{B}$. Now we are in a situation
that allows to do this only in an \emph{infinitesimal neighbourhood}
of $X$. This provides the \emph{directional derivative along} $\v v$
of a scalar field $f(X)$ in a certain chart. In classical tensor
calculus one would write this expression as
\begin{equation}
(\v v\cdot\nabla)f=\left(v^{i}\v e_{i}\cdot\v e^{k}\frac{\partial}{\partial x^{k}}\right)f=\left(v^{i}\delta_{i}^{k}\frac{\partial}{\partial x^{k}}\right)f=\left(v^{k}\frac{\partial}{\partial x^{k}}\right)f=v^{k}\frac{\partial f}{\partial x^{k}}.
\end{equation}
Here we have put brackets to avoid interpretation as a gradient, which
will later be associated to covectors. We realize that we do not require
geometrical basis vectors 
\begin{equation}
\v e_{k}=\frac{\partial\v r}{\partial x^{k}},
\end{equation}
since they cancel out. We can rather introduce an abstract basis 
\begin{equation}
\v e_{k}\equiv\frac{\v{\partial}}{\v{\partial}x^{k}}\equiv\v{\partial}_{k}.
\end{equation}
where we have stripped away the reference to the ``vector'' $\v r$
that has bugged us in tensor calculus, and wrote the $\v{\partial}$
symbols in bold face to identify $\v{\partial}/\v{\partial}x^{k}$
as a vector and distinguish it from a usual partial differential operator.
Vectors are represented by
\begin{equation}
\v v=v^{k}\v{\partial}_{k},
\end{equation}
This means that now vectors can actually \emph{act} on scalar fields
with the usual gradient in any chart,
\begin{equation}
\v v[f(X)]\equiv v^{k}\frac{\partial f}{\partial x^{k}}.
\end{equation}
Basis vectors $\v{\partial}_{k}$ acting on $f$ yield directional
derivatives
\begin{equation}
\v{\partial}_{k}[f]=\frac{\partial f}{\partial x^{k}}
\end{equation}
 along curves in direction of $x^{k}$.

The generalization of the inner product between vectors leads to the
concept of \emph{covectors }or \emph{one-forms} $\v w\in T_{X}^{\star}\mathbb{M}$.
Here $T_{X}^{\star}\mathbb{M}$ is called the \emph{cotangent space
}at $X$ and a covectors form the space of \emph{linear operators}
that can be \emph{applied} to a vector $\v v$ to yield a scalar.
Gradients of scalar fields yield covectors, which can be seen in the
classical definition of a scalar differential
\begin{equation}
\d f=\left(\frac{\partial f}{\partial x^{i}}\v e^{i}\right)\cdot\left(\v e_{k}\d x^{k}\right)=\frac{\partial f}{\partial x^{i}}\delta_{k}^{i}\d x^{k}=\frac{\partial f}{\partial x^{k}}\d x^{k}.
\end{equation}
containing components of the gradient $\nabla f$ as coefficients.
Now instead of defining 
\begin{equation}
\v e^{i}=\nabla x^{i}
\end{equation}
we again ``strip away'' the reference to $\v r$ hidden in the gradient
and directly define 
\begin{equation}
\v e^{i}\equiv\v dx^{i}
\end{equation}
via differentials. Covectors are expanded as 
\begin{equation}
\v w=w_{i}\v dx^{i}.
\end{equation}

A maybe surprising consequence of the separation into vectors and
covectors is the fact that there exists no inner product \emph{between}
the two. Instead a covector $\v w$ can \emph{act }on a vector $\v v$
as 
\begin{equation}
\v w[\v v]=w_{i}v^{i},
\end{equation}
yielding the same scalar result independently from the chosen coordinate
chart. In particular
\begin{equation}
\v dx^{i}\left[\v{\partial}_{k}\right]=\delta_{k}^{i}
\end{equation}
remains valid for basis vectors. When skipping brackets and formally
writing $\v w[\v v]=\v w\cdot\v v$ as a scalar product one should
write the covector $\v w$ first. Vectors and covectors thus exist
as distinct mathematical objects, even though $T_{X}\mathbb{M}$ and
$T_{X}^{\star}\mathbb{M}$ are isomorphic as finite-dimensional vector
spaces of equal dimension $d$. A visual representation of basis vectors,
covectors and 2-forms is available in Fig.~\ref{fig:manifold}.

\begin{figure}
\begin{centering}
\includegraphics[scale=1.2]{fig/manifold2}
\par\end{centering}
\caption{Local basis $\protect\v{\partial}/\protect\v{\partial}\protect\ph$
for vectors, $\protect\v d\protect\ph$ for covectors, and $\protect\v d\theta\wedge\protect\v d\protect\ph$
for 2-forms. While vectors can be interpreted as arrows \emph{along}
a curve, covectors appear as fluxes \emph{across} a level hypersurface
$\protect\ph=\mathrm{const}$ in direction of the gradient of $\protect\ph(X)$.
In $2D$, hypersurface coincide with curves, and the 2-form provides
the area as a $2D$ volume element. \label{fig:manifold}}
\end{figure}


\subsubsection*{Tensor fields and forms}

Starting from the base manifold $\mathbb{M}$ and \textquotedbl attaching\textquotedbl{}
the tangent space $T_{X}\mathbb{M}$ in each points $X$ (actually
it just means taking the disjoint union ``$\bigsqcup$'' of all
tangent spaces) yields the $2N$-dimensional \emph{tangent bundle}\textbf{\emph{
\begin{equation}
T\mathbb{M}\equiv\bigsqcup_{X\in\mathbb{M}}T_{X}\mathbb{M}
\end{equation}
}}whose elements are tuples $(X,\v v)$ of possible combinations of
a position $X\in\mathbb{M}$ and a vector $\v v\in T_{X}\mathbb{M}$
tangent to $\mathbb{M}$ at $X$. If $\v v$ has a specific functional
dependency on $X$ in these tuples our first try would be to just
define $\v v(X)$ as a vector-valued field on $\mathbb{M}$. Since
the vector space $T_{X}\mathbb{M}$ where $\v v(X)$ points to depends
on the argument $X$, we require a more technical construction. We
define a \emph{vector field }
\begin{align}
\v V: & \mathbb{M}\rightarrow T\mathbb{M},\nonumber \\
 & X\mapsto(X,\v v(X))
\end{align}
that keeps the information of $X$ in the target $(X,\v v(X))\in T\mathbb{M}$.
Such a map is called a \emph{section} of \textbf{\emph{$T\mathbb{M}$}}
and is the analogue to the graph $(x,y(x))$ of a function $y(x)$.
Usually we ignore the technical detail of writing $(X,\v v(X))$,
since $X$ appears as an argument anyway.

Similarly, attaching cotangent spaces $T_{X}^{\star}\mathbb{M}$ in
each point on $\mathbb{M}$ results in the \emph{cotangent bundle
}$T^{\star}\mathbb{M}$ containing pairs $(X,\v w)$. \emph{Covector
fields} $\v W:\mathbb{M}\rightarrow T^{\star}\mathbb{M}$ are specified
via sections $(X,\v w(X))$ of the cotangent bundle $T^{\star}\mathbb{M}$.
Similar to vectors and covectors, a covector field acts on a vector
field via
\begin{equation}
\v w(X)[\v v(X)]=w_{i}(X)v^{i}(X),
\end{equation}
where the point $X$ appears as an additional argument and can be
specified in any suitable set of coordinates $\xset$ in a chart via
the chart map $X(\xset)$. We are only interested in sufficiently
smooth vector fields, where components $v^{i}(\xset)\equiv v^{i}(X(\xset))$
and $w_{i}(\xset)\equiv w_{i}(X(\xset))$ are sufficiently smooth
maps from $\mathbb{R}^{d}$ to $\mathbb{R}^{d}$. Note that up to
now we didn't require a metric tensor. \textbf{TODO: mention tensor
fields and forms}

\begin{table}
\caption{Maps and their properties}

\centering{}%
\begin{tabular}{|c|c|c|c|c|}
\hline 
Map & from & to & space & remark\tabularnewline
\hline 
\hline 
Scalar field $\Phi$ & $\mathbb{R}$ & $\mathbb{M}$ & $C^{\infty}$ & \tabularnewline
\hline 
Vector $\v v$ & $C^{\infty}$ & $\mathbb{R}$ & $T_{X}\mathbb{M}$ & \tabularnewline
\hline 
Covector $\v w$ & $T_{X}\mathbb{M}$ & $\mathbb{R}$ & $T_{X}^{\star}\mathbb{M}$ & linear\tabularnewline
\hline 
Vector field $(X,\v v(X))$ & $\mathbb{M}$ & $T\mathbb{M}$ & $\Gamma(T\mathbb{M})$ & $\v v(X)$ is a vector in $T_{X}\mathbb{M}$\tabularnewline
\hline 
Covector field $(X,\v w(X))$ & $\mathbb{M}$ & $T^{\star}\mathbb{M}$ & $\Gamma(T^{\star}\mathbb{M})$ & $\v w(X)$ is a covector in $T_{X}^{\star}\mathbb{M}$\tabularnewline
\hline 
\end{tabular}
\end{table}


\subsubsection*{The metric tensor}

On manifolds with a metric $g$, the \emph{metric tensor} $\mathbf{g}$
provides a natural isomorphism to between vectors and covectors. More
precisely, $\t g$ is a symmetric $(0,2)$-tensor field (that can
depend on position $\v q$) on $Q$ with components $g_{ik}=g_{ik}(\v q)$
in a certain coordinate chart. The action of $\mathbf{g}$ on two
vectors is the commonly known \emph{inner product 
\begin{equation}
\v u\cdot\v v\equiv\mathbf{g}(\v u,\v v)=\mathbf{g}(\v v,\v u)=u^{i}g_{ik}v^{k}=v^{i}g_{ik}u^{k},\label{eq:guv}
\end{equation}
}which is a \emph{symmetric} \emph{bilinear} form, i.e. linear in
both arguments $\v u$ and $\v v$ which can be swapped. Similarly
the inner product in cotangent space
\begin{equation}
\v p\cdot\v w\equiv\bar{\mathbf{g}}(\v p,\v w)=\bar{\mathbf{g}}(\v w,\v p)=p_{j}\bar{g}^{jl}w_{l}=w_{j}\bar{g}^{jl}p_{l}
\end{equation}
is provided via the \emph{reciprocal metric} $\bar{\mathbf{g}}$ being
a symmetric $(2,0)$-tensor field on $Q$ defined by
\begin{equation}
\bar{g}^{ij}g_{jk}=\delta_{k}^{i}\equiv\begin{cases}
0 & \text{for }i\neq k,\\
1 & \text{for\,}i=k
\end{cases}
\end{equation}
This means that the component matrix $(\bar{g}^{jl})=(g_{ik})^{-1}$
is the inverse matrix of the metric tensor component matrix in any
set of coordinates. It is important to emphasize that $\mathbf{g}$
and $\bar{\mathbf{g}}$ are not the same tensor field, even though
the common notation $g^{jl}$ without a bar would suggest this. Since
$u_{k}\equiv u^{i}g_{ik}$ are components of a covector dual to $\v u$
and $w^{l}\equiv w_{j}\bar{g}^{jl}$ the ones of a vector dual to
$\v w$, one can ``convert'' between vectors (contravariant representation)
and covectors (covariant representation) by raising and lowering indexes
via the (inverse) metric. This allows the reduction to only vectors
as fundamental objects, as used in traditional tensor calculus. If
quantities are represented in their ``natural'' way in the traditional
formalism, i.e. contravariant components for tangent vectors, the
two and covariant components for gradient ``vectors'', formulas
become equivalent to the differential geometrical variant, since the
metric tensor disappears. As soon as divergence and curl operators
enter, a description using either differential forms or tensor densities
is required to be able to work in a metric-independent way.

\section{Mechanics}

In classical mechanics the phase-space $\mathbb{M}$ is a $2N$-dimensional
symplectic manifold constructed from combination of positions $Q$
being elements of an $N$-dimensional configuration manifold $\mathbb{Q}$
with their respective momenta as covectors $\v p$ in the $N$-dimensional
cotangent space $T_{Q}^{\star}\mathbb{Q}$ at position $Q$. As defined
above $T_{Q}^{\star}\mathbb{Q}$ is the cotangent space dual to the
tangent space $T_{Q}\mathbb{Q}$, where the latter contains velocity
vectors $\dot{\v q}$. Here the notation $\dot{\v q}$ for velocity
is conventional \emph{a priori} and must not be identified with time
derivatives $\d Q(t)/\d t$ appearing only for a specific orbit curve
$Q(t)$. The \emph{Lagrangian} formalism in $(Q,\dot{\v q})$ is constructed
on the tangent bundle $T\mathbb{Q}$ of $\mathbb{Q}$, while the \emph{Hamiltonian
}formalism is stated on the cotangent bundle $T^{\star}\mathbb{Q}$.
In terms of notation the outlined construction leads to upper i.e.
contravariant index notation for generalized coordinates $q^{i}$
representing points $Q$ in a certain chart, and for velocity components
$\dot{q}^{i}$ in the according local chart basis of $T_{Q}\mathbb{Q}$
at $Q$. Lower indexes are used for $p_{j}$ being covariant components
of $\v p$ with respect to the local chart basis of $T_{Q}^{\star}\mathbb{Q}$.

\subsubsection*{General phase-space}

Phase-space $\mathbb{M}=T^{\star}\mathbb{Q}$ still allows for a clear
distinction of $Q$ and $\v p$ as different mathematical objects
and underlies the canonical \emph{Hamiltonian} formalism in $(Q,\v p)$.
For a more general non-canonical treatment it is however useful to
introduce points $Z\in\mathbb{M}$ independently from the underlying
canonical foundation. This makes the Hamiltonian formalism applicable
to phase-spaces $\mathbb{M}$ that haven't been constructed as a cotangent
bundle and in this sense more general than the Lagrangian formalism.
The Hamiltonian formalism in possibly non-canonical coordinates is
a special case of the even more general treatment via the \emph{phase-space
Lagrangian} on the $4N$-dimensional tangent bundle $T\mathbb{M}$
of $\mathbb{M}$, being the tangent bundle $T(T^{\star}\mathbb{Q})$
of the cotangent bundle of $\mathbb{Q}$ in classical mechanics. Elements
$(Z,\dot{\v z})$ of $T\mathbb{M}$ are treated analogously to $(Q,\dot{\v q})$
in the usual Lagrangian formalism. Thus, coordinates $z^{\alpha}$
are written with upper index $\alpha=1\dots2N$, ignoring the previous
separation in $q^{i}$ and $p_{j}$ being \emph{canonical} coordinates
for $\mathbb{M}$, but also individual coordinates for $\mathbb{Q}$
and $T^{\star}\mathbb{Q}$, respectively. Euler-Lagrange equations
of the phase-space Lagrangian $L_{\mathrm{ph}}(Z,\dot{\v z},t)$ correspond
to Hamiltonian equations of motion for Hamiltonian $H(Z,t)$ in possibly
non-canonical coordinates as long as the canonical representation
of $L_{\mathrm{ph}}$ is of the form 
\begin{equation}
L_{\mathrm{ph}}(Z,\dot{\v z},t)=p_{k}\dot{q}^{k}-H(Z,t).
\end{equation}
This means that no components $\dot{p}_{k}$ of momentum components
of tangent vectors $\dot{\v z}$ must appear in $L_{\mathrm{ph}}$.
This feature is known as the \emph{degeneracy} of the phase-space
Lagrangian and prevents the construction of a hypothetical higher-level
\textquotedbl phase-space-Hamiltonian\textquotedbl{} formalism on
$T^{\star}\mathbb{M}$ via a Legendre transform. A more elegant representation
of the phase-space Lagrangian is achieved by ``stripping away''
$\d t$ via multiplication and writing it as the Lagrangian one-form
\begin{equation}
\v{\theta}_{L}(Z,t)\equiv p_{k}\v dq^{k}-H(Z,t)\v dt.
\end{equation}
Since this one-form is now defined on the manifold $\mathbb{R}\times\mathbb{M}$
including $t\in\mathbb{R}$ we have added a bold differential $\v dt$.

\subsubsection*{Symplectic structure}

\textbf{TODO: check sign errors}

The 2-form $\v{\omega}$ is an \emph{antisymmetric} \emph{$(0,2)$}-tensor
field and can be seen as an antisymmetric analogue to the metric tensor
$\mathbf{g}$. While also a bilinear form, instead of the symmetric
relation in Eq.~(\ref{eq:guv}) it switches sign when argument vectors
$\v y,\v z$ are swapped, i.e.
\begin{equation}
\v{\omega}(\v y,\v z)=-\v{\omega}(\v z,\v y)=y^{\alpha}\omega_{\alpha\beta}z^{\beta}=-z^{\alpha}\omega_{\alpha\beta}y^{\beta}.
\end{equation}
In the context of Hamiltonian systems components $\omega_{jk}$ are
represented by the inverse matrix of the Poisson matrix $\Pi^{ij}$
representing the Poisson tensor $\v{\Pi}$ that appears in equations
of motion
\begin{equation}
\dot{z}^{\alpha}=\Pi^{\alpha\beta}\frac{\partial H}{\partial z^{\beta}}.\label{eq:zadot}
\end{equation}
One could say that $\v{\Pi}$ takes over the role of the inverse metric
tensor $\bar{\mathbf{g}}$ in the symplectic analogy. Before introducing
a particular set of coordinates we emphasize that it is important
to distinguish invariant objects such as $\v{\omega},\v{\Pi}$ from
their coordinate representation $\omega_{ij},\Pi^{ij}$. The fact
that it is possible to \emph{locally} find canonical coordinates in
any manifold $\mathbb{M}$ with a symplectic structure $\v{\omega}$
is known as \emph{Darboux's theorem}. This is the analogy to the possibility
to find a locally flat representation of a curved manifold with a
metric structure $\t g$, which is important in general relativity.

In any phase-space coordinate basis
\begin{equation}
\v e_{\alpha}\equiv\frac{\v{\partial}}{\v{\partial}z^{\alpha}}\equiv\v{\partial}_{\alpha}
\end{equation}
we can expand the Poisson tensor $\v{\Pi}$ as
\begin{equation}
\v{\Pi}=\Pi^{\alpha\beta}\v{\partial}_{\alpha}\otimes\v{\partial}_{\beta},
\end{equation}
using the Poisson matrix $(\Pi^{\alpha\beta})$. Since the second
unit vector $\v{\partial}_{\beta}$ can on a scalar field, in particular
on $H$, one can denote equations of motion (\ref{eq:zadot}) in a
coordinate free way via
\begin{equation}
\dot{\v z}[f]=\v{\Pi}[f,H].\label{eq:zadot-1}
\end{equation}
In canonical coordinates $z_{\mathrm{can}}^{\alpha}\equiv q^{i},p_{j}$
the Poisson matrix becomes the totally antisymmetric matrix
\begin{equation}
\Pi^{\alpha\beta}=J^{\alpha\beta}=\left(\begin{array}{cc}
0 & I\\
-I & 0
\end{array}\right)\quad\text{for}\,z_{\mathrm{can}}^{\alpha},
\end{equation}
where $I$ is the identity matrix of dimension $N$. Thus we can write
the identity
\begin{align}
\v{\Pi} & =\delta_{k}^{i}\frac{\v{\partial}}{\v{\partial}q^{i}}\otimes\frac{\v{\partial}}{\v{\partial}p_{k}}-\delta_{k}^{i}\frac{\v{\partial}}{\v{\partial}p_{k}}\otimes\frac{\v{\partial}}{\v{\partial}q^{i}}\nonumber \\
 & =\frac{\v{\partial}}{\v{\partial}q^{i}}\otimes\frac{\v{\partial}}{\v{\partial}p_{i}}-\frac{\v{\partial}}{\v{\partial}p_{i}}\otimes\frac{\v{\partial}}{\v{\partial}q^{i}}.
\end{align}
in any set of canonical coordinates, where sums over $i,k$ run from
$1$ to $N$. This finally recovers canonical equations of motion
\begin{equation}
\dot{q}^{i}=\frac{\partial H}{\partial p_{i}},\quad\dot{p}_{i}=\frac{\partial H}{\partial q^{i}}.
\end{equation}

Components of the symplectic form
\[
\v{\omega}=\omega_{\alpha\beta}\,\v dz^{\alpha}\wedge\v dz^{\beta}
\]
with respect to the dual basis
\begin{equation}
\v e^{\alpha}\equiv\v dz^{\alpha}
\end{equation}
are defined such that
\begin{equation}
\omega_{\alpha\gamma}\Pi^{\gamma\beta}=\delta_{\alpha}^{\beta}
\end{equation}
in any system of coordinates in phase-space. Components $\omega_{\alpha\beta}$
are sometimes called the Lagrange matrix\emph{ }which is the inverse
of the Poisson matrix $\Pi^{\alpha\beta}$. Showing the invariance
of the symplectic form $\v{\omega}$ under the numerical flow map
generated by an integrator is thus equivalent to showing the invariance
of $\v{\Pi}$. Multiplying equations of motion~(\ref{eq:zadot})
from the left via $\omega_{\alpha\beta}$ and renaming indexes yields
the equivalent system
\begin{equation}
\omega_{\alpha\beta}\dot{z}^{\beta}=\frac{\partial H}{\partial z^{\alpha}}.\label{eq:zadot-2}
\end{equation}

In particular in canonical coordinates we have
\begin{align}
\omega_{\alpha\beta} & =(\Pi^{\alpha\beta})^{-1}\nonumber \\
 & =(J^{\alpha\beta})^{-1}=\left(\begin{array}{cc}
0 & -I\\
I & 0
\end{array}\right)\quad\text{for}\,z_{\mathrm{can}}^{i}.\label{eq:cani}
\end{align}

We introduce the antisymmetric wedge product as a basis for 2-forms,
i.e. antisymmetric rank-2 tensor fields in covariant representation.
The space of $2$-forms can be seen as a subspace of the space of
all rank-2 tensor fields spanned by
\begin{equation}
\v dz^{i}\otimes\v dz^{j},\text{ i.e.}\,(\v dq^{i}\otimes\v dq^{j},\v dq^{i}\otimes\v dp_{j},\v dp_{i}\otimes\v dq^{j},\v dp_{i}\otimes\v dp_{j}),
\end{equation}
where the wedge notation allows only for antisymmetric tensor fields
via
\begin{equation}
\v dq^{i}\wedge\v dp_{j}\equiv\v dq^{i}\otimes\v dp_{j}-\v dp_{i}\otimes\v dq^{j}.
\end{equation}
 This means that the wedge product fulfills the antisymmetry relations
\begin{align}
\v dq^{i}\wedge\v dp_{j}=-\v dq^{j}\wedge\v dp_{i} & .
\end{align}
Finally, according to Eq.~(\ref{eq:cani}) we can express the symplectic
2-form $\v{\omega}$ via the sum over canonical wedge tuples,
\begin{equation}
\v{\omega}=-\delta_{i}^{j}\v dp_{i}\otimes\v dq^{j}+\delta_{j}^{i}\v dq^{i}\otimes\v dp_{j}=\delta_{j}^{i}\v dq^{i}\wedge\v dp_{j}=\v dq^{i}\wedge\v dp_{i}.
\end{equation}

\textbf{TODO:}

Variants of the special 2-form $\v J$ correspond to vectors in a
similar way as diagonal matrices being generalizations of the identity
matrix $\v I$ correspond to vectors containing only diagonal elements.
Make some connection to the two representations of $\v B$ as a vector
density and a 2-form. The idea comes from the incomplete eigendecomposition
where $\v{\lambda}_{m}\hat{V}_{m}=\hat{V}_{m}\hat{\Lambda}_{mm}=\hat{V}_{m}\hat{I}_{m}\v{\lambda}_{m}$

\section{Forms and densities}

or: \textquotedbl why vector density fields are the same as 2-forms
and what this means for the magnetic field\textquotedbl . The reason
for this section is the fact that physicists take divergence and curl
of vector fields, while mathematicians use them on differential forms.
Tensor densities provide a link between the two, yielding identical
coefficients of a 2-form and a contravariant vector density representation,
both used to describe the magnetic field $\v B$. Right now this description
requires already some understanding of forms, wedge product, and the
Hodge star operator. \textbf{TODO: explain more.}

More information on this issue can be found in ``CONTRAVARIANCE,
COVARIANCE, DENSITIES, AND ALL THAT: AN INFORMAL DISCUSSION ON TENSOR
CALCULUS'' by Chris Tiee, p.19 - densitization. In this sense, also
the Levi-Civita symbol appears as a tensor density.

In classical tensor analysis (see e.g. book \textquotedbl Flux coordinates
and magnetic field structure\textquotedbl{} of Callen/d'haeseleer)
the divergence operator acts on covariant components of a vector field
and yields a scalar field $D$,
\begin{equation}
F=\nabla\cdot\v B\equiv\frac{1}{\sqrt{g}}\frac{\partial}{\partial x^{k}}(\sqrt{g}B^{k}).
\end{equation}
This operator becomes metric-independent if we define a representation
of $D$ via a scalar density $\mathcal{D}$ and a representation of
$\v B$ via contravariant density component $\mathcal{B}^{k}$, defined
via
\begin{equation}
\mathcal{F}\equiv\sqrt{g}F,\quad\mathcal{B}^{k}\equiv\sqrt{g}B^{k}.
\end{equation}
Then we can write the divergence as
\begin{equation}
\mathcal{F}=\mathrm{div}\,\v B\equiv\sqrt{g}(\nabla\cdot\v B)=\frac{\partial}{\partial x^{k}}\mathcal{B}^{k}.
\end{equation}
In 3D space, the curl operator acts on covariant components of a vector
field $\v A$
\begin{equation}
B^{i}=(\nabla\times\v A)^{i}\equiv\frac{\varepsilon^{ijk}}{\sqrt{g}}\frac{\partial}{\partial x^{j}}A_{k}.
\end{equation}
Using contravariant density components we can write instead a metric-free
variant
\begin{equation}
\mathcal{B}^{i}=(\mathrm{curl}\,\v A)^{i}\equiv\sqrt{g}(\nabla\times\v A)^{i}=\varepsilon^{ijk}\frac{\partial}{\partial x^{j}}A_{k}.
\end{equation}
The divergence of a curl vanishes, since
\begin{align}
\mathrm{div}\,\mathrm{curl}\,\v A & =\frac{\partial}{\partial x^{i}}(\varepsilon^{ijk}\frac{\partial}{\partial x^{j}}A_{k})=\varepsilon^{ijk}\frac{\partial^{2}}{\partial x^{i}\partial x^{j}}A_{k}\nonumber \\
 & =-\varepsilon^{jik}\frac{\partial^{2}}{\partial x^{i}\partial x^{j}}A_{k}=-\varepsilon^{ijk}\frac{\partial^{2}}{\partial x^{j}\partial x^{i}}A_{k}\nonumber \\
 & =-\varepsilon^{ijk}\frac{\partial^{2}}{\partial x^{i}\partial x^{j}}A_{k}.\label{eq:zerodiv}
\end{align}
where we have first used the antisymmetric property of $\varepsilon^{ijk}$,
then swapped names of indices $i$ and $j$, and finally used the
symmetry of second derivatives to swap their order. The only way that
relation~(\ref{eq:zerodiv}) with both $+$ and $-$ in front of
the same term can be fulfilled, is if it is identically zero.

Now we are going to show the equivalence of contravariant density
representation to the representation of $\v B$ as a 2-form. We have
\begin{equation}
\v{\omega}_{B}=\frac{1}{2}B_{jk}\v dx^{j}\wedge\v dx^{k}=B_{12}\v dx^{1}\wedge\v dx^{2}+B_{23}\v dx^{2}\wedge\v dx^{3}+B_{31}\v dx^{3}\wedge\v dx^{1}.
\end{equation}

Taking a divergence of this form means to take an exterior derivative
\begin{align}
\v{\omega}_{D}=\v d\v{\omega}_{B} & =\frac{1}{2}\v d(B_{jk}\v dx^{j}\wedge\v dx^{k})=\frac{1}{2}\v dB_{jk}\wedge\v dx^{j}\wedge\v dx^{k}\nonumber \\
 & =\frac{1}{2}\frac{\partial B_{jk}}{\partial x^{i}}\v dx^{i}\wedge\v dx^{j}\wedge\v dx^{k}=\frac{1}{2}\varepsilon^{ijk}\frac{\partial B_{jk}}{\partial x^{i}}\v dx^{i}\wedge\v dx^{j}\wedge\v dx^{k}\nonumber \\
 & =\left(\frac{\partial B_{12}}{\partial x^{1}}+\frac{\partial B_{23}}{\partial x^{2}}+\frac{\partial B_{31}}{\partial x^{3}}\right)\v dx^{1}\wedge\v dx^{2}\wedge\v dx^{3}=\omega_{D}\v dx^{1}\wedge\v dx^{2}\wedge\v dx^{3}.
\end{align}
We see that coincides with the divergence defined above via density
components, if we identify
\begin{equation}
\mathcal{D}\equiv\omega_{D},\quad\mathcal{B}^{1}\equiv B_{23},\quad\mathcal{B}^{2}\equiv B_{31},\quad\mathcal{B}^{3}\equiv B_{12}.\label{eq:div}
\end{equation}
This is also consistent the form $\v{\omega}_{B}$ follows from the
curl of an 1-form $\v A$ which means
\begin{align}
\v{\omega}_{B} & =\v d\v A=\v d(A_{k}\v dx^{k})=\v dA_{k}\wedge\v dx^{k}=\frac{\partial A_{k}}{\partial x^{i}}\v dx^{i}\wedge\v dx^{k}\nonumber \\
 & =\left(\frac{\partial A_{2}}{\partial x^{1}}-\frac{\partial A_{1}}{\partial x^{2}}\right)\v dx^{1}\wedge\v dx^{2}+\left(\frac{\partial A_{3}}{\partial x^{2}}-\frac{\partial A_{2}}{\partial x^{3}}\right)\v dx^{2}\wedge\v dx^{3}+\left(\frac{\partial A_{1}}{\partial x^{3}}-\frac{\partial A_{3}}{\partial x^{1}}\right)\v dx^{3}\wedge\v dx^{1}\nonumber \\
 & =\varepsilon^{ijk}\frac{\partial}{\partial x^{j}}A_{k}\,\v dx^{j}\wedge\v dx^{k}.
\end{align}
We thus can again identify
\begin{equation}
B_{jk}=\varepsilon^{ijk}\frac{\partial}{\partial x^{j}}A_{k}=\mathcal{B}^{i}
\end{equation}
with density components of the vector field $\v B$ defined earlier,
and consistent with~(\ref{eq:div}) meaning in particular
\begin{align}
B_{12} & =\frac{\partial A_{2}}{\partial x^{1}}-\frac{\partial A_{1}}{\partial x^{2}}=\mathcal{B}^{3},\\
B_{23} & =\frac{\partial A_{3}}{\partial x^{2}}-\frac{\partial A_{2}}{\partial x^{3}}=\mathcal{B}^{1},\\
B_{31} & =\frac{\partial A_{1}}{\partial x^{3}}-\frac{\partial A_{3}}{\partial x^{1}}=\mathcal{B}^{2}.
\end{align}

We can do the same using the Hodge-star operator, see \href{https://unapologetic.wordpress.com/2011/10/13/the-divergence-operator/}{https://unapologetic.wordpress.com/2011/10/13/the-divergence-operator/},
\url{https://en.wikipedia.org/wiki/Hodge_star_operator}, \href{https://lamington.wordpress.com/2014/05/26/div-grad-curl-and-all-this/}{https://lamington.wordpress.com/2014/05/26/div-grad-curl-and-all-this/},
\href{http://www.thp.uni-koeln.de/gravitation/courses/rcii14/fwh.pdf}{http://www.thp.uni-koeln.de/gravitation/courses/rcii14/fwh.pdf}.
Claim: the usual divergence is
\begin{equation}
\nabla\cdot\v B=\star(\v d(\star(\v B^{\flat})).\label{eq:hodgestar}
\end{equation}

Take the Hodge-star operator for a $p$-form, 
\begin{equation}
\star\v{\omega}=\frac{\sqrt{g}}{(n-p)!p!}\varepsilon_{\alpha_{1}\dots\alpha_{p}\beta_{1}\dots\beta_{n-p}}g^{\alpha_{1}\gamma_{1}}\dots g^{\alpha_{p}\gamma_{p}}\omega_{\gamma_{1}\dots\gamma_{p}}\v dx^{\beta_{1}}\wedge\dots\wedge\v dx^{\beta_{n-p}}.
\end{equation}
Then we start with
\begin{equation}
\v B=B^{i}\frac{\partial}{\partial x^{i}}
\end{equation}
and represent as a 1-form
\begin{equation}
\v B^{\flat}=B_{k}\v dx^{k}=B^{i}g_{ik}\v dx^{k}.
\end{equation}
Then we apply the hodge star
\begin{align}
\star\v B^{\flat} & =\frac{\sqrt{g}}{2}\varepsilon_{lmn}g^{lk}B^{i}g_{ik}\v dx^{m}\wedge\v dx^{n}\nonumber \\
 & =\frac{\sqrt{g}}{2}\varepsilon_{lmn}\delta_{i}^{l}B^{i}\v dx^{m}\wedge\v dx^{n}\nonumber \\
 & =\frac{1}{2}\varepsilon_{lmn}\mathcal{B}^{l}\,\v dx^{m}\wedge\v dx^{n}\nonumber \\
 & =\mathcal{B}^{1}\,\v dx^{2}\wedge\v dx^{3}+\mathcal{B}^{2}\,\v dx^{3}\wedge\v dx^{1}+\mathcal{B}^{3}\,\v dx^{1}\wedge\v dx^{2}=\v{\omega}_{B}.
\end{align}
Here the metric and the inverse metric could be cancelled to a $\delta$
via contraction. Then we apply an exterior derivative, and finally
taking another hodge $\star$ yields a scalar field instead of a density
in (\ref{eq:hodgestar}). The same works for the curl (see \href{https://en.wikipedia.org/wiki/Curl_\%28mathematics\%29}{https://en.wikipedia.org/wiki/Curl\_(mathematics)}
) with
\begin{equation}
\nabla\times\v B=(\star(\v d(\v B^{\flat})))^{\sharp}.
\end{equation}

\pagebreak{}

\part{Draft}

\section{Kinetic theory in spacetime}
\begin{itemize}
\item Spatial density 
\begin{equation}
f(t,\zset)=\frac{1}{NV}\sum_{n}\int_{V}\d^{3}x\,\delta(\zset-\zset_{(n)}(t))
\end{equation}
 becomes ``event density'' 
\begin{equation}
f(t,\zset)=\frac{1}{NV}\int\d^{3}x\,\d t\,\delta(\zset-\zset_{(n)}(s))\delta(t-t_{(n)}(s))
\end{equation}
that looks formally the same, but allows for multiple ``times''
for each particle.
\item Particles interact via fields evaluated at $F(t_{(n)},\xset_{(n)})$
and depending on the overall configuration.
\item Collision operators can separate short-scale and long-scale interaction
by averaging.
\end{itemize}

\section{Mechanics}

To specify motion of our mechanical system we introduce a $4$-dimensional
spacetime manifold $Q$ with coordinates $\yset=(x^{0}\equiv t,x^{1},x^{2},x^{3})$.
Based on this we define a $4N$-dimensional configuration manifold
$Q$ with coordinates $\qset=(x_{(n)}^{0}\equiv t_{(n)},x_{(n)}^{1},x_{(n)}^{2},x_{(n)}^{3})$,
meaning ``particle $k$ was at position $x_{(n)}^{1},x_{(n)}^{2},x_{(n)}^{3}$
at time $t_{(n)}$''. The phase-spacetime manifold $M$ follows as
the $8N$-dimensional cotangent bundle $T^{\star}Q$ of $Q$. According
$4N$ canonical momentum coordinates are $\pset=(p_{0,\,(n)},p_{1,\,(n)},p_{2,\,(n)},p_{3,\,(n)})$.
\begin{equation}
S[\gamma]=\int_{\gamma}\Theta_{L}.\label{eq:S}
\end{equation}


\section{Conservation laws}
\begin{itemize}
\item Partial differential equations
\begin{equation}
\frac{\partial f}{\partial t}+\nabla_{z}f=s_{f}.
\end{equation}
Write this as
\begin{equation}
\partial_{\alpha}f=s_{f}
\end{equation}
with $\alpha=0\dots N$.
\item Ordinary differential equations
\begin{equation}
\frac{\d z^{i}}{\d t}=\Lambda^{ik}\partial_{k}H.
\end{equation}
Extended phase-space:
\begin{align*}
\frac{\d z^{0}}{\d s} & =\partial_{s}\\
\frac{\d p_{0}}{\d s} & =-\partial_{s}\mathcal{H}
\end{align*}
\item Eulerian
\begin{equation}
\frac{\partial a}{\partial t}+\nabla\cdot\Gamma_{a}=s_{a}.
\end{equation}
\item Lagrangian
\begin{equation}
\frac{Da}{Dt}=s_{a}.
\end{equation}
\end{itemize}

\section{Basics}
\begin{itemize}
\item What is the use of differential forms?
\begin{itemize}
\item We can write even more stuff coordinate-independent than with usual
co-contravariant notation.
\item The geometric structure of equations and how conservation laws follow
become apparent.
\item Discrete version tells us how to do numerics the ``right'' way.
\item A prime example are Maxwell's equations.
\end{itemize}
\item What is a vector?
\begin{itemize}
\item It points along a curve as a tangent.
\end{itemize}
\item What can we do with a vector?
\begin{itemize}
\item We can differentiate along its direction.
\end{itemize}
\item What is a covector (1-form)?
\begin{itemize}
\item A map that yields a scalar to each vector.
\item A quantity associated with a surface.
\end{itemize}
\item What can we do with a covector?
\begin{itemize}
\item Apply it to a vector. 
\item This yields a scalar product i.e. the angle between the tangent vector
and the surface normal.
\end{itemize}
\item What is the metric?
\begin{itemize}
\item It defines how lengths and distances work.
\end{itemize}
\item How can we use the metric?
\begin{itemize}
\item Get the covector to a vector that yields scalar product 1.
\item i.e. the plane whose normal vector coincides with a given vector.
\end{itemize}
\item What is an inner product?
\begin{itemize}
\item Apply $p$-form to vector and get a $p-1$-form.
\item But wait, didn't you say that a $p$-form acts on $p$ vectors?
\item Yes, we define our $p-1$-form such that it gives the same scalar
as the $p$-form when the vector is inserted at the first position.
\item Examples:
\begin{itemize}
\item $\v E\cdot\vec{v}=i_{v}\v E$
\item $\vec{v}\times\v B=i_{v}\v B$
\item $\rho\vec{v}=i_{v}\rho$
\end{itemize}
\end{itemize}
\item What is a Hodge operator?
\begin{itemize}
\item Associates a $n-p$-form to a $p$-form.
\end{itemize}
\end{itemize}

\section{How to do a Fourier expansion in curvilinear coordinates?}

\subsection{Simple cases}

In Cartesian coordinates the basis $\v e_{x},\v e_{y},\v e_{z}$ doesn't
depend on the coordinates and is orthonormal. Thus, physical, co-
and contravariant components of vector fields become one, and we can
write any vector field $\v B$ as a Fourier series in either up to
all coordinates, i.e.
\begin{align}
\v B & =B^{x}(x,y,z)\v e_{x}+B^{y}(x,y,z)\v e_{y}+B^{z}(x,y,z)\v e_{z}\\
 & =\sum_{k=-\infty}^{\infty}\left(B_{k}^{x}(x,y)\v e_{x}+B_{k}^{y}(x,y)\v e_{y}+B_{k}^{z}(x,y)\v e_{z}\right)e^{ikz}\\
 & =\sum_{\v k}(B_{\v k}^{x}\v e_{x}+B_{\v k}^{y}\v e_{y}+B_{\v k}^{z}\v e_{z})e^{i\v k\cdot\v x},
\end{align}
where also in the latter $\v k=(k_{x},k_{y},k_{z})$ all three indexes
run from $-\infty$ to $\infty$.

\subsection{General co- and contravariant case}

In any curvilinear coordinates $\v x=(x^{1},x^{2},x^{3})$ we can
write a Fourier series for a vector in contravariant components with
\begin{align*}
\v B & =B^{1}(\v x)\v e_{1}(\v x)+B^{2}(\v x)\v e_{2}(\v x)+B^{3}(\v x)\v e_{3}\\
 & =\sum_{n=-\infty}^{\infty}\left(B_{n}^{1}(x^{1},x^{2})\v e_{1}(\v x)+B_{n}^{2}(x^{1},x^{2})\v e_{2}(\v x)+B_{n}^{3}(x^{1},x^{2})\v e_{3}(\v x)\right)e^{inx^{3}}
\end{align*}
or for a vector $\v A$ in covariant components with
\begin{align*}
\v A & =A_{1}(\v x)\nabla x^{1}(\v x)+A_{2}(\v x)\nabla x^{2}(\v x)+A_{3}(\v x)\nabla x^{3}(\v x)\\
 & =\sum_{n=-\infty}^{\infty}\left(A_{1n}(x^{1},x^{2})\nabla x^{1}(\v x)+A_{2n}(x^{1},x^{2})\nabla x^{2}(\v x)+A_{3n}(x^{1},x^{2})\nabla x^{3}(\v x)\right)e^{inx^{3}}.
\end{align*}
Here we intentionally keep the full dependency on all coordinates
in the basis vectors $\v e_{k}$ and $\v e^{k}=\nabla x^{k}$ and
apply the Fourier expansion only on vector components. It makes a
difference whether one expands in co- or contravariant components.
E.g. a vector potential $\v A$ is naturally given in covariant form
such that its curl can be nicely separated in Fourier form (try it).
\end{document}
