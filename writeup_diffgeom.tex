%% LyX 2.3.0 created this file.  For more info, see http://www.lyx.org/.
%% Do not edit unless you really know what you are doing.
\documentclass[english,notitlepage]{revtex4-1}
\usepackage{tgpagella}
\usepackage[T1]{fontenc}
\usepackage[latin9]{inputenc}
\setcounter{secnumdepth}{3}
\usepackage{amsmath}
\usepackage{graphicx}
\usepackage{babel}
\begin{document}
\global\long\def\tht{\vartheta}
\global\long\def\ph{\varphi}
\global\long\def\balpha{\boldsymbol{\alpha}}
\global\long\def\btheta{\boldsymbol{\theta}}
\global\long\def\bJ{\boldsymbol{J}}
\global\long\def\bGamma{\boldsymbol{\Gamma}}
\global\long\def\bOmega{\boldsymbol{\Omega}}
\global\long\def\d{\text{d}}
\global\long\def\t#1{\text{#1}}
\global\long\def\m{\text{m}}
\global\long\def\v#1{\boldsymbol{#1}}

\global\long\def\t#1{\mathbf{#1}}


\title{Differential geometry from the perspective of tensor calculus}

\author{Christopher Albert}

\affiliation{Max-Planck-Institut f�r Plasmaphysik, Boltzmannstra�e 2, 85748 Garching,
Germany}

\maketitle
In this text a short explanation on the treatment of space as a differentiable
manifold and phase-space as a symplectic manifold for readers acquainted
to terminology from tensor calculus in co-/contravariant notation
is given. A main feature distinguishing differential geometry from
tensor calculus is the treatment of covectors as first-class objects
in addition to vectors. As a concequence in many derivations no metric
tensor has to be specified, and the construction of structure-preserving
numerical methods becomes more simple and transparent. Apart from
conceptual differences, the practical use of the formalisms is similar,
in particular the notation with upper and lower indexes for contravariant
and covariant components and the convention to sum over indexes appearing
once up and once down.

In classical mechanics phase-space $M$ is a $2N$-dimensional symplectic
manifold constructed as a combination of positions $\v q$ in an $N$-dimensional
configuration manifold $Q$ and momenta as covectors $\v p$ in the
according $N$-dimensional \emph{cotangent space} $T_{\v q}^{\star}Q$.
The latter is the dual space to the \emph{tangent space }$T_{\v q}Q$
containing velocity vectors $\dot{\v q}$. Here the notation $\dot{\v q}$
for velocity vectors is conventional \emph{a priori} and must not
be identified with time derivatives $\dot{\v q}(t)=\frac{\d\v q(t)}{\d t}$
appearing only for a specific orbit curve $\v q(t)$. Below we will
use the symbol $\v v$ for vectors in general. The subscript $\v q$
in $T_{\v q}Q$ and $T_{\v q}^{\star}Q$ indicates that vectors $\dot{\v q}$
and covectors $\v p$ only have a meaning with respect to a specific
point $\v q$, i.e. vectors on a curved manifold where the tangent
space $T_{\v q}Q$ depends on $\v q$ require an origin point.

In terms of notation the outlined construction leads to upper i.e.
contravariant index notation for generalized coordinates $q^{i}$
representing points $\v q$ in a certain chart, and for velocity components
$\dot{q}^{i}$ in the according local chart basis of $T_{\v q}Q$
at $\v q$. Lower indexes are used for $p_{j}$ being covariant components
of $\v p$ with respect to the local chart basis of $T_{\v q}^{\star}Q$.
A maybe surprising consequence of the separation into vectors and
covectors is the fact that there exists no inner product \emph{between}
the two. Instead a covector $\v p$ can \emph{act }on a vector $\v v$
as a \emph{linear form}
\begin{equation}
\v p(\v v)=p_{i}v^{i},
\end{equation}
yielding the same scalar result independently from the chosen coordinate
chart. Vectors and covectors thus exist as distinct mathematical objects,
even though $T_{\v q}Q$ and $T_{\v q}^{\star}Q$ are isomorphic as
finite-dimensional vector spaces of equal dimension $N$. A \emph{metric
tensor} $\mathbf{g}$ can fix the structure of the manifold to provide
such an isomorphism. More precisely, $\t g$ is a symmetric $(0,2)$-tensor
field (i.e. depending on position $\v q$) on $Q$ with components
$g_{ik}=g_{ik}(\v q)$ in a certain coordinate chart. The action of
$\mathbf{g}$ on two vectors is the commonly known \emph{inner product
\begin{equation}
\v u\cdot\v v\equiv\mathbf{g}(\v u,\v v)=\mathbf{g}(\v v,\v u)=u^{i}g_{ik}v^{k}=v^{i}g_{ik}u^{k},\label{eq:guv}
\end{equation}
}which is a \emph{symmetric} \emph{bilinear} form, i.e. linear in
both arguments $\v u$ and $\v v$ which can be swapped. Similarly
the inner product in cotangent space
\begin{equation}
\v p\cdot\v w\equiv\bar{\mathbf{g}}(\v p,\v w)=\bar{\mathbf{g}}(\v w,\v p)=p_{j}\bar{g}^{jl}w_{l}=w_{j}\bar{g}^{jl}p_{l}
\end{equation}
is provided via the \emph{reciprocal metric} $\bar{\mathbf{g}}$ being
a symmetric $(2,0)$-tensor field on $Q$ defined by
\begin{equation}
\bar{g}^{ij}g_{jk}=\delta_{k}^{i}\equiv\begin{cases}
0 & \text{for }i\neq k,\\
1 & \text{for\,}i=k
\end{cases}
\end{equation}
This means that the component matrix $\bar{g}^{jl}=(g_{ik})^{-1}$
is the inverse matrix of the metric tensor component matrix in any
set of coordinates. It is important to emphasize that $\mathbf{g}$
and $\bar{\mathbf{g}}$ are not the same tensor field, even though
the common notation $g^{jl}$ without a bar would suggest this. Since
$u_{k}\equiv u^{i}g_{ik}$ are components of a covector dual to $\v u$
and $p^{l}\equiv p_{j}\bar{g}^{jl}$ the ones of a vector dual to
$\v p$, one can ``convert'' between vectors (contravariant representation)
and covectors (covariant representation) by raising and lowering indexes
via the (inverse) metric. This allows the reduction to only vectors
as fundamental objects, as used in traditional tensor calculus. If
quantities are represented in their ``natural'' way in the traditional
formalism, i.e. contravariant components for tangent vectors, the
two and covariant components for gradient ``vectors'', formulas
become equivalent to the differential geometrical variant, since the
metric tensor disappears. As soon as divergence and curl operators
enter, a description using either differential forms or tensor densities
is required to be able to work in a metric-independent way.

\subsubsection*{Basis for vectors and covectors}

\begin{figure}
\includegraphics[scale=1.2]{fig/manifold2}

\caption{TODO\label{fig:manifold}}
\end{figure}

TODO: describe $\frac{\v{\partial}}{\v{\partial}q^{i}}$ and $\v dq^{j}$
replacing $\v e_{i}=\frac{\partial\v r}{\partial q^{i}}$ and $\v e^{j}=\nabla q^{j}=\frac{\partial q^{j}}{\partial\v r}$
from usual co/contravariant tensor analysis.

\subsubsection*{Tangent and cotangent bundle}

Merging configuration space $Q$ with tangent spaces $T_{\v q}Q$
in all points $\v q$ yields the $2N$-dimensional \emph{tangent bundle}\textbf{\emph{
$TQ$}} whose elements are tuples $(\v q,\dot{\v q})$ of positions
and velocities. If $\dot{\v q}$ have a specific functional dependency
on $\v q$ in these tuples it is called a \emph{vector field, }with
the graph $(\v q,\dot{\v q}(\v q))$ denoted as a \emph{section} of
\textbf{\emph{$TQ$}}. The \emph{Lagrangian} formalism in $(\v q,\dot{\v q})$
is constructed on the tangent bundle $TQ$ of $Q$. Similarly combination
of $Q$ with cotangent spaces $T_{\v q}Q$ results in the \emph{cotangent
bundle }$T^{\star}Q$ containing pairs $(\v q,\v p)$ and where covector
fields can be specified. The resulting phase-space $M=T^{\star}Q$
still allows for a clear distinction of $\v q$ and $\v p$ as different
mathematical objects and underlies the canonical \emph{Hamiltonian}
formalism in $(\v q,\v p)$. For a more general non-canonical treatment
it is however useful to introduce points $\v z\in M$ independently
from the underlying canonical foundation. This makes the Hamiltonian
formalism available to phase-spaces $M$ that haven't been constructed
as a cotangent bundle and in this sense more general than the Lagrangian
formalism. The non-canonical Hamiltonian formalism is a special case
of the even more general treatment via the \emph{phase-space Lagrangian}
on the $4N$-dimensional tangent bundle $TM$ of $M$, being the tangent
bundle $T(T^{\star}Q)$ of the cotangent bundle of $Q$ in classical
mechanics. Elements $(\v z,\dot{\v z})$ of $TM$ are treated analogously
to $(\v q,\dot{\v q})$ in the usual Lagrangian formalism. Thus, coordinates
$z^{\alpha}$ are written with upper index $\alpha=1\dots2N$, ignoring
the previous separation in $q^{i}$ and $p_{j}$ being \emph{canonical}
coordinates for $M$, but also for $Q$ and $T^{\star}Q$ individually.
Euler-Lagrange equations of the phase-space Lagrangian $L_{\mathrm{ph}}(\v z,\dot{\v z},t)$
correspond to non-canonical Hamiltonian equations of motion for Hamiltonian
$H(\v z,t)$ as long as the canonical representation of $L_{\mathrm{ph}}$
is of the form 
\begin{equation}
L_{\mathrm{ph}}(\v z,\dot{\v z},t)=p_{k}\dot{q}^{k}-H(\v z,t).
\end{equation}
This means that no components $\dot{p}_{k}$ of momentum components
of tangent vectors $\dot{\v z}$ must appear in $L_{\mathrm{ph}}$.
This feature is known as the \emph{degeneracy} of the phase-space
Lagrangian and prevents the construction of a higher-level Hamiltonian
formalism on $T^{\star}M$ via a Legendre transform.

\subsubsection*{Symplectic structure}

The 2-form $\v{\omega}$ is an \emph{antisymmetric} \emph{$(0,2)$}-tensor
field and can be seen as an antisymmetric analogue to the metric tensor
$\mathbf{g}$. While also a bilinear form, instead of the symmetric
relation in Eq.~(\ref{eq:guv}) it switches sign when argument vectors
$\v y,\v z$ are swapped, i.e.
\begin{equation}
\v{\omega}(\v y,\v z)=-\v{\omega}(\v z,\v y)=y^{\alpha}\omega_{\alpha\beta}z^{\beta}=-z^{\alpha}\omega_{\alpha\beta}y^{\beta}.
\end{equation}
In the context of Hamiltonian systems components $\omega_{jk}$ are
represented by the inverse matrix of the Poisson matrix $\Pi^{ij}$
representing the Poisson tensor $\v{\Pi}$ that appears in equations
of motion
\begin{equation}
\dot{z}^{\alpha}=\Pi^{\alpha\beta}\frac{\partial H}{\partial z^{\beta}}.\label{eq:zadot}
\end{equation}
One could say that $\v{\Pi}$ takes over the role of the inverse metric
tensor $\bar{\mathbf{g}}$ in the symplectic analogy. Before introducing
a particular set of coordinates we emphasize that it is important
to distinguish invariant objects such as $\v{\omega},\v{\Pi}$ from
their coordinate representation $\omega_{ij},\Pi^{ij}$. The fact
that it is possible to \emph{locally} find canonical coordinates in
any manifold $M$ with a symplectic structure $\v{\omega}$ is known
as \emph{Darboux's theorem}. This is the analogy to the possibility
to find a locally flat representation of a curved manifold with a
metric structure $\t g$, which is important in general relativity.

In any phase-space coordinate basis
\begin{equation}
\v e_{\alpha}\equiv\frac{\v{\partial}}{\v{\partial}z^{\alpha}}
\end{equation}
we can expand the Poisson tensor $\v{\Pi}$ as
\begin{equation}
\v{\Pi}=\Pi^{\alpha\beta}\frac{\v{\partial}}{\v{\partial}z^{\alpha}}\otimes\frac{\v{\partial}}{\v{\partial}z^{\beta}},
\end{equation}
using the Poisson matrix $\Pi^{\alpha\beta}$. Since the second unit
vector $\frac{\v{\partial}}{\v{\partial}z^{\beta}}$ is meant to act
on a scalar field such as $H$ one can denote equations of motion
(\ref{eq:zadot}) in a coordinate free way via
\begin{equation}
\dot{\v z}=\v{\Pi}H.\label{eq:zadot-1}
\end{equation}
In canonical coordinates $z_{\mathrm{can}}^{\alpha}\equiv q^{i},p_{j}$
the Poisson matrix becomes the totally antisymmetric matrix
\begin{equation}
\Pi^{\alpha\beta}=J^{\alpha\beta}=\left(\begin{array}{cc}
0 & I\\
-I & 0
\end{array}\right)\quad\text{for}\,z_{\mathrm{can}}^{\alpha},
\end{equation}
where $I$ is the identity matrix of dimension $N$. Thus we can write
the identity
\begin{align}
\v{\Pi} & =\delta_{k}^{i}\frac{\v{\partial}}{\v{\partial}q^{i}}\otimes\frac{\v{\partial}}{\v{\partial}p_{k}}-\delta_{k}^{i}\frac{\v{\partial}}{\v{\partial}p_{k}}\otimes\frac{\v{\partial}}{\v{\partial}q^{i}}\nonumber \\
 & =\frac{\v{\partial}}{\v{\partial}q^{i}}\otimes\frac{\v{\partial}}{\v{\partial}p_{i}}-\frac{\v{\partial}}{\v{\partial}p_{i}}\otimes\frac{\v{\partial}}{\v{\partial}q^{i}}.
\end{align}
in any set of canonical coordinates, where sums over $i,k$ run from
$1$ to $N$. This finally recovers canonical equations of motion
\begin{equation}
\dot{q}^{i}=\frac{\partial H}{\partial p_{i}},\quad\dot{p}_{i}=\frac{\partial H}{\partial q^{i}}.
\end{equation}

Components of the symplectic form
\[
\v{\omega}=\omega_{\alpha\beta}\,\v dz^{\alpha}\wedge\v dz^{\beta}
\]
with respect to the dual basis
\begin{equation}
\v e^{\alpha}\equiv\v dz^{\alpha}
\end{equation}
are defined such that
\begin{equation}
\omega_{\alpha\gamma}\Pi^{\gamma\beta}=\delta_{\alpha}^{\beta}
\end{equation}
in any system of coordinates in phase-space. Components $\omega_{\alpha\beta}$
are sometimes called the Lagrange matrix\emph{ }which is the inverse
of the Poisson matrix $\Pi^{\alpha\beta}$. Showing the invariance
of the symplectic form $\v{\omega}$ under the numerical flow map
generated by an integrator is thus equivalent to showing the invariance
of $\v{\Pi}$. Multiplying equations of motion~(\ref{eq:zadot})
from the left via $\omega_{\alpha\beta}$ and renaming indexes yields
the equivalent system
\begin{equation}
\omega_{\alpha\beta}\dot{z}^{\beta}=\frac{\partial H}{\partial z^{\alpha}}.\label{eq:zadot-2}
\end{equation}

In particular in canonical coordinates we have
\begin{align}
\omega_{\alpha\beta} & =(\Pi^{\alpha\beta})^{-1}\nonumber \\
 & =(J^{\alpha\beta})^{-1}=\left(\begin{array}{cc}
0 & -I\\
I & 0
\end{array}\right)\quad\text{for}\,z_{\mathrm{can}}^{i}.\label{eq:cani}
\end{align}

We introduce the antisymmetric wedge product as a basis for 2-forms,
i.e. antisymmetric rank-2 tensor fields in covariant representation.
The space of $2$-forms can be seen as a subspace of the space of
all rank-2 tensor fields spanned by
\begin{equation}
\v dz^{i}\otimes\v dz^{j},\text{ i.e.}\,(\v dq^{i}\otimes\v dq^{j},\v dq^{i}\otimes\v dp_{j},\v dp_{i}\otimes\v dq^{j},\v dp_{i}\otimes\v dp_{j}),
\end{equation}
where the wedge notation allows only for antisymmetric tensor fields
via
\begin{equation}
\v dq^{i}\wedge\v dp_{j}\equiv\v dq^{i}\otimes\v dp_{j}-\v dp_{i}\otimes\v dq^{j}.
\end{equation}
 This means that the wedge product fulfills the antisymmetry relations
\begin{align}
\v dq^{i}\wedge\v dp_{j}=-\v dq^{j}\wedge\v dp_{i} & .
\end{align}
Finally, according to Eq.~(\ref{eq:cani}) we can express the symplectic
2-form $\v{\omega}$ via the sum over canonical wedge tuples,
\begin{equation}
\v{\omega}=-\delta_{i}^{j}\v dp_{i}\otimes\v dq^{j}+\delta_{j}^{i}\v dq^{i}\otimes\v dp_{j}=\delta_{j}^{i}\v dq^{i}\wedge\v dp_{j}=\v dq^{i}\wedge\v dp_{i}.
\end{equation}

\end{document}
