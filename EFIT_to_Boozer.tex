\documentclass[12pt]{article}
%\sloppy 
\textheight24.7cm                                                              
\textwidth18.0cm                                                               
\topmargin-1.5cm                                                               
\oddsidemargin-1.0cm                                                            
\evensidemargin0.0cm
\usepackage{times}
\usepackage{amsmath}
%\renewcommand{\baselinestretch}{2}
\newcommand{\be}[1]{\begin{equation} \label{#1}}
\newcommand{\ee}{\end{equation}}
\newcommand{\bea}[1]{\begin{eqnarray} \label{#1}}
\newcommand{\eea}{\end{eqnarray}}
\newcommand{\bean}{\begin{eqnarray*}}
\newcommand{\eean}{\end{eqnarray*}}

\newcommand{\non}{\nonumber\\}
\newcommand{\eq}[1]{(\ref{#1})}
\newcommand{\difp}[2]{\frac{\partial #1}{\partial #2}}
\newcommand{\br}{{\bf r}}
\newcommand{\bR}{{\bf R}}
\newcommand{\bA}{{\bf A}}
\newcommand{\bB}{{\bf B}}
\newcommand{\bE}{{\bf E}}
\newcommand{\bm}{{\bf m}}
\newcommand{\bn}{{\bf n}}
\newcommand{\bN}{{\bf N}}
\newcommand{\bp}{{\bf p}}
\newcommand{\bF}{{\bf F}}
\newcommand{\bz}{{\bf z}}
\newcommand{\bZ}{{\bf Z}}
\newcommand{\bV}{{\bf V}}
\newcommand{\bv}{{\bf v}}
\newcommand{\bu}{{\bf u}}
\newcommand{\bx}{{\bf x}}
\newcommand{\bX}{{\bf X}}
\newcommand{\bJ}{{\bf J}}
\newcommand{\bj}{{\bf j}}
\newcommand{\bk}{{\bf k}}
\newcommand{\bTheta}{{\bf \Theta}}
\newcommand{\btheta}{{\boldsymbol\theta}}
\newcommand{\bOmega}{{\bf \Omega}}
\newcommand{\bomega}{{\boldsymbol\omega}}
\newcommand{\brho}{{\boldsymbol\rho}}
\newcommand{\re}{{\rm e}}
\newcommand{\rd}{{\rm d}}
\newcommand{\rJ}{{\rm J}}
\newcommand{\ph}{{\varphi}}
\newcommand{\te}{\theta}
\newcommand{\tht}{\vartheta}
\newcommand{\vpar}{v_\parallel}
\newcommand{\vparkb}{v_{\parallel k b}}
\newcommand{\vparkm}{v_{\parallel k m}}
\newcommand{\Jpar}{J_\parallel}
\newcommand{\ppar}{p_\parallel}
\newcommand{\Bpstar}{B_\parallel^*}
\newcommand{\intpi}{\int\limits_{0}^{2\pi}}
\newcommand{\summ}{\sum \limits_{m=-\infty}^\infty}
\newcommand{\tb}{\tau_b(\uv)}
\newcommand{\bh}{{\bf h}}
\newcommand{\cE}{{\cal E}}
\newcommand{\cB}{{\cal B}}
\newcommand{\cg}{{\cal G}}
\newcommand{\odtwo}[2]{\frac{\rd #1}{\rd #2}}
\newcommand{\pdone}[1]{\frac{\partial}{\partial #1}}
\newcommand{\pdtwo}[2]{\frac{\partial #1}{\partial #2}}
\newcommand{\ds}{\displaystyle}
%\newcommand{\bc}{\begin{center} 
%\newcommand{\ec}{\end{center}}
%\input{prespan}

\begin{document}

\section{Generation of Boozer file from EFIT output}

\subsection{Magnetic field from EFIT data}
\label{ssec:efitdata}

Code EFIT provides axisymmetric equilibrium field which can be written as a sum of poloidal and toroidal field as follows,
\be{axifield}
\bB=\nabla\times\left(A_\varphi\nabla\varphi\right)+B_\varphi\nabla\varphi,
\ee
where the toroidal covariant vector-potential component $A_\varphi=\psi$ is the poloidal flux specified in the nodes of rectangular grid
in cylindrical coordinates, $\psi=\psi(R,Z)$, and the toroidal covariant magnetic field component $B_\varphi=B_\varphi(\psi)$ is a flux function
equal up to a factor to a poloidal current, $B_\varphi= 2 I_{\rm pol} c^{-1}$.
Interface to EFIT data, ``field\_divB0'', uses 2D splines of 5-th order to interpolate $\psi$ and provides physical cylindrical components of 
the magnetic field as functions of cylindrical variables $(R,\varphi,Z)$ according to~\eq{axifield},
\be{cylcomps}
\hat B_R = -\frac{1}{R}\difp{\psi}{Z},
\qquad
\hat B_Z = \frac{1}{R}\difp{\psi}{R},
\qquad
\hat B_\varphi = \frac{B_\varphi}{R}.
\ee
These components as well as their derivatives over all coordinates are available via formal arguments of subroutine ``field''. 
In addition, function $\psi$ and its first and second derivatives are available via module ``field\_eq\_mod''.

\subsection{Symmetry flux coordinates by field line integration}
Equations of field line in cylindrical coordinates are
\bea{flint_cyl}
\frac{\rd R}{\rd \varphi}=\frac{B^R}{B^\varphi}=\frac{R\hat B_R}{\hat B_\varphi},
\nonumber \\
\frac{\rd Z}{\rd \varphi}=\frac{B^Z}{B^\varphi}=\frac{R\hat B_Z}{\hat B_\varphi},
\eea
while the same equations in symmetry (and any other straight field line) flux coordinates are
\bea{flint_sym}
\frac{\rd r}{\rd \varphi}=\frac{B^r}{B^\varphi}=0,
\nonumber \\
\frac{\rd \vartheta}{\rd \varphi}=\frac{B^\vartheta}{B^\varphi}=\iota,
\eea
where $\iota=\iota(r)$. Integrating Eqs.~\eq{flint_cyl} over one poloidal turn what requires finding the periodic boundary 
upon integrating over $\varphi$ from 0 to this boundary, $\varphi_{\rm max}$, we obtain safety factor $q=\iota^{-1}$ as
$q=\varphi_{\rm max}(2\pi)^{-1}$.
Repeating the integration with a fixed step $\Delta\varphi=2\pi q K^{-1}$ where $K$ is the number of steps we obtain
the data on equi-distant grid of symmetry flux angle $\vartheta$ ranging in $[0, 2\pi]$, i.e. a set of array elements
$(\vartheta_k, R_k, Z_k, B_k, \sqrt{g}_k)$ belonging to the same point $\vartheta_k = 2\pi k K^{-1}$.
Metric determinant here is obtained from
\be{divB}
\nabla\cdot\bB=0 \qquad \Rightarrow\qquad \difp{}{\vartheta}\sqrt{g}B^\vartheta + \difp{}{\varphi}\sqrt{g}B^\varphi
=\difp{}{\vartheta}\sqrt{g}B^\vartheta = \iota \difp{}{\vartheta}\sqrt{g}B^\varphi=0
\ee
as
\be{sqrtg_sym}
\sqrt{g} = \frac{C_s}{B^\varphi}=\frac{C_s R}{\hat B_\varphi},
\ee
where constant $C_s=C_s(r)$ is a flux function. The value of this constant is not important since it cancels in neoclassical averages.
Note that Eq.~\eq{sqrtg_sym} is valid for general 3D field geometries.

\noindent
Repeating the integration for a set of flux surfaces $r=r_j$ we obtain the data on the rectangular grid equidistant in $\vartheta$.
This data is interpolated over $\vartheta$ with help of periodic splines and over $r$ with help of Lagrange polynomials.
Thus, continuous functions $R=R(r,\vartheta)$, $Z=Z(r,\vartheta)$, $B=B(r,\vartheta)$ and $\sqrt{g}=\sqrt{g(r,\vartheta)}$
are available everywhere in plasma volume together with their derivatives computed from splines and Lagrange polynomials.



\subsection{Boozer transformation function from field line integration}
\label{ssec:boozerdata}

For Boozer coordinates radial variable $r$ is unchanged.
Then, the Jacobian the Jacobian $\sqrt{g}$ of symmetry flux coordinates $(r,\vartheta,\varphi)$
is expressed through the Jacobian $\sqrt{g_B}$ of Boozer coordinates $(r,\vartheta_B,\varphi_B)$
as follows
\be{jacs}
\sqrt{g} = \difp{(\vartheta_B,\varphi_B)}{(\vartheta,\varphi)}\sqrt{g_B} \equiv J_B \sqrt{g_B}.
\ee
Substituting in~\eq{jacs} Boozer angles from direct transformation via transformation function $G=G(r,\vartheta,\varphi)$,
\be{invtrans}
\vartheta_B = \vartheta+\iota G,
\qquad
\varphi_B = \varphi+G,
\ee
magnetic differential equation for $G$ is obtained as
\be{magdifG}
\frac{\rd G}{\rd \varphi} \equiv \frac{1}{B^\varphi}\bB\cdot\nabla G \equiv \iota \difp{G}{\vartheta}+\difp{G}{\varphi}=\frac{\sqrt{g}}{\sqrt{g_B}}-1
=J_B-1.
\ee
The Jacobian of Boozer coordinates for the particular choice of radial variable $r=\psi_{\rm tor}$ is
\be{jacbooz}
\sqrt{g_B}=\frac{\iota B_\vartheta + B_\varphi}{B^2},
\ee
where $B_\vartheta=B_\vartheta(r)$ and $B_\varphi=B_\varphi(r)$ are flux functions. Therefore, for arbitrary choice of $r$ this Jacobian is
\be{jacboozarb}
\sqrt{g_B}=\frac{C_B}{B^2},
\ee
with $C_B=C_B(r)$ being a flux function too. Substituting~\eq{jacboozarb} and~\eq{sqrtg_sym} in~\eq{magdifG} this equation is transformed to
\be{magdifG_expl}
\frac{\rd G}{\rd \varphi} =\frac{R B^2}{C \hat B_\varphi}-1,
\ee
where $C=C_B C_s^{-1}$ is also a flux function. Particular expression for this function is not needed because it can be evaluated from the condition
that transformation function $G$ is a single-valued function on the flux surface,
\be{singval}
\lim_{L_\varphi \rightarrow \infty}\frac{1}{L_\varphi}\int\limits_0^{L_\varphi} \rd\varphi\left(\frac{R B^2}{C \hat B_\varphi}-1\right)
= \left\langle B^\varphi\left(\frac{R B^2}{C\hat B_\varphi}-1\right)\right\rangle \left\langle B^\varphi \right\rangle^{-1} = 0.
\ee
In axisymmetric field where $G=G(r,\vartheta)$ it is sufficient to set the upper integration limit over $\varphi$ not to infinity
but to $L_\varphi=\varphi_{\rm max}=2\pi q$ what results in
\be{Cdef}
C = \frac{1}{2\pi q} \int\limits_0^{2\pi q} \rd\varphi \frac{R B^2}{\hat B_\varphi}.
\ee
Thus, constant $C$ can be determined during the first field line integration together with safety factor $q$.
Second integration of field line equations~\eq{flint_cyl} extended by Eq.~\eq{magdifG_expl} will result in additional quantity $G_k$ 
in the set of array elements $(\vartheta_k, R_k, Z_k, B_k, \sqrt{g}_k, G_k)$ such that subsequent 2D interpolation provides $G$ continuously 
everywhere together with necessary derivatives.

\noindent
This primary information from field line integration is already sufficient for computation of mod-B spectrum, $B_{mn}$, which for axisymmetric field
contains finite amplitudes only for $n=0$,
\be{Bm0}
B_{m0}=\frac{1}{2\pi}\int\limits_0^{2\pi} \rd\vartheta_B B \re^{-i m \vartheta_B}
=\frac{1}{2\pi}\int\limits_0^{2\pi} \rd\vartheta \left(1+\iota \difp{G}{\vartheta}\right) B \re^{-i m \left(\vartheta+\iota G\right)}.
\ee
Here, one can express derivative of $G$ via~\eq{magdifG_expl} which taking into account~\eq{magdifG} and axisymmetry of $G$ results in
the explicit form of the transformation Jacobian from symmetry flux to Boozer coordinates,
\be{transfactor}
J_B=1+\iota \difp{G}{\vartheta}=\frac{R B^2}{C \hat B_\varphi}.
\ee
Integrals~\eq{Bm0} of periodic sub-integrand are efficiently evaluated on an equidistant grid of $\vartheta$,
\be{thetak}
\vartheta_k = \frac{2\pi k}{K}, \qquad k=1,2,\dots,K.
\ee
Denoting values of different function at the nodes of the grid as follows,
\be{fnodes}
J_B^{(k)} = \frac{R(r,\vartheta_k) B^2(r,\vartheta_k)}{C(r)\hat B_\varphi(r,\vartheta_k)},
\qquad
B_{(k)} = B(r,\vartheta_k),
\qquad
\cE_{(k)} = \exp\left(- i \left(\vartheta_k + \iota(r)G(r,\vartheta_k)\right)\right),
\ee
Fourier amplitudes~\eq{Bm0} are obtained as
\be{Bm0num}
B_{m0}=\lim_{K\rightarrow\infty}\frac{1}{K}\sum\limits_{k=1}^K J_B^{(k)} B_{(k)} \cE_{(k)}^m.
\ee

\subsection{Covariant field components}
\label{ssec:covar}

For computation of covariant field components in Boozer coordinates 
%(actually we need only $B_r^B$ but $B_\vartheta^B$ and $B_\varphi^B$ are computed here for
%a formula check) 
we use the fact that we know cylindrical coordinates $R$ and $Z$ as functions of symmetry flux coordinates $\bx=(r,\vartheta,\varphi)$
everywhere together with their derivatives over $x^i$. Thus, we can assume that we know the covariant unit vectors 
$\partial \br / \partial x^i$ of symmetry flux coordinates since transformation from cylindrical to Cartesian coordinates is straightforward.

\noindent
Radial covariant component is obtained as
\be{B_r}
B_r^B = \bB\cdot\left(\difp{\br}{r}\right)_B = \bB\cdot\difp{(\br,\vartheta_B,\varphi_B)}{(r,\vartheta_B,\varphi_B)}
= \bB\cdot\difp{(\br,\vartheta_B,\varphi_B)}{(r,\vartheta,\varphi)}\difp{(\vartheta,\varphi)}{(\vartheta_B,\varphi_B)}
= \bB\cdot\difp{(\br,\vartheta_B,\varphi_B)}{(r,\vartheta,\varphi)}\frac{1}{J_B}.
\ee
For more explicit expressions we use axial symmetry in order to obtain
\bea{B_r_expl}
B_r^B 
&=&
\bB\cdot\difp{\br}{r}-\bB\cdot\difp{\br}{\vartheta}\frac{1}{J_B}\difp{}{r}\iota G
-\bB\cdot\difp{\br}{\varphi}\frac{1}{J_B}\left(\difp{G}{r}-G\difp{\iota}{r}\difp{G}{\vartheta}\right)
\\
&=&
\hat B_R \difp{R}{r}+\hat B_Z \difp{Z}{r}
-\left(\hat B_R \difp{R}{\vartheta}+\hat B_Z \difp{Z}{\vartheta}\right)\frac{1}{J_B}\difp{}{r}\iota G
- \hat B_\varphi \frac{R}{J_B}\left(\difp{G}{r}-G\difp{\iota}{r}\difp{G}{\vartheta}\right),
\nonumber
\eea
where $J_B$ is given by~\eq{transfactor}. Cylindrical coordinates $R$ and $Z$ and function $G$ and their derivatives 
in the last expression~\eq{B_r_expl} are known from the interpolation of field line integration result. 
Physical cylindrical components of the magnetic field in this expression should be computed by ``field'' routine
for known $R$ and $Z$. Therefore, covariant radial component~\eq{B_r_expl} is known at equidistant grid
nodes $\vartheta_k$,
\be{B_r_nodes}
B_{r(k)}^B = B_r^B(r,\vartheta_k),
\ee
and its Fourier amplitudes can be computed in analogy to~\eq{Bm0num} as
\be{Brm0num}
B_{r,m0}=\lim_{K\rightarrow\infty}\frac{1}{K}\sum\limits_{k=1}^K J_B^{(k)} B_{r(k)}^B \cE_{(k)}^m.
\ee

\noindent
Component over poloidal angle is computed in a similar way to~\eq{B_r},
\be{B_theta}
B_\vartheta^B = \bB\cdot\left(\difp{\br}{\vartheta_B}\right)_B = \bB\cdot\difp{(r,\br,\varphi_B)}{(r,\vartheta_B,\varphi_B)}
= \bB\cdot\difp{(\br,\varphi_B)}{(\vartheta,\varphi)}\frac{1}{J_B}
= \bB\cdot\left(\difp{\br}{\vartheta}-\difp{G}{\vartheta}\difp{\br}{\varphi}\right)\frac{1}{J_B}.
\ee
Eliminating poloidal derivative of $G$ with help of~\eq{transfactor} and using then explicit expression for $J_B$ we get
\be{B_theta_expl}
B_\vartheta^B = \frac{1}{J_B}\left(B_\vartheta+\frac{B_\varphi}{\iota}\left(1-J_B\right)\right)
=\frac{\iota B_\vartheta+B_\varphi}{\iota J_B} - \frac{B_\varphi}{\iota}
=\frac{B^2}{\iota J_B B^\varphi} - \frac{B_\varphi}{\iota}
=q \left(C-B_\varphi\right).
\ee
Obviously $B_\vartheta^B$ is a flux function because both, $B_\varphi$ and constant $C$ are flux functions known already after the 
first field line integration. In order to see that this is a toroidal current we rewrite~\eq{Cdef} from the form of field line integral
to the form of integral over the poloidal angle using the axial symmetry,
\be{Cdef_pol}
C = \frac{1}{2\pi} \int\limits_0^{2\pi} \rd\vartheta \frac{R B^2}{\hat B_\varphi}
= \frac{1}{2\pi} \int\limits_0^{2\pi} \rd\vartheta \frac{B^2}{B^\varphi}.
\ee
Substituting this in the last expression~\eq{B_theta_expl} we get
\bea{torcurr}
B_\vartheta^B 
=\frac{q}{2\pi} \int\limits_0^{2\pi} \rd\vartheta \left(\frac{B^2}{B^\varphi}-B_\varphi\right)
=\frac{q}{2\pi} \int\limits_0^{2\pi} \rd\vartheta \frac{B_\vartheta B^\vartheta}{B^\varphi}
=\frac{1}{2\pi} \int\limits_0^{2\pi} \rd\vartheta B_\vartheta =\frac{2}{c} I_{\rm tor},
\eea
where the last expression follows from Ampere's law and Stokes theorem.

\noindent
We do not need to repeat the ansatz for the toroidal component because in axisymmetric field both flux coordinate systems
have the same toroidal covariant vector (since $G$ is independent of $\varphi$),
\be{bothsame}
\difp{\br}{\varphi_B}=\difp{\br}{\varphi}.
\ee
Respectively, toroidal covariant field component is also unchanged,
\be{B_phi}
B_\varphi^B = B_\varphi = R \hat B_\varphi=\frac{2}{c} I_{\rm pol}.
\ee

\end{document}
